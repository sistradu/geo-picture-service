\section{JSF}
\label{JumpJSF}
Schon immer wird in der Programmierung versucht bestehendes wiederzuverwenden. Dieser Ansatz findet sich auch in der Programmierung von Web-Applikationen. F�r die Programmierung von Web-Applikationen ist so ein Framework JSF. Dieses Framework basiert auf Servlets und JSPs und bietet eine gute M�glichkeit komfortabel Webseiten zu entwickeln.\\
\\
Nat�rlich gibt es noch weitere Frameworks, so dass die Frage im Raum steht warum in diesem Projekt JSF als Basis-Framework verwendet wird. Zum einem w�re hier zu nennen, dass JSF schon eine Weile auf dem Markt ist, was f�r die Programmierung einen gro�en Vorteil darstellt. Es geht nichts �ber eine gute Dokumentation eines Frameworks. Au�erdem gibt es im Internet fast immer eine L�sung f�r ein Problem, da fast immer jemand anderes vorher das Problem auch hatte. Durch diesen Vorteil konnte die Entwicklung beschleunigt werden.\\
\\
Ein weiterer Grund f�r die Verwendung von JSF ist die Anforderung des Auftraggebers. Dieser hat vorgeschlagen das Projekt mit JSF zu verwirklichen. Da dies aus oben genannten Gr�nden ein guter Vorschlag war, wurde JSF verwendet.\\
\\
In diesem Projekt wird die Version 2.0.1 von Mojarra verwendet. Sie bietet eine gute Unterst�tzung f�r Facelets und ist im Moment die aktuelle Version von JSF. Die Version von Mojarra bietet die gleichen Funktionalit�ten wie die Referenzimplementierung von Apache MyFaces. Der Grund f�r die Verwendung von Mojarra ist darin begr�ndet, dass die IDE bei der Entwicklung Mojarra als einzige M�glichkeit f�r JSF 2 vorgeschlagen hat. Da, wie bereits erw�hnt, keine Unterschiede zu anderen Implementierungen bestehen, haben wir diesen Vorschlag angenommen. Ein weiterer Grund liegt in der Verwendung von Primefaces. Wie in Abschnitt \ref{JumpPrimefacesTechnologien} beschrieben, verwenden wir Primefaces 2.2.1, diese ben�tigt unbedingt JSF 2 mit Facelets.