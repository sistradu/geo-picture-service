\chapter{Aufteilung des Projekts}
\label{JumpProjektaufteilung}
Das Projektteam umfasst 4 Mitglieder. Die Aufteilung der Aufgaben erfolgte nach dem Motto, wer hat zu was Lust? Anschlie�end wurden noch offene Themen so zugeordnet, dass die einzelnen Projektmitglieder gleich viel zu tun hatten. Der Hintergrund daf�r, warum dis so getan wurde, ist der, dass der Abschluss des Studiums kurz bevor steht und eine gewisse Spezialisierung bereits zu erkennen ist. Trotzdem wurde immer darauf geachtet, dass jeder Entwickler mindestens �berblickartig �ber die anderen Bereiche Bescheid wei�. Damit konnte erreicht werden, dass f�r komplizierte Umsetzungen zum Pair-Programming gewechselt werden konnte. Au�erdem hatte dies den Vorteil, dass es nicht nur einen Ansprechpartner f�r ein m�gliches aufgetretenes Problem zur Verf�gung stand. Der Hauptteil der Entwicklung erfolgte w�hrend festgelegten Zeiten in einem Raum. Dadurch konnten Anfragen schnell bearbeitet werden und wurden nicht durch eine Wartezeit unterbrochen. In der folgenden Tabelle sind die einzelnen Programmmodule aufgelistet und mit dem Namen des verantwortlichen Entwickler gekennzeichent.

\begin{table}[htbp]
	\centering
		\begin{tabular}{|l|l|}
			\hline
				Thema & Entwickler\\
				\hline\hline
				Funktionsklassen: CoordinateCalculator u. ImageDataExtractor & Stefan Radusch\\
				\hline
				DB-Mock-Klasse  & Markus Ullrich\\
				\hline
				YAML & Martin Schicht\\
				\hline
				Breadcrumb u. Men�  & Rico Scholz\\
				\hline
				Startseite mit Login und Nutzerverwaltung  & Stefan Radusch\\
				\hline
				Album erstellen/bearbeiten  & Stefan Radusch\\
				\hline
				Bild bearbeiten  & Martin Schicht\\
				\hline
				Album�bersicht  & Martin Schicht\\
				\hline
				eigene Alben anzeigen & Rico Scholz\\
				\hline
				Album betrachten  & Rico Scholz\\
				\hline
				DB-Anbindung  & Markus Ullrich\\
				\hline
			
		\end{tabular}
	\caption{Projektaufteilung}
	\label{tab:Projektaufteilung}
\end{table}
Die Belegaufteilung erfolgte entsprechend der entwickelten Teile im System. Die anderen Themen, die in dieser Arbeit beschrieben wurden, lassen sich in der nachfolgenden Tabelle mit dem Bearbeiter finden.
\begin{table}[htbp]
	\centering
		\begin{tabular}{|l|l|}
			\hline
				Kapitel/Abschnitt & Bearbeiter\\
				\hline\hline
				Abstract & Jeder\\
				\hline
				Einleitung & Rico Scholz\\
				\hline
				Problem/Aufgabenstellung & Rico Scholz\\
				\hline
				Anforderungsanalyse & Martin Schicht, Markus Ullrich\\
				\hline
				Drupal & Martin Schicht\\
				\hline
				Technologien/Primefaces & Stefan Radusch\\
				\hline
				Technologien/JSF & Stefan Radusch\\
				\hline
				Technologien/eXist & Markus Ullrich\\
				\hline
				Technologien/Yaml & Martin Schicht\\
				\hline
				Systementwurf & Martin Schicht, Markus Ullrich\\
				\hline
				Analysierte Technologien/Metadatenextraktion/Exif & Stefan Radusch\\
				\hline
				Analysierte Technologien/Webservices & Stefan Radusch\\
				\hline
				Analysierte Technologien/KML & Martin Schicht\\
				\hline
				Entwicklungsumgebung/Tomcat & Rico Scholz\\
				\hline
				Entwicklungsumgebung/Hudson & Rico Scholz\\
				\hline
				Entwicklungsumgebung/IntelliJ & Martin Schicht\\
				\hline
				Entwicklungsumgebung/Googlecode & Rico Scholz\\
				\hline
				Evaluierung Primefaces & Rico Scholz\\
				\hline
				Implementation/Datenbank & Markus Ullrich\\
				\hline
				Implementation/Applikation & Jeder\\
				\hline
				Implementation/Validatoren & Martin Schicht, Markus Ullrich\\
				\hline
				Implementation/Primefaces & Stefan Radusch\\
				\hline
				Implementation/Benutzeroberfl�che & Martin Schicht\\
				\hline
				Tests & Jeder\\
				\hline
				Aufteilung des Projekts & Jeder\\
				\hline
				Zusammenfassung & Jeder\\
				\hline
		\end{tabular}
	\caption{Belegaufteilung}
	\label{tab:Belegaufteilung}
\end{table}
