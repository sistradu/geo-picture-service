\section{Tomcat}
\label{JumpTomcat}
Der von der Hochschule bereitgestellte virtuelle Server lie� eine freie Konfiguration und Installation von Software zu. Die Eckdaten des Systems lauten: 1 vCPU (entspr. 1 Intel XEON 2,266 GHz), 1 GByte RAM und 8 GByte Festplattenspeicher. Um die Webseite im Internet zu pr�sentieren, wird ein Servlet-Container ben�tigt auf der die Webanwendung gestartet wird Da in dieser Arbeit eine Reihe neuer Technologien bereits genutzt wurden, sollte die Wahl auf ein bekanntes und schon mal verwendetes Programm fallen. Zur Auswahl standen somit Apache Tomcat und GlassFish. Da GlassFish ein JavaEE-Server und dadurch ein vollst�ndiger Application-Server ist, wurde dieser nicht gew�hlt. Es war abzusehen, dass f�r das Projekt keine der zus�tzlichen M�glichkeiten die so ein System im Gegensatz zu einem reinen Servlet-Container bietet genutzt werden sollte. Somit fiel die Wahl auf Apache Tomcat. Da dieser f�r andere Projekte bereits verwendet wurde und ein gewisses Ma� an Expertise in Umgang und Konfiguration vorhanden war, war diese Entscheidung nicht schwer zu treffen. Bei Projektstart wurde Version 6 verwendet, da dieser bereits auf dem virtuellen Server vorinstalliert war. Im Laufe des Projektes fand eine Migration auf die aktuelle Version 7 statt. Der gro�e Vorteil gegen�ber der Alten ist, dass man damit in der Lage ist aus einer XHTML-Datei heraus Methoden eines Controllers aufzurufen und gleichzeitig Parameter zu �bergeben. Dadurch konnte ein Problem behoben werden, welches in Verbindung mit Primefaces auftrat, dass Parameter die �ber den Tag "`param"' aus der Taglibrary JSF-Core, gar nicht oder erst bei einem wiederholten Request, �bergeben wurde. Dies wird im Kapitel Implementation noch n�her beleuchtet.\newline\newline
Im Allgemeinen kann gesagt werden, dass die Entscheidung f�r Apache Tomcat sehr gut war. Es sollte jedoch darauf geachtet werden, dass nur absolut notwendige Dinge in den Logs ausgegeben werden, da diese eine beachtliche Gr��e annehmen k�nnen. So waren Logdateien von mehreren Gigabyte vorhanden, die die Festplattenausnutzung fast an 100\% brachten.
