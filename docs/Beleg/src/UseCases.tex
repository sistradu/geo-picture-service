\section{Usecases}
\label{JumpUseCases}
\begin{table}
	\begin{center}
			\begin{tabular}{|p{3.5cm}|p{9cm}|}
				\hline
				\textbf{Name} & Nutzer-Login \\
				\hline
				\textbf{Umfang} & Nutzer \\
				\hline
				\textbf{Ebene} & Startseite \\
				\hline
				\textbf{Prim�r-Actor} & Nutzer \\
				\hline
				\textbf{Stakeholder und Interessen} & Nutzer: m�chte sich einloggen \\
				\hline
				\textbf{Vorbedingungen} & Nutzer besitzt ein Konto \\
				\hline
				\textbf{Nachbedingungen} & Nutzer ist angemeldet \\
				\hline
				\textbf{Standard-Szenario} & \begin{enumerate} \item Nutzer tr�gt Daten in Login-Felder ein. \item Nutzer klickt auf OK. \item Nutzerspezifische Optionen werden angezeigt. \end{enumerate} \\
				\hline
				\textbf{Erweiterungen} & \begin{enumerate} \item[2a] Login-Daten sind falsch eingegeben. \end{enumerate} \textit{Wiederhole 1. - 2. bis erfolgreicher Login.} \\
				\hline
				\textbf{Spezielle Anforderungen} & Passwort wird verschl�sselt �bertragen \\
				\hline
			\end{tabular}
		\caption{Usecase : Nutzerlogin}
		\label{JumpUsecaseNutzerlogin}
	\end{center}
\end{table}

\begin{table}
	\begin{center}
			\begin{tabular}{|p{3.5cm}|p{9cm}|}
				\hline
				\textbf{Name} & Nutzer l�schen \\
				\hline
				\textbf{Umfang} & Nutzer \\
				\hline
				\textbf{Ebene} & Admin-Page \\
				\hline
				\textbf{Prim�r-Actor} & Admin \\
				\hline
				\textbf{Stakeholder und Interessen} & \begin{itemize} \item Admin: m�chte/soll Nutzer aus System entfernen \item Nutzer: m�chte/soll entfernt werden \end{itemize} \\
				\hline
				\textbf{Vorbedingungen} & \begin{itemize} \item Nutzer besitzt ein Konto \item Admin ist angemeldet \end{itemize} \\
				\hline
				\textbf{Nachbedingungen} & Nutzerkonto ist gel�scht \\
				\hline
				\textbf{Standard-Szenario} & \begin{enumerate} \item Admin ruft Nutzeradministration auf. \item Admin w�hlt f�r den Nutzer die Option L�schen. \item Nutzer wird nicht mehr in Liste angezeigt. \end{enumerate} \\
				\hline
				\textbf{Erweiterungen} & keine \\
				\hline
			\end{tabular}
		\caption{Usecase : Nutzer l�schen}
		\label{JumpUsecaseNutzerLoeschen}
	\end{center}
\end{table}

\begin{table}
	\begin{center}
			\begin{tabular}{|p{3.5cm}|p{9cm}|}
				\hline
				\textbf{Name} & Album anlegen \\
				\hline
				\textbf{Umfang} & Album \\
				\hline
				\textbf{Ebene} & Nutzer-Frontend \\
				\hline
				\textbf{Prim�r-Actor} & Nutzer \\
				\hline
				\textbf{Stakeholder und Interessen} & Nutzer: m�chte ein neues Album anlegen \\
				\hline
				\textbf{Vorbedingungen} & Nutzer ist angemeldet. \\
				\hline
				\textbf{Nachbedingungen} & Ein neues Album wurde angelegt. \\
				\hline
				\textbf{Standard-Szenario} & \begin{enumerate} \item Nutzer klickt auf 'Neues Album'. \item Nutzer tr�gt Albuminformationen ein. \item Nutzer klickt auf OK. \item Das neue Album wird angezeigt. \end{enumerate} \\
				\hline
				\textbf{Erweiterungen} & \begin{enumerate} \item[2a] Nutzer l�dt zus�tzlich Bilder f�r das Album hoch (\nameref{JumpUsecaseBildHochladen}) \item[3a] Ein Album mit demselben Namen existiert bereits. \end{enumerate} \textit{Bei 3a: Wiederhole 2 bis Albumname noch nicht vorhanden.} \\
				\hline
			\end{tabular}
		\caption{Usecase : Album anlegen}
		\label{JumpUsecaseAlbumAnlegen}
	\end{center}
\end{table}

\begin{table}
	\begin{center}
			\begin{tabular}{|p{3.5cm}|p{9cm}|}
				\hline
				\textbf{Name} & Album bearbeiten \\
				\hline
				\textbf{Umfang} & Album \\
				\hline
				\textbf{Ebene} & Nutzer-Frontend \\
				\hline
				\textbf{Prim�r-Actor} & Nutzer \\
				\hline
				\textbf{Stakeholder und Interessen} & Nutzer: m�chte Album bearbeiten \\
				\hline
				\textbf{Vorbedingungen} & Nutzer ist angemeldet. \\
				\hline
				\textbf{Nachbedingungen} & Das Album wurde ge�ndert. \\
				\hline
				\textbf{Standard-Szenario} & \begin{enumerate} \item Nutzer klickt auf das Album \item Nutzer klickt auf 'Bearbeiten'. \item Nutzer tr�gt neue Albuminformationen ein. \item Nutzer klickt auf OK. \item Das bearbeitete Album wird angezeigt. \end{enumerate} \\
				\hline
				\textbf{Erweiterungen} & \begin{enumerate} \item[3a] Nutzer l�dt zus�tzlich Bilder f�r das Album hoch. (\nameref{JumpUsecaseBildHochladen}) \item[3b] Nutzer l�scht bereits vorhandene Bilder (\nameref{JumpUsecaseBildL�schen}) \item[4a] Ein Album mit demselben Namen existiert bereits. \end{enumerate} \textit{Bei 4a: Wiederhole 3 bis Albumname noch nicht vorhanden.} \\
				\hline
			\end{tabular}
		\caption{Usecase : Album bearbeiten}
		\label{JumpUsecaseAlbumBearbeiten}
	\end{center}
\end{table}