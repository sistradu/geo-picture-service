\section{eXist}
\label{JumpeXist}
%eXist ist einfach und unkompliziert --> Daten hochladen, Ausf�hren von Abfragen --> keine Einbindung zus�tzlicher Frameworks f�r DB-Zugriff!!!
%Schnell eingerichtet, einfaches Prinzip.
%Schnelle Abfragen.
%Relationale DB-Systeme bestechen durch referentielle Integrit�t, gut durchdachtes Konzept, Jahrelanger Praxiseinsatz. Doch aufw�nfige Transformationen machen die VErwendung unhandlich, oder nur bestimmtes Format der XML-Dateien zul�ssig f�r universelle TRansformation, physische Struktur der XML-Dokumente wird zerst�rt (Kommentare auch, aber das macht uns nichts aus) --> physische Struktur schon wichtig, da z.B. Reihenfolge der Bilder in DB wichtig --> bereits sortiert, nach Zeitpunkt des Eintragens, Alben auch
%Ein zentraler Bestandteil unseres Fotoarchivs ist die Datenbank. Hier werden alle wichtigen Informationen zu Nutzern, Bildern und Alben gespeichert.
%Bei der Auswahl einer passenden Datenbank zur Verwaltung von XML-Dateien, gibt es zun�chst, wie unter \url{http://dbs.uni-leipzig.de/files/projekte/XML/paper/XMLDB\_IAOforum.pdf} nachzulesen ist, 2 gro�e Kategorien. Zum Einen gibt es klassische relationale Datenbanksysteme, die mit einer XML-Erweiterung ausgestattet sind. Das hei�t, sie sind in der Lage, XML-Dokumente in ihre interne Datenstruktur und umgekehrt zu transformieren. Da diese Systeme nicht vorrangig f�r die Speicherung von XML-Dokumenten konzipiert wurden, ergeben sich aus ihrem Einsatz verschiedene Nachteile: (Siehe: \url{http://wwwlgis.informatik.uni-kl.de/archiv/wwwdvs.informatik.uni-kl.de/courses/seminar/WS0203/folien5.pdf} und \url{http://www.markwiesemann.eu/download/proseminar-folien.pdf})
%\begin{itemize}
%	\item Die XML-Dokumente m�ssen ein bestimmtes Format besitzen
%	\item Verlust von physikalischer Struktur und Kommentaren
%	\item aufw�ndige Transformation der Daten
%\end{itemize}
%Das hei�t, relationale Datenbanksysteme sind f�r unseren Anwendungsfall eher ungeeignet.
%\begin{enumerate}
%	\item Native XML-Datenbanken
%	\item XML-enabled Datenbanken
%\end{enumerate}
%Bei XML-enabled Datenbanken handelt es sich um klassische realationale Datenbanksysteme, die ein Mapping der Daten ins XML-Format erlauben. Man spricht auch von einem datenorientiertem Ansatz.
Ein zentraler Bestandteil unseres Fotoarchivs ist die Datenbank. Hier werden alle wichtigen Informationen zu Nutzern, Bildern und Alben gespeichert.
Da wir XML-Dokumente verarbeiten, liegt es nahe, eine native XML-Datenbank zum Speichern der Daten zu verwenden. Dabei haben wir uns f�r eXist entschieden, was wir im Folgenden begr�nden wollen.
\subsection{Warum eXist?}
Und warum kein relationales Datenbank-Managementsystem?
Relationale Datenbanken basieren auf einem gut durchdachtem Konzept, das sich jahrelang bew�hrt hat und bieten wichtige Vorz�ge, die man bei einer nativen XML-Datenbank wie eXist nicht vorfindet:
\begin{enumerate}
	\item Referentielle Integrit�t:
	\item[] In eXist kann dieser Punkt leider nicht garantiert werden. Hier muss von Hand sichergestellt werden, dass alle Referenzen auf ein gel�schtes Objekt ebenfalls gel�scht werden. Da unser Datenmodell aber �berschaubar ist, stellt dieser Punkt kein gro�es Problem dar.
	%\item automatische Indizierung von Inhalten beherscht eXist --> Vorteil gegen�ber anderen nativen XML-DBs
	\item Sequenzen zur Vergabe eindeutiger Primary Keys:
	\item[] Gerade bei einem Multinutzersystem ist dieser Punkt klar von Vorteil. Bei der Verwendung von eXist, muss dieses Problem auf eine andere Art und Weise gel�st werden. N�heres dazu unter \autoref{JumpDatenbank}.
\end{enumerate}
Da relationale Datenbanken aber nicht prim�r f�r die Speicherung von XML-Dateien geeignet sind, k�nnen sich eine Reihe von Problemen ergeben. Zum Einen k�nnen aufw�ndige Transformationen notwendig sein, um XML-Dokumente in die internen Strukturen zu �bersetzen. Mit der Angabe des passenden Schemas, l�sst sich dieser Prozess zwar beschleunigen, was aber dazu f�hrt, dass nur noch Dateien mit einer bestimmten Struktur akzeptiert werden...

Weitere wichtige Eigenschaften relationaler Datenbanksysteme beherrscht eXist jedoch, unter anderem:
\begin{enumerate}
	\item Effiziente und strukturierte Speicherung:
	\item[] Wie viele native XML-Datenbanken, verwendet eXist einen modellbasierten Ansatz zur Speicherung der Daten, was eine schnellere und effizientere Suche, als bei XML f�higen Datenbanken, die eine rein zeichenkettenbasierte Speicherung der Daten vornehmen, erm�glicht. 
	%\item automatische Indizierung von Inhalten beherscht eXist --> Vorteil gegen�ber anderen nativen XML-DBs
	\item Indizes:
	\item[] Deren Verwendung bewirkt ebenfalls schnellere Suchvorg�nge. Dabei erzeugt eXist bereits einige wichtige Indizes,welche in der Regel auch ausreichen um vor allem XPath und XQuery zu beschleunigen. Es k�nnen aber noch weitere Indizes in den Konfigurationsdateien zu jeder Collection vom Nutzer definiert werden.
	\item Transaktionssicherheit:
	\item[] eXist ist in der Lage nach einem Absturz, alle vollst�ndig beendeten Transaktionen wiederherzustellen und alle nicht abgeschlossenen Transaktionen zur�ckzusetzen.
\end{enumerate}
\subsection{Alternativen}
Nicht nur eXist, sondern auch andere Systeme eignen sich f�r den Umgang mit XML-Dokumenten. Im folgenden m�chten wir deshalb Alternativen vorstellen und begr�nden, warum wir uns gegen diese entschieden haben.