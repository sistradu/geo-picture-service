\section{Yaml}
\label{JumpYaml}
In heutigen Webseiten verfolgt man oft die strikte Trennung von Gestalltung/Layout und Inhalt. Das Layout wird durch die Cascading Style Sheets definiert. Um dies schnell und einfach zu erledigen gibt es ein Framework namens YAML. YAML steht für "Yet Another Multicolumn Layout" und ist ein (X)HTML/CSS Framework zur Erstellung von Layouts. Es liefert ein valides Grundgerüst aus validem XHTML- und CSS-Code, die eine hohe Browserkompatibilität bieten, d.h. eine browserübergreifende korrekte Darstellung garantieren. So muss sich nicht der Programmierer oder Designer um die verschiedenen Browser mit ihren Eigenschaften und Schwächen beschäftigen. 

Dies ist möglich, da durch YAML bereits viele Browser-Bugs abgefangen werden um die sich nicht speziell gekümmert werden muss.
Mit Hilfe von YAML ist eine fast vollständige Trennung zwischen Layout und späteren Inhalt möglich. 

