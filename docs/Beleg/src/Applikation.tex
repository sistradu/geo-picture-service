\section{Applikation}
\label{JumpApplikation}
%TODO anderen Namen vergeben?, Unterabschnitte anpassen
\subsection{Controller}
\subsubsection{UserController}
Die gesamte Benutzerverwaltung regelt der Usercontroller. Er stellt u.a. Methoden f�r die Kontoerstellung, die Anmeldung und das Zusenden eines neuen Passworts zur Verf�gung. Au�erdem wird in einer Variable die Information bereitgestellt, ob ein Benutzer angemeldet ist oder nicht. Dies ist f�r die Darstellung der Webseiten von gro�er Bedeutung. Da nicht alles f�r unangemeldete Benutzer angezeigt werden soll, werden "`private"' Bereiche mittels "`rendered"'-Attribut nur angezeigt, wenn der Benutzer angemeldet ist.\\
\\
Das Anlegen eines neuen Benutzkontos l�uft wie folgt ab. Der Benutzer w�hlt einen Benutzernamen und gibt seine EMail-Adresse an. An diese Adresse wird ein zuf�lllig generiertes Passwort gesendet. Auf diese Weise soll verifiziert werden, ob es die EMail-Adresse wirklich gibt. Sollte ein Benutzer n�mlich sein Passwort vergessen, wird ein neues Passwort an diese EMail-Adresse gesendet. F�r das zuf�llige Erzeugen des Passworts wird mittels einer Zufallszahl ein Buchstabe oder eine Zahl aus einer vorgegebenen Auswahl von Zeichen gewh�hlt und in einem String verkettet. Den Code der Methode findet man im Anhang auf Seite \pageref{lstRandomPasswort} im Listing \autoref{lstRandomPasswort}.\\
\\
Beim Umgang mit Benutzerdaten gibt es immer auch Anforderungen an die Sicherheit. So soll es vermieden werden, dass Passw�rter unverschl�sselt gespeichert werden. Unter Beachtung dieses Hinweises werden die Passw�rter deswegen verschl�ssselt gespeichert. Als Verschl�sselungsmethode wird der Hash-Algorithmus MD5 verwendet. Sobald ein Passwort von der View an den Controller �bergeben wird, erfolgt die Verschl�sselung bereits im Setter f�r die Passwortvariable. Auf diese Weise wird verhindert, dass irgendwo in der Anwendung ein unverschl�sseltes Passwort zur Verf�gung steht. Um nun festzustellen ob das vom Benutzer eingegebene Passwort das Richtige ist, wird das bei der Eingabe verschl�sselte Passwort mit dem Passwort aus der Datenbank verglichen. Sobald diese beiden Passw�rter �bereinstimmen erfolgt die Anmeldung. Die Methode, die das Verschl�sseln bewerkstelligt findet man wieder im Anhang im Listing \autoref{lstEncryptPasswort} auf Seite \pageref{lstEncryptPasswort}.\\
\\
Wie bereits weiter oben erw�hnt, werden EMails zur Verifizierung bzw. Zum Zur�cksetzen des Passworts verschickt. Daf�r wird die JavaMailAPI von Oracle verwendet. Aus dieser API werden allerdings nur die "`mailapi.jar"' und die "`smtp.jar"' verwendet, da die anderen jar-Dateien wie z.B "`imap.jar"' nicht ben�tigt werden. Die JavaMailAPI bietet Methoden an, damit der Inhalt der EMail und Verbindungseigenschaften zum Server gesetzt werden k�nnen. So kann z.B. eingestellt werden, dass die Verbindung zum Server �ber SSL hergestellt werden soll. Im Listing \autoref{lstMailversand} auf Seite \pageref{lstMailversand} im Anhang findet man die Methode, mit der die EMails versendet werden.
\subsection{Scopes}
\subsection{Beanzugriff}
\subsection{MessageProperties}

\subsection{EMail}
F�r die Benutzerkontoerstellung und das Zusenden eines neuen Passworts haben wir in unserem Projekt die M�glichkeit vorgesehen EMails zu versenden. Da der EMail-Versand zwingend ein EMail-Account erfordert, haben wir einen Mailserver eingerichtet, womit diese EMails verschickt werden konnten. Allerdings haben wir relative schnell Probleme festgestellt. So funktionierte das Schicken der Mails nur an die Hochschul-EMail-Adresse. Bei anderen Mail-Anbietern kamen keine EMails an. Sie wurden noch nicht einmal im Spam Ordner angezeigt. Der Grund hierf�r liegt wahrscheinlich darin, dass die anderen Mail-Anbieter unsere Adresse nicht als vertrauensw�rdig einstuften.\\
\\
Dies war aber nicht das einzigste Problem. Probleme mit unserem VServer erzwangen ein Neustart des Mail-Servers. Danach hat das Senden von Mails nicht mehr funktioniert. Wo die Gr�nde liegen ist leider nicht ohne eine aufwendige Problemanalyse raus zu bekommen. In Anbetracht der knappen Zeit haben wir uns daf�r entschieden, bei GMX ein Mailkonto einzurichten und �ber diese Adresse die Mails zu schicken.\\
\\
Nun steht nat�rlich die Frage im Raum warum wir das nicht von Anfang an gemacht haben und uns mit dem Aufwand, dem Einrichten eines Mailservers, viel Arbeit aufgeladen haben. Der Grund daf�r waren bedenken, einen "`Sinnlos-Account"' einzurichten. Es werden ja nur einige wenige EMails versendet. Au�erdem macht eine eigene EMail-Adresse immer einen besseren Eindruck. W�re mehr Zeit gewesen h�tten wir sicher das Problem mit unserem Mailserver l�sen k�nnen, aber so mussten wir eben diese Alternative verwenden.