\section{Applikation}
\label{JumpApplikation}
%TODO anderen Namen vergeben?, Unterabschnitte anpassen
\subsection{Controller}
\subsection{Scopes}
\subsection{Beanzugriff}
\subsection{MessageProperties}
\subsection{KML}
\subsection{File-Upload}
F�r das Hochladen der Bilder wird im Projekt die File-Upload Komponente von PrimeFaces verwendet. Dadurch konnte diese Anforderung relativ schnell umgesetzt werden. Zu Beginn traten allerdings einige Probleme auf. So ben�tigt diese Komponente eine Reihe von Biblotheken damit der Upload funktioniert. Dies wurde allerdings weder auf der PrimeFaces-Webseite noch in dem PDF erw�hnt. Letztendlich musste eine Recherche im Internet gemacht werden, wo dann in einem Forum erw�hnt wurde, welche Bibliotheken ben�tigt werden.\\
\\
Ansonsten ist die Verwendung der Komponente ziemlich einfach. Es ist sogar m�glich die Schaltfl�chen umzubenennen. Diese M�glichkeit wurde nat�rlich genutzt. Sobald man ein Bild hochl�dt, wird ein FileUploadEvent ausgel�st. Mit dem entsprechenden Event-Objekt kann man auf die Daten zugreifen. Wir haben uns dazu entschieden das Bild im Dateisystem abzuspeichern und in der Datenbank lediglich den Pfad abzuspeichern. Der Grund daf�r ist, dass sonst bei der Speicherung in der Datenbank die Bilder erst aufwendig encodiert werden m�ssten. Sobald dann sp�ter die Bilder aus der Datenbank geholt werden, m�sste man au�erdem wieder die Daten zu einem JPEG-Bild konvertieren. Diesen Mehraufwand konnten wir durch die Speicherung im Dateisystem umgehen. Au�erdem gibt es mit unserem Verfahren weniger Probleme wenn mehrere Bilder hochgeladen werden.\\
\\
Das Upload-Verzeichnis an sich liegt irgendwo im Dateisystem. Die einfachste M�glichkeit w�re zwar der web-Ordner der Applikation gewesen, aber dort w�re die Bilder bei jedem Neudeploy gel�scht worden. Aus diesem Grund gibt es ein extra Verzeichnis wo alle Bilder abgespeichert werden.
\subsection{EMail}
F�r die Benutzerkontoerstellung und das Zusenden eines neuen Passworts haben wir in unserem Projekt die M�glichkeit vorgesehen EMails zu versenden. Da der EMail-Versand zwingend ein EMail-Account erfordert, haben wir einen Mailserver eingerichtet, womit diese EMails verschickt werden konnten. Allerdings haben wir relative schnell Probleme festgestellt. So funktionierte das Schicken der Mails nur an die Hochschul-EMail-Adresse. Bei anderen Mail-Anbietern kamen keine EMails an. Sie wurden noch nicht einmal im Spam Ordner angezeigt. Der Grund hierf�r liegt wahrscheinlich darin, dass die anderen Mail-Anbieter unsere Adresse nicht als vertrauensw�rdig einstuften.\\
\\
Dies war aber nicht das einzigste Problem. Probleme mit unserem VServer erzwangen ein Neustart des Mail-Servers. Danach hat das Senden von Mails nicht mehr funktioniert. Wo die Gr�nde liegen ist leider nicht ohne eine aufwendige Problemanalyse raus zu bekommen. In Anbetracht der knappen Zeit haben wir uns daf�r entschieden, bei GMX ein Mailkonto einzurichten und �ber diese Adresse die Mails zu schicken.\\
\\
Nun steht nat�rlich die Frage im Raum warum wir das nicht von Anfang an gemacht haben und uns mit dem Aufwand, dem Einrichten eines Mailservers, viel Arbeit aufgeladen haben. Der Grund daf�r waren bedenken, einen "`Sinnlos-Account"' einzurichten. Es werden ja nur einige wenige EMails versendet. Au�erdem macht eine eigene EMail-Adresse immer einen besseren Eindruck. W�re mehr Zeit gewesen h�tten wir sicher das Problem mit unserem Mailserver l�sen k�nnen, aber so mussten wir eben diese Alternative verwenden.