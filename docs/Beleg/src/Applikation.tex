\section{Applikation}
\label{JumpApplikation}
%TODO anderen Namen vergeben?, Unterabschnitte anpassen
\subsection{Controller}
\subsubsection{UserController}

\subsection{Scopes}
\subsection{Beanzugriff}
\subsection{MessageProperties}

\subsection{EMail}
F�r die Benutzerkontoerstellung und das Zusenden eines neuen Passworts haben wir in unserem Projekt die M�glichkeit vorgesehen EMails zu versenden. Da der EMail-Versand zwingend ein EMail-Account erfordert, haben wir einen Mailserver eingerichtet, womit diese EMails verschickt werden konnten. Allerdings haben wir relative schnell Probleme festgestellt. So funktionierte das Schicken der Mails nur an die Hochschul-EMail-Adresse. Bei anderen Mail-Anbietern kamen keine EMails an. Sie wurden noch nicht einmal im Spam Ordner angezeigt. Der Grund hierf�r liegt wahrscheinlich darin, dass die anderen Mail-Anbieter unsere Adresse nicht als vertrauensw�rdig einstuften.\\
\\
Dies war aber nicht das einzigste Problem. Probleme mit unserem VServer erzwangen ein Neustart des Mail-Servers. Danach hat das Senden von Mails nicht mehr funktioniert. Wo die Gr�nde liegen ist leider nicht ohne eine aufwendige Problemanalyse raus zu bekommen. In Anbetracht der knappen Zeit haben wir uns daf�r entschieden, bei GMX ein Mailkonto einzurichten und �ber diese Adresse die Mails zu schicken.\\
\\
Nun steht nat�rlich die Frage im Raum warum wir das nicht von Anfang an gemacht haben und uns mit dem Aufwand, dem Einrichten eines Mailservers, viel Arbeit aufgeladen haben. Der Grund daf�r waren bedenken, einen "`Sinnlos-Account"' einzurichten. Es werden ja nur einige wenige EMails versendet. Au�erdem macht eine eigene EMail-Adresse immer einen besseren Eindruck. W�re mehr Zeit gewesen h�tten wir sicher das Problem mit unserem Mailserver l�sen k�nnen, aber so mussten wir eben diese Alternative verwenden.