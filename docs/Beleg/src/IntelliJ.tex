\section{IntelliJ}
\label{JumpIntelliJ}
Dieses Projekt wurde mit der Entwicklungsumgebung IntelliJ IDEA der Firma JetBrains implementiert. Von der Umgebung werden Programmiersprachen wie Java, PHP und noch viele mehr unterst�tzt. Features wie Refactoring, Ant und JUnit und eine Versionskontrolle sind ebenfalls in Intellij zu finden.\\
\\
Was ist uns positiv an Intellij aufgefallen und was hat uns bei der Implementierung viel Zeit gekostet bzw. die Arbeit erschwert?\\
\\
Positiv aufgefallen sind bei der Entwicklung einer Applikation, die Codevervollst�ndigung, welche unter anderem auch Vorschl�ge bei der Variablendeklaration angeboten hat. Zudem war schnell ein �berblick zwischen lokalem Quellcode und der SVN-Version m�glich, da Differenzen automatisch im Quellcode am Rand angezeigt wurden, ohne extra einen Befehl aufzurufen. \\
\\
Zu den neutralen Dingen, die aufgefallen sind, war die Anordnung der Men�s rund um das Codefenster, siehe \autoref{fig:intellij_menu} im Anhang. Das erschwerte die Arbeit am Beginn der Entwicklung, �nderte sich aber mit der Zeit zu einem Vorteil, als man wusste, wo sich was befindet. Auch der Tooltip bei der Parameter�bergabe einer Methode, war vor der Version 10.5 nicht immer verf�gbar, funktionierte aber seit der Version 10.5 stabil und bot eine gute �bersicht.\\
\\
Zu den negativen Punkten von Intellij geh�rt unter anderem die inkonsequente Umsetzung der Hotkeys. Es wird angeboten, diese auf andere Entwicklungsumgebungen anzupassen, aber das leider nicht komplett. Auch die Gl�hbirne welche ab und zu mal erschien. war sehr speziell. Sie bot die M�glichkeit, schnell UnitTest zu erstellen, aber leider verschwand sie oft, bevor man mit dem Mauszeiger auf ihr war. Teilweise sorgten auch die sehr �berladenen Men�s f�r einiges Suchen, um eine kleine Einstellung zu �ndern. Wie auch andere bekannte Entwicklungsumgebungen verf�gt auch diese �ber die Nutzung verschiedener Plugins. Die Anzahl der angebotenen Plugins ist aber eher klein, da es sich bei Intellij um eine kostenpflichtige Software handelt. In diesem Zusammenhang muss man auch sagen, dass die Auswahl der Plugins, bei der Installation getroffen werden musste, was zu einem sehr gro�en Speicherverbrauch bei der Ausf�hrung sorgte, da man nicht genau wusste, welche Plugins ben�tigt werden und man somit alle ausgew�hlt hat. Mit dem Erscheinen der Version 10.5 folgte zwar auch eine Community-Edition, welche aber nicht betrachtet wurde. 