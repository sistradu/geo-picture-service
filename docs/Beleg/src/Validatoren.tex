\section{Validatoren}
\label{JumpValidatoren}
\subsection{Allgemeines}
Oft haben Applikationen mit fehlerhaften Eingaben zu k�mpfen, die der Nutzer nicht einmal bewusst t�tigt. Um einen reibunglosen Ablauf zu garantieren, gibt es die M�glichkeit, sogenannte Validatoren zu benutzen.
Sie sind daf�r da, eine Eingabe auf eine bestimmte Art von Muster bzw. Standard zu �berpr�fen. Da das Ganze aber automatisch bei jeder Eingabe ablaufen soll, kann dabei nur auf syntaktische Korrektheit gepr�ft werden. \\
\subsection{Validatoren in JSF}
Um einen Validator in JSF nutzen zu k�nnen, muss er in der faces-config.xml bekannt gemacht werden. Das sieht wie folgt aus.
\begin{lstlisting}[caption=Einbinden eines Validators ind die faces-config.xml]{Name}
<validator>
   <validator-id>EmailValidator</validator-id>
   <validator-class>de.hszigr.gpics.validation.EmailValidator</validator-class>
</validator>
\end{lstlisting}
Alle Validatoren, die genutzt werden sollen, m�ssen sich in dem Tag "'validator'" der faces-config befinden. Jeder Validator muss �ber eine eindeutige ID verf�gen und der Klassenname muss auch hinterlegt werden. \\
Das allein reicht allerdings noch nicht aus, damit der Validator auch aktiv wird. An den Stellen, wo die Eingabe validiert werden soll, muss er noch der jeweiligen Komponente zugewiesen werden. Das geschieht so:
\begin{lstlisting}[caption=Zuweisung eines Validators einer Komponente]{Name}
<h:inputText id="createUserMail" value="#{userController.email}" required="true" requiredMessage="#{msg.forgotEmail}">
  <f:validator validatorId="EmailValidator"/>
</h:inputText>
\end{lstlisting}
Wie hier zu sehen ist, wird der EmailValidator einem Input-Textfeld zugewiesen. Das "'f:"' am Anfang des Tags ist der Alias f�r jsf-core.
\subsection{Regul�re Ausdr�cke}
Mit regul�ren Ausdr�cken kann der oben beschriebene Test auf syntaktische Korrektheit durchgef�hrt werden. Neben dem EmailValidator werden noch 2 weitere Validatoren verwendet, der NameValidator, um den Nutzernamen zu validieren, und der KoordValidator, um vom Nutzer eingegebene Koordinaten zu validieren. Der Aufbau dieser Klassen, dargestellt am Beispiel des NameValidators in \autoref{JumpValidatorName}, ist dabei immer gleich. Das einzige was sich �ndert ist der regul�re Ausdruck in dieser Zeile:
\begin{lstlisting}[caption=Der regul�re Ausdruck zur Validierung des Namens,label=JumpValidatorRegName]
if(!name.matches("[\\w\\d]+") || name.matches("[\\d]+"))
\end{lstlisting}
Hier wird die Syntaxpr�fung vorgenommen. Im Falle des Nutzernamens ist das noch ein sehr einfacher Ausdruck. Damit werden nur Nutzernamen akzeptiert, die aus einer Menge von Zahlen und mindestens einem Buchstaben bestehen. 'Nutzer123' ist damit ein g�ltiger Name, ebenso '12hgf3Fh4'. Namen wie '1234' oder 'Nut-zer\#12+3' werden vom System aber nicht akzeptiert. Dadurch sollen Namen mit zu vielen Sonderzeichen vermieden werden.\\
Bei E-Mailadressen sieht dieser Ausdruck schon komplizierter aus:
\begin{lstlisting}[caption=Der regul�re Ausdruck zur Validierung der Emailadresse,label=JumpValidatorRegEmail]
"([\\w\\d]+[-\\.]{1})*[\\w\\d]+@([\\w\\d-]+\\.)+[\\w]{2,4}"
\end{lstlisting}
Dieser Ausdruck deckt dabei nicht alle m�glichen E-Mailadressen ab, aber zumindest die meisten und das mit relativ geringem Aufwand. So werden vor dem \texttt{@} beliebige Kombinationen aus Buchstaben und Zahlen zugelassen. Diese k�nnen entweder durch genau einen Punkt oder genau einen Bindestrich getrennt werden, das aber beliebig oft. Nach dem \texttt{@} k�nnen wieder beliebige Kombinationen aus Buchstaben, Zahlen und diesmal auch beliebig vielen Bindestrichen folgen. Jedesmal getrennt durch genau einen Punkt. Die Emailadresse muss abgeschlossen werden mit einer Kombination aus mindestens 2 bis maximal 4 Buchstaben. Somit werden Adressen wie 'test@e-mail.de' oder 'a-b.c-d@ef--gh.ijkl.xyz' zugelassen, nicht aber Adressen wie '1234...@test.com' oder 'e-mail@test.account'.\\
Bei den Koordinaten wurde entschieden nur die Eingabe von Koordinaten im Dezimalformat zuzulassen. Das erleichtert die Validierung auch erheblich. Der korrekte regul�re Ausdruck daf�r lautet:
\begin{lstlisting}[caption=Der regul�re Ausdruck zur Validierung der Koordinaten,label=JumpValidatorRegKoord]
"-?[0-9]{1,3}\\.[0-9]{1,20}"
\end{lstlisting}
Damit werden nur Koordinaten wie '5.1234' oder '-123.997283' zugelassen, aber nicht '-12-1234' oder '13.'\\
Leider wurde erst nach Abschluss des Projektes bemerkt, dass sich in diesem Ausdruck ein Tippfehler eingeschlichen hatte. Dieser Tippfehler entstand durch die Vereinfachung des Ausdrucks (vorher wurden auch andere Notationen zugelassen). Da zwar Tests f�r den urspr�nglichen Ausdruck existieren, diese aber nicht �bernommen wurden, fiel der Tippfehler vorher nicht auf. Bei der aktuellen Version wird statt eines Punktes, jedes beliebige Zeichen als Trennzeichen zugelassen, was nat�rlich nicht erw�nscht ist.