\section{Validatoren}
\label{JumpValidatoren}
Oft haben Applikationen mit fehlerhaften Eingaben zu k�mpfen, die der Nutzer nicht einmal bewusst t�tigt. Um daraus bedingten Fehlfunktionen und einem reibunglosen Ablauf zu garantieren, gibt es die M�glichkeit sogenannte Validatoren zu benutzen.
Sie sind daf�r da, eine Eingabe auf eine bestimmte Art von Muster bzw. Standard zu �berpr�fen. Da das Ganze aber automatisch bei jeder Eingabe ablaufen soll, kann dabei nur auf syntaktische Korrektheit gepr�ft werden. \\

Um einen Validator in JSF nutzen zu k�nnen muss er in der faces-config.xml bekannt gemacht werden. Das sieht wie folgt aus.
\begin{lstlisting}[caption=Einbinden eines Validators ind die faces-config.xml]{Name}
<validator>
   <validator-id>EmailValidator</validator-id>
   <validator-class>de.hszigr.gpics.validation.EmailValidator</validator-class>
</validator>
\end{lstlisting}
Alle Validatoren, die genutzt werden sollen, m�ssen sich in dem Tag "validator" der faces-config befinden. Jeder Validator muss �ber eine eindeutige ID verf�gen und der Klassenname muss auch hinterlegt werden. \\
Das allein reicht noch nicht aus damit der Validator auch aktiv wird. An den Stellen, wo die Eingabe validiert werden soll, muss er noch der jeweiligen Komponente zugewiesen werden. Das geschieht so:
\begin{lstlisting}[caption=Zuweisung eines Validators einer Komponente]{Name}
<h:inputText id="createUserMail" value="#{userController.email}" required="true" requiredMessage="#{msg.forgotEmail}">
  <f:validator validatorId="EmailValidator"/>
</h:inputText>
\end{lstlisting}
Wie hier zu sehen ist, wird der Email-Validator einem Input-Textfeld zugewiesen. Das "f:" am Anfang des Tags ist das alias f�r jsf-core.


