\section{GoogleCode}
\label{JumpGooglecode}
Ein weiterer wichtiger Bestandteil der Entwicklungsumgebung ist ein System, dass automatisch allen Teammitgliedern den aktuellen Entwicklungsstand zug�nglich macht. Anbieter von sogenannten Repositorys gibt es viele, so sind sowohl kostenlose als auch kostenpflichtige zu finden. Im Laufe des Studiums wurde bisher das System von ProjectLocker\footnote{URL: http://www.projectlocker.com/}, sowie ein Subversion-Server der Hochschule genutzt. Da das Projekt 4 Studenten bearbeiteten, konnte ProjectLocker nicht verwendet werden, da dieses in der kostenlosen Variante nur 3 Benutzer erlaubt. Eine kostenpflichtige Nutzung h�tte das Problem behoben, kam aber auf Grund von verf�gbaren, kostenlosen Alternativen nicht in Frage. Die Nutzung eines Hochschul-Servers wurde in der Gruppe besprochen, jedoch angesichts des Umzuges des Fachbereiches in ein anderes Geb�ude und der daraus resultierenden Ausfallzeit abgelehnt.\newline\newline
Ein Alternative zu den genannten Varianten ist das GoogleCode-Projekt. Voraussetzung zur Er�ffnung eines Projektes bei GoogleCode ist, dass man dieses unter einer OpenSource-Lizenz ver�ffentlicht. Anschlie�end kann man den vollen Funktionsumfang der Plattform nutzen, der Link zur Projektseite lautet http://code.google.com/p/geo-picture-service/. Zu den Funktionen z�hlen neben dem SVN-Repository, ein Wiki, welches zum Publizieren der aktuellen Fortschritte bei der Programmierung, sowie zum Ver�ffentlichen der gehaltenen Zwischenstandspr�sentationen in der Lehrveranstaltung diente. Au�erdem ist ein Issue-Tracker vorhanden. Dieser dient dazu, gefundene Fehler dem Entwickler direkt mitzuteilen. Dies erm�glicht eine schnelle und f�r andere nachvollziehbare M�glichkeit aufgetretene Probleme zu kommunizieren. Diese Art der Fehlerkorrektur wurde jedoch eher selten genutzt, da gr��tenteils die Entwickler sich in ein und dem selben Raum befanden. Eine direkte Kontaktierung erm�glichte eine schnellere und effektivere Behebung des Fehlers. Bei gr��eren Projekten und einer gr��eren r�umlichen Distanz zwischen den Programmierern, ist die Nutzung des Dienstes jedoch sehr zu empfehlen.\newline\newline
W�hrend der Bearbeitung des Projektes kam es zu keinerlei gr��eren Problemen mit dem Service. Lediglich das Einreichen von ge�nderten Code-Abschnitten konnte sehr selten nicht durchgef�hrt werden. Ein Grund daf�r konnte nicht gefunden werden. Internetrecherchen ergaben nur, dass dies sehr selten und unregelm��ig auftreten kann. Nach einer kurzen Wartezeit von wenigen Minuten funktionierte es wieder und der Quellcode konnte erfolgreich in das Repository aufgenommen werden. Die GoogleCode-Plattform ist zur Verwaltung von Programmen w�hrend der Entwicklungsphase und des Produktiveinsatzes absolut zu empfehlen.