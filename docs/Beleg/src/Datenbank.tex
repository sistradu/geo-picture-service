\section{Datenbank}
\label{JumpDatenbank}
\subsection{Aufbau der Datenbank}
Der wir nur 3 Klassen besitzen, fiel auch der Aufbau der Datenbank entsprechend einfach aus. Zun�chst wurde f�r jede Klasse eine neue Collection angelegt, was etwas einer Tabelle in einer relationalen Datenbank oder einem Ordner in einem Dateisystem entspricht. In diesen Collections k�nnen nun ein, oder mehrere XML-Files gespeichert werden. Wie wir in \autoref{JumpDBOneBIGManysmall} begr�ndet haben, speichern wir hier nur jeweils eine Datei.
Zus�tzlich wurde noch eine Collection angelegt f�r unsere Stored XQuerys. In dieser sind mehrere .xql-Dateien hinterlegt, f�r alle Abfragen, die durchgef�hrt werden m�ssen:
\begin{itemize}
	\item exist/rest/db/
	\begin{itemize}
		\item nutzer/nutzers.xml
		\item alben/alben.xml
		\item bilder/bilder.xml
		\item queries/
		\begin{itemize}
			\item allNutzer.xql
			\item albenForNutzer.xql
			\item[] ...
			\item bilderForAlbum.xql
		\end{itemize}
	\end{itemize}
\end{itemize}
F�r den Aufbau der einzelnen XML-Dateien sind im Anhang Beispiele unter \autoref{JumpListingNutzerXML}, \autoref{JumpListingAlbumXML} und \autoref{JumpListingBildXML} zu finden.
\todo{XML-Dateien in Anhang verschieben}
\begin{lstlisting}[caption=Beispiel f�r einen Eintrag in der nutzers.xml,label=JumpListingNutzerXML,language=XML]
<nutzer>
    <id>0</id>
    <name>Karl</name>
    <password>98f6bcd4621d373cade4e832627b4f6</password>
    <email>karl@web.de</email>
</nutzer>
\end{lstlisting}
\begin{lstlisting}[caption=Beispiel f�r einen Eintrag in der alben.xml,label=JumpListingAlbumXML,language=XML]
<album>
    <id>0</id>
    <name>G�rlitz</name>
    <password>9e229f114819f62674b5b1a031a9f91</password>
    <description>Bilder �ber die Stadt G�rlitz.</description>
    <nutzer>0</nutzer>
</album>\end{lstlisting}
\begin{lstlisting}[caption=Beispiel f�r einen Eintrag in der bilder.xml,label=JumpListingBildXML,language=XML]
<bild>
    <id>2</id>
    <name>Vogtshof Innenhof</name>
    <description>Studenten beim Grillen</description>
    <ispublic>false</ispublic>
    <date>2011-02-21 19:05:01</date>
    <fileposition>/pics/Vogtshof_Innenhof_3.jpg</fileposition>
    <album>0</album>
    <position>
        <latitude>51.148833</latitude>
        <longitude>51.148833</longitude>
        <altitude>123</altitude>
        <direction>120</direction>
    </position>
</bild>
\end{lstlisting}
\subsection{Mockklassen}
Da die Datenbank noch nicht zu Beginn der Implementierungsphase komplett einsatzf�hig war, musste eine Zwischenl�sung her, damit bereits einfache Funktionalit�ten getestet werden konnten. Die Idee war es Mockklassen zu verwenden. Dazu wurden zun�chst Interfaces f�r alle sogenannten Konnektor-Klassen angelegt. Die Mockklassen zu den einzelnen Interfaces implementieren bereits einen gro�en Teil dieser Funktionen, ohne dabei im Hintergrund mit der Datenbank zu kommunizieren. Demnach bieten die Mockklassen auch nur eine eingeschr�nkte Funktionalit�t, wie die R�ckgabe eines Benutzers zu einer ID oder die Suche nach bestimmten Alben, basierend auf einer Menge von Testdaten. Eintragen, L�schen oder Modifizieren dieser Daten ist mit den Mockklassen dabei nicht m�glich.\\
Wird nun eine solche Klasse erzeugt, werden die Testdaten automatisch generiert und k�nnen genutzt werden. Sp�ter kann die jeweilige Mockklasse einfach mit der 'echten' Implementierung ausgetauscht werden. 
Ein Beispiel einer solchen Mockklasse, in dem Fall der MockAlbumConnector ist im Anhang unter \autoref{JumpListingMockAlbumConnector} angegeben. Der Einfachheit halber werden �hnliche oder nicht implementierte Methoden weggelassen.

\begin{lstlisting}[caption=Die Klasse MockAlbumConnector,label=JumpListingMockAlbumConnector]
public class MockAlbumConnector implements IAlbumConnector {
  //Mehrere HashMaps um den Implementierungsaufwand in Grenzen zu Halten
  private final HashMap<Integer, Document> albenIDMap;
  private final HashMap<String, Document> albenNameMap;
  private final HashMap<Integer, Document> albenNutzerIDMap;
  private final HashMap<String, Document> albenDescriptionMap;
  int id = -1;

  public MockAlbumConnector(){
    this.albenIDMap = new HashMap<Integer,Document>();
    this.albenNameMap = new HashMap<String,Document>();
    this.albenNutzerIDMap = new HashMap<Integer,Document>();
    this.albenDescriptionMap = new HashMap<String,Document>();
    //Passwortverschl�sselung f�r ein Album
    String password = "goerlitz";
    try {
      MessageDigest md = MessageDigest.getInstance("md5");
      byte[] digest = md.digest(password.getBytes());
      StringBuffer hexString = new StringBuffer();
      for (int i = 0; i < digest.length; i++) {
        hexString.append(Integer.toHexString(0xFF & digest[i]));
      }
      password = hexString.toString();
    } catch (NoSuchAlgorithmException e) {
      e.printStackTrace();
    }
    this.addAlbum("G�rlitz", password, "Bilder �ber die Stadt G�rlitz.", 0);
    ...
  }
  //Methode um ein Album den Testdaten hinzuzuf�gen
  private void addAlbum(String name, String password, String description, int nutzerID){
    this.id++;
    try {
      DocumentBuilderFactory factory = DocumentBuilderFactory.newInstance();
      Document doc = factory.newDocumentBuilder().newDocument();
      Element idElem = doc.createElement("id");
      Element nameElem = doc.createElement("name");
      Element passwordElem = doc.createElement("password");
      Element descriptionElem = doc.createElement("description");
      Element nutzerElem = doc.createElement("nutzer");
      idElem.setTextContent(""+this.id);
      nameElem.setTextContent(name);
      passwordElem.setTextContent(password);
      descriptionElem.setTextContent(description);
      nutzerElem.setTextContent("" + nutzerID);
      Node albumNode = doc.createElement("album");
      albumNode.appendChild(idElem);
      albumNode.appendChild(nameElem);
      albumNode.appendChild(passwordElem);
      albumNode.appendChild(descriptionElem);
      albumNode.appendChild(nutzerElem);
      doc.appendChild(albumNode);
      this.albenIDMap.put(this.id, doc);
      this.albenNameMap.put(name, doc);
      this.albenNutzerIDMap.put(nutzerID + this.id, doc);
      this.albenDescriptionMap.put(description, doc);
    } catch (ParserConfigurationException e) {
      System.out.println("Fehler beim Erstellen des Albums " + name + ".");
      e.printStackTrace();
    }
  }
  ...
  public Document getAlbumByID(int id) throws ConnectException, IllegalArgumentException {
    if(this.albenIDMap.containsKey(id))
      return this.albenIDMap.get(id);
    else
      throw new IllegalArgumentException("");
  }
  ...
  public Document getAllAlben() throws ConnectException {
    try {
      DocumentBuilderFactory factory = DocumentBuilderFactory.newInstance();
      Document doc = factory.newDocumentBuilder().newDocument();
      Node albenNode = doc.createElement("alben");
      for(Map.Entry<Integer,Document> entry : this.albenIDMap.entrySet()){
        albenNode.appendChild(doc.adoptNode(entry.getValue().getFirstChild()));
      }
      doc.appendChild(albenNode);
      return doc;
    } catch (ParserConfigurationException e) {
      System.out.println("Fehler beim Erstellen des Dokuments f�r alle Alben.");
      e.printStackTrace();
    }
    return null;
  }

  //�hnlich wie alle Alben, nur mit Pr�fung, ob das Album dem angegebenen Nutzer geh�rt
  public Document getAlbenForNutzer(int nutzerID) throws ConnectException, IllegalArgumentException {
    try {
      DocumentBuilderFactory factory = DocumentBuilderFactory.newInstance();
      Document doc = factory.newDocumentBuilder().newDocument();
      Node albenNode = doc.createElement("alben");
      for(Map.Entry<Integer,Document> entry : this.albenIDMap.entrySet()){
        if(entry.getValue().getElementsByTagName("nutzer").item(0).getTextContent().equals(""+nutzerID))
          albenNode.appendChild(doc.adoptNode(entry.getValue().getFirstChild()));
      }
      doc.appendChild(albenNode);
      return doc;
    } catch (ParserConfigurationException e) {
      System.out.println("Fehler beim Erstellen des Dokuments f�r alle Alben.");
      e.printStackTrace();
    }
    return null;
  }
  ...
}
\end{lstlisting}
\subsection{Datensicherheit}

\subsection{Stored XQueries}

\subsection{Zugriff auf die Datenbank}

\subsection{Verhinderung gleichzeitiger Zugriffe}