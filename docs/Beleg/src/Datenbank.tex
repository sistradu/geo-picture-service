\section{Datenbank}
\label{JumpDatenbank}
\subsection{Aufbau der Datenbank}
Der wir nur 3 Klassen besitzen, fiel auch der Aufbau der Datenbank entsprechend einfach aus. Zun�chst wurde f�r jede Klasse eine neue Collection angelegt, was etwas einer Tabelle in einer relationalen Datenbank oder einem Ordner in einem Dateisystem entspricht. In diesen Collections k�nnen nun ein, oder mehrere XML-Files gespeichert werden. Wie wir in \autoref{JumpDBOneBIGManysmall} begr�ndet haben, speichern wir hier nur jeweils eine Datei.
Zus�tzlich wurde noch eine Collection angelegt f�r unsere Stored XQuerys. In dieser sind mehrere .xql-Dateien hinterlegt, f�r alle Abfragen, die durchgef�hrt werden m�ssen:
\begin{itemize}
	\item exist/rest/db/
	\begin{itemize}
		\item nutzer/nutzers.xml
		\item alben/alben.xml
		\item bilder/bilder.xml
		\item queries/
		\begin{itemize}
			\item allNutzer.xql
			\item albenForNutzer.xql
			\item[] ...
			\item bilderForAlbum.xql
		\end{itemize}
	\end{itemize}
\end{itemize}
F�r den Aufbau der einzelnen XML-Dateien sind im Anhang Beispiele unter \autoref{JumpListingNutzerXML}, \autoref{JumpListingAlbumXML} und \autoref{JumpListingBildXML} zu finden.
\subsection{Mockklassen}
Da die Datenbank noch nicht zu Beginn der Implementierungsphase komplett einsatzf�hig war, musste eine Zwischenl�sung her, damit bereits einfache Funktionalit�ten getestet werden konnten. Die Idee war es Mockklassen zu verwenden. Dazu wurden zun�chst Interfaces f�r alle sogenannten Konnektor-Klassen angelegt. Die Mockklassen zu den einzelnen Interfaces implementieren bereits einen gro�en Teil dieser Funktionen, ohne dabei im Hintergrund mit der Datenbank zu kommunizieren. Demnach bieten die Mockklassen auch nur eine eingeschr�nkte Funktionalit�t, wie die R�ckgabe eines Benutzers zu einer ID oder die Suche nach bestimmten Alben, basierend auf einer Menge von Testdaten. Eintragen, L�schen oder Modifizieren dieser Daten ist mit den Mockklassen dabei nicht m�glich.\\
Wird nun eine solche Klasse erzeugt, werden die Testdaten automatisch generiert und k�nnen genutzt werden. Sp�ter kann die jeweilige Mockklasse einfach mit der 'echten' Implementierung ausgetauscht werden. 
Ein Beispiel einer solchen Mockklasse, in dem Fall der MockAlbumConnector, ist im Anhang unter \autoref{JumpListingMockAlbumConnector} angegeben. Der Einfachheit halber werden nur die wichtigsten Methoden dargestellt.
\subsection{Datensicherheit}
%\begin{figwindow}[1, r, \includegraphics[width=0.35\textwidth]{img/Zugriffrechteverwaltung-eXist.png}, {Sperrung von Zugriffsrechten in eXist %\label{JumpFigureZugriffsrechteeXist}}]
Die Datensicherheit zu gew�hrleisten funkioniert unter eXist sehr einfach. Durch die einfache Zugriffsrechteverwaltung f�r Collections und einzelnen Dateien, k�nnen diese f�r �ffentliche Nutzer schnell und einfach gesperrt werden. Nun m�ssen wir aber noch unserer Applikation vor einer Anfrage mitteilen, wie sich diese zu authentifizieren hat. Sonst w�rde diese sich als Gast anmelden und somit keine Zugriffsrechte besitzen. Das kann einfach umgesetzt werden, indem wir die Klasse java.net.Authenticator erweitern, dargestellt in \autoref{JumpListingAuthenticator}.
%\end{figwindow}
\begin{lstlisting}[caption=Erweiterung der Klasse Authenticator,label=JumpListingAuthenticator]
public class ExistAuthentificator extends Authenticator {
    protected PasswordAuthentication getPasswordAuthentication(){
        String promptString = getRequestingPrompt();
        String hostname = getRequestingHost();
        InetAddress ipaddr = getRequestingSite();
        int port = getRequestingPort();
        String username = USERNAME;
        String password = PASSWORD;
        return new PasswordAuthentication(username, password.toCharArray());
    }
}
\end{lstlisting}
Vor einer Anfrage muss nun einfach der Authenticator gesetzt werden:
\begin{lstlisting}
Authenticator.setDefault(new ExistAuthentificator());
\end{lstlisting}
Damit kann unser System problemlos auf die Datenbank zugreifen, w�hrend nicht authorisierte Nutzer dazu nicht in der Lage sind.
\subsection{Stored XQueries}

\subsection{Zugriff auf die Datenbank}

\subsection{Verhinderung gleichzeitiger Zugriffe}