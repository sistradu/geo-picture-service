\chapter{Tests}
\label{JumpTests}
%TODO Aufteilen in Unterabschnitte
\section{UserControllerTest}
Um nach �nderungen sicher zu sein das die Klasse noch richtig funktioniert m�ssen Tests durchgef�hrt werden. Aus diesem Grund wurden beim UserController die Methoden sendPasswortEmail, Login, Logout, erzeugeBenutzer, setPasswort und updateBenutzer getestet. Dabei soll vor allem �berpr�ft werden, ob auch wirklich im Controller die Attribute richtig gesetzt werden. So wird zum Beispiel in der Methode testSendPasswortEmail nicht �berpr�ft, ob eine EMail versendet wird, sondern ob ein neues zuf�lliges Passwort generiert wurde.\\
\\
Leider treten bei den meisten Tests Fehler auf. Das liegt haupts�chlich daran, dass aus einem Controller heraus Nachrichten auf der Webseite gesetzt werden sollen. Dazu wird der FacesContext verwendet. Dieser ist allerdings bei einem Test stets null. Daher sind wirklich sinnvolle Tests nicht mehr m�glich. Ein Framework was bei diesem Punkt Abhilfe schaffen k�nnte w�re JSFUnit. F�r eine Evaluierung dieses Frameworks war aber keine Zeit mehr sodass es in diesem Projekt nicht eingesetzt wird.

\section{ImageDataExtractorTest und CoordinateCalculatorTest}
F�r die Extrahierung der Geo-Daten aus den Bildern ist auch ein Test erstellt worden. Damit sollte sichergestellt werden, dass die GPS-Koordinaten richtig aus einem Bild geholt werden. Die Erzeugung eines Thumbnails aus dem Bild mit dem Framework "`Metadata Extractor"' wurde ebenfalls �berpr�ft. Dazu ist auf dem Server auch ein Testbild mit Geoinformationen bereitgestellt worden, so dass der Build-Server die Test ordentlich durchf�hren kann.\\
\\
Der CoordinateCalculatorTest �berpr�ft ob die Umrechnung der Koordinaten von z.B. N 51''8'55.8, E 14''59'55.985 in das von Google gew�nschte Dezimalformat\\ (z.B.: 51.148833333333336, 14.998884722222222) richtig arbeitet.\\
\\
\section{GPicSUtilTest und PasswortUtilTest}
In der Klasse GPicSUtil werden einige Hilfsmethoden bereitgestellt, die an mehreren Stellen bei den Controllern ben�tigt werden. Es wurde hier besonders die Methode getStreamedContent �berpr�ft. Diese soll von ein Bild von der Festplatte holen und es als StreamedContent zur�ckgeben. Nach einem Aufruf der Methode darf also das zur�ckgegebene StreamedContent-Objekt nicht null sein. Sollte dies nicht der Fall sein, ist der Test erfolgreich durchgelaufen.\\
\\
Die Klasse PasswortUtil stellt zwei Methoden zur Verf�gung. Eine soll ein Passwort mit MD5 verschl�sseln und die zweite ein zuf�lliges Passwort erzeugen. F�r den Verschl�sselungstest wurde ein Wort ausgew�hlt, wo der MD5-Hash schon bekannt ist. Dann wurde dieses Wort an die Methode �bergeben und der R�ckgabewert mit dem bekannten MD5-Hash verglichen. Der Test ist erfolgreich, wenn die beiden Werte gleich sind. F�r den zweiten Test wird die Methode zum Erzeugen von Zufallspassw�rtern zweimal aufgerufen und anschlie�end die beiden R�ckgabewerte miteinander verglichen. Diese beiden Werte d�rfen nicht gleich sein. Sollte dies dennoch der Fall sein schl�gt der Test fehl.