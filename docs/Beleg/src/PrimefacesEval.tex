\subsection{Evaluierung}
\label{JumpPrimefacesEval}
Unsere Erfahrung mit dem Framework Primefaces l�sst sich sehr gut in einem Satz zusammenfassen, dieser lautet: "`Einfach zu implementieren, schlecht zu personalisieren."'. Zu Beginn stellten sich sehr schnell Erfolge ein, besonders das Showcase auf der Projektseite\footnote{\url{http://www.primefaces.org/showcase/ui/home.jsf}} bietet Einsteigern genug Informationen und Anschauungsmaterial f�r eigene Umsetzungen. Durch die ebenfalls sehr gute und ausf�hrliche Beschreibung im Handbuch, die das Showcase erg�nzte, konnten weitere Komponenten in das eigene Projekt integriert werden. Jedoch gelangt man irgendwann zu dem Punkt, an dem Mann den Weg des Showcase verlassen muss und auf Grund der eigenen Anforderungen Dinge anders implementieren muss. Besonders bei der Verwendung von dynamischen Bildern, also Bildern die nicht im Web-Ordner abgelegt sind, sondern via Streamed Content geladen werden, verh�lt sich Primefaces nicht wie erwartet. Auf Grund der Tatsache, dass man wenig Feedback per Fehlermeldungen bekommt, sieht man nur, dass etwas nicht funktioniert hat, aber nicht wo der Fehler ist. Bez�glich dieser Bilder gibt es noch einen weiteren Kritikpunkt. Immer wenn solche Objekte geladen werden, f�hrt Primefaces mindestens drei Requests aus, beobachtet wurden bis zu sechs. Bei einem Controller der einen Request Scope besitzt bedeutet dies, dass genauso viele neue Controller initialisiert werden und trotzdem die R�ckgabe des Bildes nicht funktioniert hat. Deshalb mussten fast alle Controller die mit Primefaces-Komponenten in Verbindung stehen mit Session Scope ausgestattet werden. Dies f�hrt zu den bekannten Seiteneffekten, die beachtet werden m�ssen.\newline\newline
Als positve Punkte sind zu erw�hnen, dass Primefaces eine Vielzahl von Komponenten bereitstellt und diese eine breite Palette von Aufgabenstellungen abdecken. Des Weiteren ist das Design der sehr modern und kann zus�tzlich noch angepasst werden.\newline\newline
Insgesamt l�sst sich sagen, dass Primefaces sehr viel bietet und man schnell zu Ergebnissen kommt. Bei multimedialen Inhalten sind, bei der getesteten Version 2.2.1, gr��ere Schw�chen zu erkennen und daher bei diesen Aufgaben nicht zu empfehlen. Sollte man diese aber nicht ben�tigen ist Primefaces ein leichtgewichtiges Framework, welches schnell zum Ziel f�hrt. Da das Projekt weiterentwickelt wird und mit einem Milestone der Version 3 bereits eine Weiterentwicklung zur Verf�gung steht, sollte beobachtet werden, wie diese in Bezug auf Medien erweitert wird.