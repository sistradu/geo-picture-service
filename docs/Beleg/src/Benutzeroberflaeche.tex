\section{Benutzeroberfl�che}
\label{JumpBenutzeroberflaeche}
%TODO Anpassen, Facelets rein
Die Benutzeroberfl�che besteht aus drei Komponenten, HTML, YAML und Facelets. Als Grundger�st nutzen wir eine HTML-Seite die folgende Struktur besitzt. 
\begin{lstlisting}[caption=HTML-Grundger�st]{Name}
<!DOCTYPE html PUBLIC "-//W3C//DTD XHTML 1.0 Transitional//EN"
        "http://www.w3.org/TR/xhtml1/DTD/xhtml1-transitional.dtd">
<html>
<head>
	<meta http-equiv="Content-type" content="text/html; charset=utf-8"/>
	<title>GPicS</title>
	<link href="yaml/css/my_layout.css" rel="stylesheet" type="text/css"/>
</head>
<body>
</body>
</html>    
\end{lstlisting}
Zudem kommt die View-Handler-Technologie Facelets zum Einsatz, welche eine Alternative zum JavaServer Faces(JSF) Framework bildet. Mit ihr ist es m�glich eine Seite als Layout zu definieren und diese immer mit dem aktuell anzuzeigenden Inhalt zu f�llen. Facelets verf�gt �ber component-aliasing, was daf�r sorgt, dass normale HTML-Tags statt der Tags f�r UI-Komponenten genutzt werden k�nnen. Um eine Verbindung zu den jeweiligen UI-Komponenten herzustellen, reicht es aus das alias-Attribut jsfc im Tag anzugeben. Hier ein kleines Beispiel:
\begin{lstlisting}[caption=Facelets HTML-Tag]{Name}
<html   xmlns:ui="http://java.sun.com/jsf/facelets" xml:lang="en" lang="en">
<body>
<ui:insert name="content"/>
</body>
</html>
\end{lstlisting}
Wie im Beispiel zu sehen ist, wird im Body ein HTML-Tag mit dem Namen "`content"' eingef�gt. Dieses muss im weiteren Verlauf nur immer mit dem jeweiligen Inhalt versorgt werden. Dies passiert wie folgt.
\begin{lstlisting}[caption=Facelets HTML-Tag]{Name}
<?xml version="1.0" encoding="UTF-8"?>
<!DOCTYPE composition PUBLIC "-//W3C//DTD XHTML 1.0 Transitional//EN"
    "http://www.w3.org/TR/xhtml1/DTD/xhtml1-transitional.dtd">
<ui:composition xmlns="http://www.w3.org/1999/xhtml" xmlns:jsp="http://java.sun.com/jsf/composite" xml:lang="en"
                lang="en"
                template="layout.xhtml"
                xmlns:ui="http://java.sun.com/jsf/facelets"
                xmlns:f="http://java.sun.com/jsf/core"
                xmlns:h="http://java.sun.com/jsf/html"
                xmlns:p="http://primefaces.prime.com.tr/ui">

    <ui:define name="content">
    	 Hier w�rde jetzt der Content der jeweiligen Seite definiert werden
    </ui:define>

</ui:composition>
\end{lstlisting}
Man erstellt eine neue XHTML-Seite und legt auf dieser Seite im UI-Tag "`composition"' mittels der Zuweisung "`template=layout.xhtml"' das Layout, welches genutzt werden soll, fest. In diesem Fall die "`layout.xhtml"'. Der eigentliche Inhalt f�r den UI-Tag "`content"' legt man mit Hilfe des Tags "`<ui:define name="'content"'>"' fest. \\
Das hat den Vorteil das man sich nicht auf jeder Seite damit besch�ftigen muss, wo der Tag sp�ter angezeigt werden soll. Sondern man definiert ihn nur und er wird immer an der Stelle angezeigt, wo er im Template festgelegt ist.\\
\\
Wie nun das HTML-Layout gestaltet ist, wird mittels CSS durch das YAML-Framework festgelegt, welches wir bis auf ein paar kleine �nderungen in Sachen Hintergrundgrafik fast unge�ndert nutzen. Es bietet durch mehrere CSS-Style-Sheets den Vorteil in allen g�ngigen Browsern gut dargestellt werden zu k�nnen. Es enth�lt speziell f�r den Internet-Explorer angepasste Style-Sheets, da dieser mit den �blichen CSS-Befehlen nicht zurecht kommt bzw. diese immer etwas anders umsetzt. Damit ist zum Beispiel gemeint, das ein deutlicher Unterschied zwischen Firefox und Internet-Explorer besteht wenn man einen Abstand mit 10px definiert. Das kann zu signifikanten Abweichungen f�hren, welche aber durch YAML bereits abgefangen werden. 




