\chapter{Einleitung}
\label{JumpEinleitung}
In dieser Belegarbeit wird die Entwicklung einer Webanwendung unter Verwendung von XML-Technologien beschrieben. Als Fallbeispiel wurde daf�r die Erstellung eines Online-Fotoarchivs gew�hlt. Die konkrete Aufgabenstellung ist ab Seite \pageref{JumpAufgabenstellung} beschrieben. Nach der Analyse dieser, wurde ein bestehendes Content Management System\footnote{kurz: CMS} namens Drupal auf die Anforderungen hin untersucht. Als Ergebnis stand fest, dass es nicht gen�gend den Anforderungen entspricht und somit eine Neuentwicklung von N�ten ist. Im Kapitel Technologien auf Seite \pageref{JumpTechnologien} ff, wurden die verwendeten Frameworks vorgestellt. Nach der Beschreibung der Entwicklungsumgebung, die aus dem Servlet-Container Tomcat, dem Continuous Integration Server Hudson,  der IDE IntelliJ und GoogleCode besteht, folgt im Kapitel \ref{JumpSystementwurf} der Systementwurf. Auf diesen aufbauend schlie�t sich die Implementation an. Diese erstreckt sich von der Datenbank, �ber die einzelnen Seitencontroller und der Beschreibung wie die Frameworks verwendet wurden, bis hin zur Erl�uterung wie die Benutzeroberfl�che erstellt wurde. Anschlie�end folgt das Kapitel Tests, welches die Vorgehensweise zum Testen der einzelnen Komponenten beschreibt. Daran f�gt sich die Beschreibung, wie die Aufgabe unter den 4 Projektmitgliedern aufgeteilt wurde und mit welche Absicht. Abschlie�end wird eine Zusammenfassung �ber das Projekt gegeben und ein Fazit gezogen.