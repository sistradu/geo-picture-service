\chapter{Zusammenfassung}
\label{JumpZusammenfassung}
Basierend auf der Aufgabenstellung wurde das existierende Content Management System Drupal untersucht. Das Ergebnis war, dass nicht alle gestellten Anforderungen damit umgesetzt werden konnten und somit die Erstellung eines neuen Systemes von N�ten war. Die formulierten W�nsche an das System wurden in Pflicht- und Zusatzaufgaben eingeteilt. Nach der Erstellung des Systementwurfes, welcher die geforderten XML-Technologien und Frameworks ber�cksichtigte, wurde eine Evaluierung von neuen Technologien durchgef�hrt. Implementiert wurde das Projekt mit Hilfe der IDE IntelliJ. In diesem Zusammenhang wurde eine Evaluierung der Entwicklungsumgebung vollzogen. Als Ergebnis steht ein Software-Projekt, welches die Pflichtanforderungen erf�llt. Gleichzeitig wurden die verwendeten Technologien dahingehend untersucht, ob sie sich zu der Entwicklung einer solchen Anwendung eignen und welche St�rken und Schw�chen sie besitzen.\newline\newline
Die Verwendung einer nativen XML-Datenbank in Verbindung mit dem Austausch von Daten �ber XML-Dateien, eignet sich hervorragend um Webprojekte zu erstellen. Unterst�tzt durch die Entwicklungsumgebung, die dem Entwickler den Freiraum f�r Kreativit�t l�sst, da er die Umgebung individuell an seine Bed�rfnisse anpassen kann. Diese Freiheit hat einen Preis, da Lizenzkosten anfallen. Die Integration von Primefaces-Komponenten bietet eine einfache M�glichkeit komplexe Anforderungen, wie eine Integration von personalisierten GoogleMaps-Karten, umzusetzen. Allerdings ist die Dokumentation an einigen Stellen l�ckenhaft und das Verhalten nicht immer so wie erwartet. Trotz dieser M�ngel eignet sich Primefaces f�r die Entwicklung von Webanwendungen, da fr�here Projekte gezeigt haben, dass andere Frameworks dieser Art, wie RichFaces oder ICEFaces, �hnliche Schw�chen gezeigt haben.