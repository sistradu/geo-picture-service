%\subsection{Exif}
%\label{JumpExif}
\section{Exif-Metadatenextraktion}
\label{JumpMetadatenextraktion}
F�r die Darstellung der Bilder in einer GoogleMaps-Karte sind nat�rlich die GPS-Koordinaten zwingend erforderlich. Wollte man fr�her GPS-Koordinaten mit einem Bild in Verbindung bringen musste man am Ort der Aufnahme zus�tzlich noch ein GPS-Empf�nger verf�gbar haben und sich die Koordinaten notieren. Inzwischen gibt es aber Kameras, die die Koordinaten als Metadaten zu einem Bild mit abspeichern. Die in diesem Projekt verwendete Kamera speichert die Bilder im JPEG-Format. Bei JPEG gibt es die M�glichkeit, Metadaten mittels einem Exif-Datenblock abzuspeichern. Auf diese Weise kann man auf die GPS-Daten zugreifen.\\
\\
Um nun mittels eines Programms die Daten zu extrahieren gibt es zwei M�glichkeiten. Die erste ist ein selbst entwickeltes Framework, mit denen man die Daten abgreift. Dabei muss eine ausf�hrlich Analyse des Bild-Formats durchgef�hrt werden. Aufgrund der knappen Zeit und unter der Beachtung der Aufgabenstellung, dass ein Fotoarchiv und nicht ein Framework zum extrahieren von Exif-Daten entwickelt werden sollte, haben wir die zweite M�glichkeit gew�hlt. Dies war die Verwendung eines bestehenden Frameworks. Um solche Frameworks zu finden bietet sich immer eine Internetrecherche an. Diese ergab, das mittels des Frameworks "`Metadata-Extractor"' die Anforderungen an ein solches Framework erf�llt wurden. Anhand des Quickstart-Guides ist zu erkennen, dass eine Verwendung obendrein relativ einfach ist.\\
\\
Das Framework bietet vorgefertigte Methoden, um die Metaddaten auszulesen. Dazu werden die Daten in einem Metadata-Objekt abgespeichert. Mittels eines Iterators kann man anschlie�end durch s�mtliche Metadaten-Tags iterieren.
\begin{lstlisting}[caption=Extrahieren von Exif-Daten aus einem Bild]
Metadata metadata = JpegMetadataReader.readMetadata(file);
Iterator directories = metadata.getDirectoryIterator();
while (directories.hasNext()) {
Directory directory = (Directory) directories.next();
Iterator tags = directory.getTagIterator();
...
}
\end{lstlisting}