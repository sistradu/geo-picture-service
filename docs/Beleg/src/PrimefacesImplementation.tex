\section{Primefaces}
\label{JumpPrimefacesImplementation}
\subsection{FileUpload}
F�r das Hochladen der Bilder wird im Projekt die File-Upload Komponente von PrimeFaces verwendet. Dadurch konnte diese Anforderung relativ schnell umgesetzt werden. Zu Beginn traten allerdings einige Probleme auf. So ben�tigt diese Komponente eine Reihe von Biblotheken damit der Upload funktioniert. Dies wurde allerdings weder auf der PrimeFaces-Webseite noch in der PDF-Dokumentation erw�hnt. Letztendlich musste eine Recherche im Internet gemacht werden, wo dann in einem Forum erw�hnt wurde, dass die Bibliotheken "`commons-logging"', "`commons-io"' und "`commons-fileupload"' ben�tigt werden. Diese sind alle Projekte der Apache Foundation und somit trat bei der Verwendung auch keine lizenzrechtlichen Probleme auf.\\
\\
Grundvoraussetzung f�r die Verwendung der Fileupload-Komponente ist, dass einige zus�tzliche Filter in die web.xml eingetragen werden. Welche Filter einzubinden sind, findet man in der PDF-Dokumentation von Primefaces.
\begin{lstlisting}[caption=Fileupload-Filter]
<filter>
        <filter-name>PrimeFaces FileUpload Filter</filter-name>
        <filter-class>
            org.primefaces.webapp.filter.FileUploadFilter
        </filter-class>
    </filter>
    <filter-mapping>
        <filter-name>PrimeFaces FileUpload Filter</filter-name>
        <servlet-name>Faces Servlet</servlet-name>
    </filter-mapping>
\end{lstlisting}
Ansonsten ist die Verwendung der Komponente ziemlich einfach. Als erstes muss die Komponente in eine HTML-Seite eingebunden werden.
\begin{lstlisting}[caption=Fileupload-Komponente einbinden, label=lstFileupload]
<p:fileUpload widgetVar="uploader" customUI="true" label="#{msg.uploadImage}" description="*.jpg;*.JPG" allowTypes="*.jpg;*.JPG" fileUploadListener="#{createEditAlbumController.handleFileUpload}" multiple="true" id="fileUploader" update="messages, cgrid"/>
\end{lstlisting}
Es ist sogar m�glich die Schaltfl�chen umzubenennen. Diese M�glichkeit wurde nat�rlich genutzt. Wie in dem Listing \autoref{lstFileupload} zu sehen ist, wurden die Beschriftungen allerdings nicht fest reingeschrieben sondern mit einem Eintrag in der Message.properties verkn�pft. Au�erdem wurden nur Dateien f�r den Upload zugelassen, die die Endung "`.jpg"' bzw. "`.JPG"' haben. Sobald man ein Bild hochl�dt, wird ein FileUploadEvent ausgel�st. Mit dem entsprechenden Event-Objekt kann man auf die Daten zugreifen. 
%\begin{lstlisting}[caption=FileUploadEvent]
%public void handleFileUpload(FileUploadEvent event) {
        %UploadedFile file = event.getFile();
        %try {
            %...
            %out = new FileOutputStream(uploadDir + username + "_" + file.getFileName());
            %out.write(file.getContents());
            %out.flush();
            %out.close();
            %....
    %}
%\end{lstlisting}
%TODO Referenz einf�gen
Die genaue Implementation des FileUploadEvents wird noch im Abschitt beschrieben.
%Wir haben uns dazu entschieden das Bild im Dateisystem abzuspeichern und in der Datenbank lediglich den Pfad abzuspeichern. Der Grund daf�r ist, dass sonst bei der Speicherung in der Datenbank die Bilder erst aufwendig encodiert werden m�ssten. Sobald dann sp�ter die Bilder aus der Datenbank geholt werden, m�sste man au�erdem wieder die Daten zu einem JPEG-Bild konvertieren. Diesen Mehraufwand konnten wir durch die Speicherung im Dateisystem umgehen. Au�erdem gibt es mit unserem Verfahren weniger Probleme wenn mehrere Bilder hochgeladen werden.\\
%\\
%Das Upload-Verzeichnis an sich liegt irgendwo im Dateisystem. Die einfachste M�glichkeit w�re zwar der web-Ordner der Applikation gewesen, aber dort w�re die Bilder bei jedem Neudeploy gel�scht worden. Aus diesem Grund gibt es ein extra Verzeichnis wo alle Bilder abgespeichert werden.

\subsection{GMap}
Primefaces bietet noch weitere n�tzliche Komponenten. So wird bereits eine Komponente zum Anzeigen einer GMap-Karte bereitgestellt. Um diese einzubinden muss nur der in Listing \autoref{lstGMap} zu sehende Code in die Webseite eingebunden werden.
\begin{lstlisting}[caption=GMap-Tag, label=lstGMap]
<p:gmap center="#{albumController.bilder[0].latitude}, #{albumController.bilder[0].longitude}" zoom="13" type="HYBRID" style="width:600px;height:400px"
                    model="#{mapBean.simpleModel}" overlaySelectListener="#{mapBean.onMarkerSelect}">
</p:gmap>
\end{lstlisting}
Voraussetzung f�r die Verwendung des GMap-Tags ist allerdings der folgende JavaScript-Code, mit dem die GoogleMaps-API eingebunden wird.
\begin{lstlisting}[caption=Einbinden der GoogleMaps-API]
<script src="http://maps.google.com/maps/api/js?sensor=true" type="text/javascript"></script>
\end{lstlisting}
F�r das Anzeigen der Bilder in der GMap-Karte gibt es verschiedene M�glichkeiten. So wird bereits in dem Showcase auf der Primefaces-Webseite vorgeschlagen f�r jedes Bild ein Marker zu verwenden und bei einem Klick auf den Marker eine Sprechblase mit dem Bild anzuzeigen. Daf�r muss zwischen den �ffnenden und den schlie�enden GMap-Tag der Inhalt des Listings \autoref{lstInfoWindow} eingetragen werden.
\begin{lstlisting}[caption=GMap-InfoWindow, label=lstInfoWindow]
<p:gmapInfoWindow>
                        <p:graphicImage value="#{mapBean.image}" width="200" height="150"/>
                        <br/>
                        <h:outputText value="#{mapBean.beschreibung}"/>
                    </p:gmapInfoWindow>
\end{lstlisting}
Das Hinzuf�gen der Marker und was bei einem Klcik auf einem Marker passiert, muss in der dazu geh�rigen ManagedBean erfolgen. Die wird in Abschnitt ???????????? noch erl�utert.\\
\\
Eine andere M�glichkeit w�re, die Bilder direkt in der GMap anzuzeigen. In einem Wegwerfprojekt, das durchgef�hrt wurde um Primefaces genauer kennen zu lernen, wurde festgestellt, dass die m�glich ist. Dazu m�ssten die Marker nicht mehr das Bild des Markers anzeigen sondern das gew�nschte Bild. Es wurde allerdings festgestellt, dass dies nur mit Bildern gehen w�rde, die im Web-Ordner des Tomcats zur Verf�gung stehen. Dazu m�sste der Pfad bekannt sein. Mit dynamisch gestreamten Bildern von der Festplatte war das Ersetzen des Marker-Bildes nicht m�glich. Au�erdem stellte sich bei diesem Versuch heraus, dass ein Anzeigen der Bilder direkt in der Map zu viel von der eigentlichen Map verdeckt. Daher wird die erste M�glichkeit in diesem Projekt verwendet.