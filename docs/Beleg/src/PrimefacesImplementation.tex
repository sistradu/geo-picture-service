\section{Primefaces}
\label{JumpPrimefacesImplementation}
\subsection{File-Upload}
F�r das Hochladen der Bilder wird im Projekt die File-Upload Komponente von PrimeFaces verwendet. Dadurch konnte diese Anforderung relativ schnell umgesetzt werden. Zu Beginn traten allerdings einige Probleme auf. So ben�tigt diese Komponente eine Reihe von Biblotheken damit der Upload funktioniert. Dies wurde allerdings weder auf der PrimeFaces-Webseite noch in dem PDF erw�hnt. Letztendlich musste eine Recherche im Internet gemacht werden, wo dann in einem Forum erw�hnt wurde, welche Bibliotheken ben�tigt werden.\\
\\
Ansonsten ist die Verwendung der Komponente ziemlich einfach. Es ist sogar m�glich die Schaltfl�chen umzubenennen. Diese M�glichkeit wurde nat�rlich genutzt. Sobald man ein Bild hochl�dt, wird ein FileUploadEvent ausgel�st. Mit dem entsprechenden Event-Objekt kann man auf die Daten zugreifen. Wir haben uns dazu entschieden das Bild im Dateisystem abzuspeichern und in der Datenbank lediglich den Pfad abzuspeichern. Der Grund daf�r ist, dass sonst bei der Speicherung in der Datenbank die Bilder erst aufwendig encodiert werden m�ssten. Sobald dann sp�ter die Bilder aus der Datenbank geholt werden, m�sste man au�erdem wieder die Daten zu einem JPEG-Bild konvertieren. Diesen Mehraufwand konnten wir durch die Speicherung im Dateisystem umgehen. Au�erdem gibt es mit unserem Verfahren weniger Probleme wenn mehrere Bilder hochgeladen werden.\\
\\
Das Upload-Verzeichnis an sich liegt irgendwo im Dateisystem. Die einfachste M�glichkeit w�re zwar der web-Ordner der Applikation gewesen, aber dort w�re die Bilder bei jedem Neudeploy gel�scht worden. Aus diesem Grund gibt es ein extra Verzeichnis wo alle Bilder abgespeichert werden.