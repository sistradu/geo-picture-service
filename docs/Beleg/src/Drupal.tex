\chapter{Evaluation eines bestehenden Systems - Drupal}
\label{JumpDrupal}

Viele Menschen wollten schon immer ihre eigene Website, ihren Blog, ihr Forum oder ähnliches haben, doch sind immer an den HTML- und Programmierkenntnissen gescheitert. Das ist seit dem Beginn von Content-Management-Systemen (CMS) vorbei. Solche Systeme ermöglichen es in den meisten Fällen ohne Programmier- und HTML-Kenntnisse eine Website im Handumdrehen zu erstellen. Bei einigen Webspaces oder Online-Hostern sind solche Systeme schon bereits vorinstalliert und können schnell und einfach genutzt werden oder sie sind dafür geeignet und man muss sein CMS nur dort hochladen. Zu den bekanntesten Content Management Systemen zur Zeit zählen Typo3, Joomla und mittlerweile auch Drupal. Da Drupal immer stärker im kommen ist und einen immer größer werdenden Zuwachs findet, wird Drupal im nächsten Abschnitt mal etwas genauer beleuchtet werden. \\
\\
Drupal ist ein CMS und Framework, welches ursprünglich vom belgischen Informatiker Dries Buytaert entwickelt wurde. Es handelt sich dabei um eine freie Software, die unter der General Public License (GNU) steht. \\
\\
Wer Drupal nutzen möchte benötigt dafür lediglich einen Webserver, auf dem sich PHP und eine SQL-fähige Datenbank befindet. Das sind alles Dinge, die die meisten Webspaces kostenlos oder für einen kleinen Preis zur Verfügung stellen. PHP ist eine Skriptsprache, die hauptsächlich zur Erstellung dynamischer Webseiten oder Webanwendungen verwendet wird. Sie wird deshalb benötigt, da Drupal in PHP programmiert wurde.\\
\\
Der Aufbau von Drupal ist sehr einfach und lässt sich wie folgt beschreiben. Drupal besteht aus zwei Teilen, einem Core(Kern) und Modulen. Der Core beinhaltet die Grundfunktionalität, welche mit weiteren Modulen erweitert werden kann. Zur Grundfunktionalität zählen Komponenten wie Template-Erstellung, Blogsystem, Benutzerverwaltung und Taxonomie. Diese Core-Funktionen reichen aus um eine simple Website zu erstellen. Man kann mit Hilfe der Template-Erstellung festlegen in welchen Bereichen der Website welcher Content angezeigt werden soll und auch die farbliche Gestaltung der gesamten Seite definieren. Die Benutzerverwaltung von Drupal ist sehr ausführlich und detailliert. Sie zieht sich über mehrere Seiten und man kann dort Benutzergruppen definieren und ihnen Rechte zuweisen. Die Rechte können von Artikelarten die ein Benutzer schreiben darf, über die Bilderanzahl pro Artikel bis hin zu Kommentarfunktion und vielen mehr gehen. Mit hilfe von Taxonomie lassen sich Menüs erstellen und verschiedene Artikel bzw. Beiträge der Website einer Kategorie zuweisen. Das kann über vorkommende Begriffe in Beiträgen oder über eine Auswahl der Kategorie beim Schreiben des Artikels passieren. \\
\\
Der andere Teil neben dem Core sind die Module. Die sind dafür da um je nach Anforderung Funktionen nachzurüsten. Diese Module werden meist von anderen Nutzer geschrieben und stehen dann allen zur Verfügung und können per Download integriert werden.
Was dabei zu beachten ist, ist die Menge der Module. Es gibt eine sehr große Anzahl und man kann nicht immer gleich am Modulnamen erkennen, was dieses Modul kann bzw. welche Funktion es liefern soll. Zudem wurde in mehreren Test festgestellt, das viele Module noch nicht richtig Funktionieren und sich oft noch im Beta-Status befinden.\\
\\
So muss man als Fazit sagen, dass Drupal eine echte Alternative in Sachen CMS ist zur schnellen und einfachen Erstellung einer Website oder eines WebBlogs, aber im Hinblick auf die Erstellung eines Foto-Archivs noch als eher ungeeignet darstellt. Da viele Funktionen die man in einem Foto-Archiv erwartet, wie eine Diashow oder eine Albumübersicht, Übersicht der Bilder nur schwer bis gar nicht realisieren lassen. Das liegt daran, dass dafür weitere Module benötigt werden, die zahlreich vorhanden sind, aber leider nicht zusammen arbeiten. So kann es sein das man nach langem Suchen ein passendes Modul finden aus der Vielzahl der Module, diese aber dann nicht mit den weiteren benötigten Modulen funktionieren. Das kommt daher, das jeder der sich mit Programmierung auskennt, sein eigenes Modul schreiben und veröffentlichen kann. Da aber jeder sein Modul so schreibt, dass es für seine Zwecke funktioniert, kommt es immer wieder zu Konflikten untereinander.
