\section{Hudson}
\label{JumpHudson}
Ein weiterer Bestandteil der Entwikcklungsumgebung war der Continuous Integration Server Hudson. Dieser wird �ber Tomcat gestartet und in Verbindung mit der Versionsverwaltungssystem bietet es die M�glichkeit automatisiert die Anwendung nach einem commit neu zu deployen und zu testen. Daraus ergeben sich Vorteile, wie z.B. einer stetig aktuellen und verf�gbaren Version der Anwendung, die automatisierte Ausf�hrung von Test �ber ein Ant-Skript, sowie sofortiges Feedback dar�ber, ob Integrations-Probleme bestehen. Die Ergebnis eines neuen Buildvorganges von Hudson l�sst sich per E-Mail an die Projektmitglieder versenden oder �ber einen RSS-Feed abrufen.\newline\newline
Die Entscheidung zu Gunsten von Hudson fiel, da in einem Vortrag �ber die Entwicklung eines Web-Projektes mit diesem System gute Erfahrungen gemacht wurden. �hnliches gilt auch f�r dieses, die Installation ist und Konfiguration erfordert etwas Zeit, wenn man dies zum ersten Mal macht. Allerdings �berwiegen die genannten Vorteile. Somit kann die Verwendung von Hudson f�r ein ��hnliches Projekt wie in dieser Belegarbeit empfohlen werden.