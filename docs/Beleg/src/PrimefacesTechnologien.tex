\section{Primefaces}
\label{JumpPrimefacesTechnologien}
In heutigen Webanwendungen werden h�ufig viele Technologien verwendet um eine ansprechende Benutzeroberfl�che zu realisieren. Da gewisse Anforderungen immer wieder auftreten, wie zum Beispiel ein Dateiupload, ist es w�nschenswert ein Framework zu verwenden. Wie in Abschnitt \ref{JumpJSF} erl�utert wird, ist das Basis-Framework f�r dieses Projekt JSF. Mit diesem k�nnen zwar grunds�tzlich alle Anforderungen erf�llt werden (u.a. Dateiupload), doch es ist immer eine Erleichterung, wenn es daf�r schon vorgefertigte Komponenten gibt. In diesem Projekt wird Primefaces verwendet. Damit k�nnen alle Anforderungen schnell umgesetzt werden.\\
\\
Au�erdem soll nat�rlich die Benutzeroberfl�che immer ansprechend gestaltet sein. Bei Primefaces sind die Komponenten schon mit einem h�bschen Design vorhanden. Dieses vorgefertigte Design muss nicht unbedingt einen Nachteil bieten, und in diesem Projekt ist es auch kein Nachteil, da dadurch ein aufwendiger Designentwurf f�r diese Komponenten entfallen konnte. Es ist sogar m�glich dieses Design ein wenig zu beeinflussen.\\
\\
Wie jeder wei� gibt es nat�rlich eine ganze Reihe solcher Frameworks, die einen das Leben erleichtern. An dieser Stelle w�ren unter anderem RichFaces und IceFaces zu nennen. Aber unsere Wahl ist trotzdem auf Primefaces gefallen, weil es zum Einem eine Anforderung war Primefaces zu nutzen und zum Anderem bietet Primefaces n�tzliche Funktionen f�r dieses Projekt. So bietet Primefaces bereits eine Komponente f�r die Einbindung einer GoogleMaps-Karte. Mit dieser Komponente ist es au�erdem m�glich, Bilder in eine GoogleMaps-Karte anzuzeigen. Mit Primefaces ist es m�glich die Kernanforderung, ein Fotoarchiv zuerstellen, das Bilder in einer GoogleMaps-Karte anzeigen kann, umzusetzen. Dies erspart eine aufwendige Neuentwicklung.\\
\\
Primefaces wird in diesem Projekt in der Version 2.2.1 verwendet. Dies ist die aktuelle stabile Version. Im Moment befindet sich die Version 3 in Entwicklung und es gibt auch schon Vorabversionen davon. Es wurde trotzdem die Version 2.2.1 verwendet, da einige Projektmitglieder bereits schlechte Erfahrungen mit Vorabversionen in anderen Projekten gemacht haben.