\section{Anforderungsanalyse}
\label{JumpAnforderungsanalyse}
Basierend auf der Aufgabenstellung, sollen nun die kongreten Anforderungen formuliert werden. Dabei wird unterschieden zwischen funktionalen und nicht-funktionalen Anforderungen.
\subsection{Funktionale Anforderungen}
Funktionale Anforderungen beschreiben Funktionen, die im fertigen System vorhanden sein sollten. Zus�tzlich kann unterschieden werden, ob eine Anforderung zu 100\% umgesetzt werden soll oder nur teilweise, bzw. nur wenn der zeitliche Rahmen es zul�sst.\\
Die Darstellung der Anforderungen erfolgt in Form von Usecases, welche im Anhang unter \autoref{JumpUseCases} zu finden sind.
\subsection{Optionale Anforderungen}
\begin{itemize}
	\item Umkreissuche von Bildern:
	\item[] Wenn sich ein Betrachter f�r eine Region besonders interessiert, so kann er sich alle sichtbaren Bilder im Umkreis eines angegebenen Punktes auf der Karte anzeigen lassen, auch von mehreren Alben und Nutzern.
	\item Routenplaner von Bild zu Bild:
	\item[] Bei einer langen Distanz zwischen 2 Bildern, sollen Cloud-Services von Google genutze werden, um sich die optimale Route zwischen diesen Bildern berechnen zu lassen.
	\item Entfernung zwischen 2 Fotos:
	\item[] Auch hierf�r k�nnen Cloud-Services genutzt werden um entweder die Luftlinienentfernung oder die L�nge der optimalen Route zwischen 2 Fotos anzugeben.
\end{itemize}
\subsection{Nicht-Funktionale Anforderungen}
Zu den nicht-funktionalen Anforderungen z�hlen alle Anforderungen, die die grundlegende Funktionalit�t des Systems nicht beeinflussen.\\
F�r unser System sind das die Folgenden:
\begin{itemize}
	\item Datensicherheit:
	\item[] Die vom Nutzer eingegebenen Daten m�ssen sicher gespeichert werden, so dass diese nicht von Dritten ausgelesen werden k�nnen. Das betrifft vor allem, die vom Nutzer hochgeladenen Bilder, welche als privat gekennzeichnet wurden.
	\item Passwortsicherheit:
	\item[] Das Passwort eines Nutzers soll bereits verschl�sselt �bertragen werden, damit auch beim Abh�ren des Datenverkehrs, dieses nicht rekonstruiert werden kann.
	\item Usability:
	\item[] Das System soll �ber eine ansprechende Oberfl�che leicht zu bedienen sein. Der Nutzer muss zu jedem Zeitpunkt alle ihm m�glichen Aktionen schnell und problemlos finden k�nnen.
	\item Mehrbenutzerf�higkeit:
	\item[] Auf einem System k�nnen sich mehrere Nutzer anmelden und gleichzeitig ihre Bilder verwalten, ohne dass Probleme durch nebenl�ufige Prozesse auftreten.
\end{itemize}
%\subsection{Anforderungen des Anwendungsbereiches}
%Anforderungen des Anwendungsbereiches charakterisieren Spezialit�ten des %Anwendungsbereiches
%??? Linux, wenig Arbeitsspeicher und kleine Festplatte, Nutzerverwaltung %DB-basiert