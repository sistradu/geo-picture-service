\section{Anforderungsanalyse}
\label{JumpAnforderungsanalyse}
Basierend auf der Aufgabenstellung, sollen nun, die kongreten Anforderungen formuliert werden. Dabei wird unterschieden zwischen funktionalen und nicht-funktionalen Anforderungen.
\subsection{Funktionale Anforderungen}
Funktionale Anforderungen beschreiben Funktionen, die im fertigen System vorhanden sein sollten. Zus�tzlich kann unterschieden werden, ob eine Anforderung zu 100\% umgesetzt werden soll oder nur teilweise, bzw. nur wenn der zeitliche Rahmen es zul�sst.\\
Die Darstellung der Anforderungen erfolgt in Form von Usecases, welche im Anhang unter \autoref{JumpUseCases} zu finden sind.
\begin{itemize}
	\item Einfache Nutzerverwaltung:
	\item[] Das System muss eine einfache Nutzerverwaltung implementieren. Dabei soll zwischen Gast, Uploader und Admin unterschieden werden, welche folgende Rechte besitzen:\\
	
	\begin{tabular}{|p{5cm}||c|c|c|}
		\hline
		 & \textbf{Gast} & \textbf{Uploader} & \textbf{Admin} \\
		 \hline
		 \hline
		 \textbf{�ffentliche Bilder ansehen} & Ja & Ja & Ja \\
		 \hline
		 \textbf{Private Bilder ansehen} & Nur mit Passwort & Ja & Ja \\
		 \hline
		 \textbf{Bilder verwalten} & Nein & Ja & Ja \\
		 \hline
		 \textbf{Benutzer l�schen} & Nein & Nein & Ja \\
		 \hline
	\end{tabular}

	\item Organisation der Fotos:
	\item[] Angemeldete Benutzer k�nnen Alben erstellen, zu denen sie mehrere Bilder hochladen k�nnen. Die Alben k�nnen auch nachtr�glich noch bearbeitet werden und Bilder hinzugef�gt oder gel�scht werden.
	\item Erweiterung der Bilddaten mit Informationen:
	\item[] Aus hochgeladenen Bildern sollen automatisch Metadaten, wie Zeit der Aufnahme und GPS-Informationen ausgelesen werden. Zus�tzlich kann der Nutzer zu jedem Bild Kommentare hinzuf�gen und auch die ausgelesenen Daten nachtr�glich bearbeiten oder hinzuf�gen, falls diese nicht ausgelesen werden konnten. Zus�tzlich kann der Nutzer Bilder �ffentlich machen, damit jeder sie betrachten kann.
	\item Anzeige von Bildern/Alben:
	\item[] Alle sichtbaren Bilder zu einem Album k�nnen direkt als Slideshow oder alternativ auf einer Google-Maps Karte mit integrierten Thumbnails angezeigt werden.
	\item L�schen von Alben:
	\item[] Der Admin des Fotoarchivs, kann Alben l�schen, wenn die Inhalte gegen die Nutzungsbedingungen versto�en.
\end{itemize}
\subsection{Optionale Anforderungen}
\begin{itemize}
	\item Umkreissuche von Bildern:
	\item[] Wenn sich ein Betrachter f�r eine Region besonders interessiert, so kann er sich alle sichtbaren Bilder im Umkreis eines angegebenen Punktes auf der Karte anzeigen lassen, auch �ber mehrere Alben und Nutzer.
	\item Routenplaner von Bild zu Bild:
	\item[] Bei einer lange Distanz zwischen 2 Bildern, sollen Cloud-Services von Google genutze werden, um sich die optimale Route zwischen diesen Bildern berechnen zu lassen.
	\item Entfernung zwischen 2 Fotos:
	\item[] Auch hierf�r k�nnen Cloud-Services genutzt werden um entweder die Luftlinienentfernung, oder die L�nge der optmialen Route zwischen 2 Fotos anzugeben.
\end{itemize}
\subsection{Nicht-Funktionale Anforderungen}
Zu den nicht-funktionalen Anforderungen z�hlen alle Anforderungen, die die grundlegende Funktionalit�t des Systems nicht beeinflussen.\\
F�r unser System sind das die folgenden:
\begin{itemize}
	\item Datensicherheit:
	\item[] Die vom Nutzer eingegebenen Daten m�ssen sicher gespeichert werden, so dass diese nicht von dritten ausgelesen werden k�nnen. Das betrifft vor allem, die vom Nutzer hochgeladenen Bilder, welche als Privat gekennzeichnet wurden.
	\item Passwortsicherheit:
	\item[] Das Passwort eines Nutzers soll bereits verschl�sselt �bertragen werden, damit auch beim Abh�ren des Datenverkehrs, dieses nicht rekonstruiert werden kann.
	\item Usability:
	\item[] Das System soll �ber eine ansprechende Oberfl�che leicht zu bedienen sein. Der Nutzer muss zu jedem Zeitpunkt alle ihm m�glichen Aktionen schnell und problemlos finden k�nnen.
	\item Mehrbenutzerf�higkeit:
	\item[] Auf einem System k�nnen sich mehrere Nutzer anmelden und gleichzeitig ihre Bilder verwalten ohne, dass Probleme durch Nebenl�ufige Prozesse auftreten.
\end{itemize}
%\subsection{Anforderungen des Anwendungsbereiches}
%Anforderungen des Anwendungsbereiches charakterisieren Spezialit�ten des %Anwendungsbereiches
%??? Linux, wenig Arbeitsspeicher und kleine Festplatte, Nutzerverwaltung %DB-basiert