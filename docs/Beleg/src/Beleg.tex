%%%%%%%%%%%%%%%%%%%%%%%%%%%%%%%%%%%%%%%%%%%%%%%%%%%%%%%%%%%%%%%%%%%%%%%
%% Optionen zum Layout des Buchs                                     %%
%%%%%%%%%%%%%%%%%%%%%%%%%%%%%%%%%%%%%%%%%%%%%%%%%%%%%%%%%%%%%%%%%%%%%%%
\documentclass[
a4paper,							% alle weiteren Papierformat einstellbar
%landscape,						% Querformat
11pt,								% Schriftgr��e (12pt, 11pt (Standard))
%BCOR1cm,							% Bindekorrektur, bspw. 1 cm
%DIVcalc,							% f�hrt die Satzspiegelberechnung neu aus
%											  s. scrguide 2.4
%oneside,							% einseitiges Layout
%twocolumn,						% zweispaltiger Satz
%openany,							% Kapitel k�nnen auch auf linken Seiten beginnen
%halfparskip*,				% Absatzformatierung s. scrguide 3.1
%headsepline,					% Trennline zum Seitenkopf	
%footsepline,					% Trennline zum Seitenfu�
%notitlepage,					% in-page-Titel, keine eigene Titelseite
%chapterprefix,				% vor Kapitel�berschrift wird "Kapitel Nummer" gesetzt
%appendixprefix,				% Anhang wird "Anhang" vor die �berschrift gesetzt 
%normalheadings,			% �berschriften etwas kleiner (smallheadings)
%idxtotoc,						% Index im Inhaltsverzeichnis
%liststotoc,					% Abb.- und Tab.verzeichnis im Inhalt
%bibtotoc,						% Literaturverzeichnis im Inhalt
%bibtotocnumbered,			% Nummerierstes Literaturverzeichnis im Inhalt 
%leqno,								% Nummerierung von Gleichungen links
%fleqn,								% Ausgabe von Gleichungen linksb�ndig
%draft,								% �berlangen Zeilen in Ausgabe gekennzeichnet
titlepage,
pdftex
]
{scrreprt}
%{scrartcl}


\newcommand{\todo}[1]{    
	\addcontentsline{tdo}{todo}{\protect{#1}}    
	\marginpar{\textbf{\textcolor{red}{#1}}}
	}

\newcommand{\GPicS}{
	\emph{GP}ic\emph{S}
}
\newcommand{\GPicSLogo}{
	\includegraphics[height=\heightof{X}]{img/GPicS_Logo_Neu.PNG}
}
\makeatletter \newcommand \listoftodos{\section*{Todo Liste} \@starttoc{tdo}}  \newcommand\l@todo[2]    {\par\noindent \textit{#2}, \parbox{10cm}{#1}\par} 


\setcounter{secnumdepth}{4}
\setcounter{tocdepth}{4}

\usepackage[ngerman, final]{hyperref}
\usepackage[ngerman]{babel}
\usepackage[T1]{fontenc}       
\usepackage[latin1]{inputenc}
%\usepackage{graphicx}
\usepackage{multicol}
\usepackage{amsmath}
%\usepackage[mediumspace,mediumqspace,Grey,squaren]{SIunits}
%\usepackage{multirow}
\usepackage{color,listings}
\usepackage{ltxtable}
\usepackage{longtable}
\usepackage{tabularx}
\usepackage{multirow}
\usepackage{lscape}
\usepackage{pdfpages}
\usepackage{array}

\lstset{language=Java, % Grundsprache ist C und Dialekt ist Sharp (C#)
captionpos=b, % Beschriftung ist unterhalb
frame=lines, % Oberhalb und unterhalb des Listings ist eine Linie
%basicstyle=\ttfamily, % Schriftart
basicstyle=\small, % Schriftart
keywordstyle=\color{blue}, % Farbe f�r die Keywords wie public, void, object u.s.w.
commentstyle=\color{green}, % Farbe der Kommentare
stringstyle=\color{red}, % Farbe der Zeichenketten
numbers=none, % Zeilennummern links vom Code
numberstyle=\tiny, % kleine Zeilennummern
numbersep=5pt,
breaklines=true, % Wordwrap a.k.a. Zeilenumbruch aktiviert
showstringspaces=false,
% emph legt Farben f�r bestimmte W�rter manuell fest
emph={double,bool,int,unsigned,char,true,false,void},
emphstyle=\color{blue},
emph={Assert,Test},
emphstyle=\color{red},
emph={[2]\using,\#define,\#ifdef,\#endif}, emphstyle={[2]\color{blue}}
}

\hypersetup
{
pdftitle = {GPicS},
pdfsubject = {GPicS},
pdfauthor = {Rico Scholz,Stefan Radusch,Martin Schicht,Markus Ullrich},
pdfkeywords = {XML,GPicS,Fotoarchiv},
colorlinks = {false},
pdfborder = 0 0 0
}
\urlstyle{same}
\title{\GPicS}
\subtitle{Entwicklung eines XML basierten Fotoarchiv}
\author{Rico Scholz\\IIAm10 \\ Fachbereich Informatik \\
Hochschule Zittau / G�rlitz
\and Stefan Radusch\\IIAm10 \\ Fachbereich Informatik \\
Hochschule Zittau / G�rlitz
\and Martin Schicht\\IIAm10 \\ Fachbereich Informatik \\
Hochschule Zittau / G�rlitz
\and Markus Ullrich\\IIAm10 \\ Fachbereich Informatik \\
Hochschule Zittau / G�rlitz}
\date{Lehrveranstaltungsleiter:\\
Prof. Dr. rer. nat. Christian Wagenknecht\\
\bigskip\today}
	
\begin{document}
	\newcommand{\markiert}{\bf\emph}
	\renewcommand{\figurename}{Abbildung}
	\renewcommand{\refname}{B�cher}
%	\addto\captionsngerman{\renewcommand{\abstractname}{\Large{Zusammenfassung}}}
	\newenvironment{nosepitemize}{\begin{itemize}\itemsep 0pt}{\end{itemize}}
	\newcolumntype{C}[1]{>{\centering\arraybackslash}m{#1}}
	\maketitle

	\tableofcontents
	\listoffigures
	\listoftables
	\lstlistoflistings
	\listoftodos
	\newpage
	\section*{Abstract}
Diese Arbeit befasst sich mit der Erstellung eines Fotoarchivs. Basierend auf dieser Aufgabenstellung wird das existierende Content Management System Drupal untersucht. Das Ergebnis ist, dass nicht alle gestellten Anforderungen damit umgesetzt werden k�nnen und somit die Erstellung eines neuen Systemes erforderlich war. Die formulierten W�nsche an das System sind in Pflicht- und Zusatzaufgaben kategorisiert. Bei Erstellung des Systementwurfes, sind die geforderten XML-Technologien und Frameworks eingeflossen. Implementiert wird das Projekt mit Hilfe der Programmierungebung IntelliJ. In diesem Zusammenhang soll untersucht werden, wie diese und auch andere Technologien sich dazu eignen eine Webanwendung zu entwickeln. Das Ergebnis ist ein Software-Projekt, welches die gestellten Pflichtanforderungen erf�llt.
	\chapter{Einleitung}
\label{JumpEinleitung}
In dieser Belegarbeit wird die Entwicklung einer Webanwendung unter Verwendung von XML-Technologien beschrieben. Als Fallbeispiel wurde daf�r die Erstellung eines Online-Fotoarchivs gew�hlt. Die konkrete Aufgabenstellung ist ab Seite \pageref{JumpAufgabenstellung} beschrieben. Nach der Analyse dieser, wurde ein bestehendes Content Management System\footnote{kurz: CMS} namens Drupal auf die Anforderungen hin untersucht. Als Ergebnis stand fest, dass es nicht gen�gend den Anforderungen entspricht und somit eine Neuentwicklung von N�ten ist. Im Kapitel Technologien auf Seite \pageref{JumpTechnologien} ff, wurden die verwendeten Frameworks vorgestellt. Nach der Beschreibung der Entwicklungsumgebung, die aus dem Servlet-Container Tomcat, dem Continuous Integration Server Hudson,  der IDE IntelliJ und GoogleCode besteht, folgt im Kapitel \ref{JumpSystementwurf} der Systementwurf. Auf diesen aufbauend schlie�t sich die Implementation an. Diese erstreckt sich von der Datenbank, �ber die einzelnen Seitencontroller und der Beschreibung wie die Frameworks verwendet wurden, bis hin zur Erl�uterung wie die Benutzeroberfl�che erstellt wurde. Anschlie�end folgt das Kapitel Tests, welches die Vorgehensweise zum Testen der einzelnen Komponenten beschreibt. Daran f�gt sich die Beschreibung, wie die Aufgabe unter den 4 Projektmitgliedern aufgeteilt wurde und mit welche Absicht. Abschlie�end wird eine Zusammenfassung �ber das Projekt gegeben und ein Fazit gezogen.
	\chapter{Problem}
\label{JumpProblem}
\section{Aufgabenstellung}
\label{JumpAufgabenstellung}
Das Ziel des Projektes ist es, ein Online-Foto-Archiv zu entwickeln, in dem Benutzer Bilder ins Internet stellen k�nnen. Dabei soll es eine M�glichkeit geben, um zu entscheiden, welche �ffentlich oder nach einer Authentifizierung zug�nglich sind. Mehrere Fotos sollen zu einem Album zusammengefasst und gesammelt angezeigt werden k�nnen. Weiterhin sollen Metainformationen aus den Bildern extrahiert und bearbeitet werden k�nnen. Diese sollen in geeigneter Form, z.B. einer Karte der Region, wo sie aufgenommen wurden, dargestellt werden.\newline\newline
Daf�r sollten eine Reihe von Technologien genutzt und evaluiert werden. Dazu geh�ren die Frameworks Primefaces und JSF, die Dateiformate JPEG und KML, die Entwicklungsumgebung IntelliJ, sowie verschiedene Webservices zur Nutzung von GoogleMaps. 
\section{Anforderungsanalyse}
\label{JumpAnforderungsanalyse}
Basierend auf der Aufgabenstellung, sollen nun, die kongreten Anforderungen formuliert werden. Dabei wird unterschieden zwischen funktionalen und nicht-funktionalen Anforderungen und Anforderungen des Anwendungsbereiches.
\subsection{Funktionale Anforderungen}
Funktionale Anforderungen beschreiben Funktionen, die im fertigen System vorhanden sein sollten. Zus�tzlich kann unterschieden werden, ob eine Anforderung zu 100\% umgesetzt werden soll oder nur teilweise, bzw. nur wenn der zeitliche Rahmen es zul�sst.\\
Die Anforderungen an unser System wurden dabei wie folgt formuliert:
\begin{itemize}
	\item Einfache Nutzerverwaltung:
	\item[] Das System muss eine einfache Nutzerverwaltung implementieren. Dabei soll zwischen Gast, Uploader und Admin unterschieden werden, welche folgende Rechte besitzen:\\
	
	\begin{tabular}{|p{5cm}||c|c|c|}
		\hline
		 & \textbf{Gast} & \textbf{Uploader} & \textbf{Admin} \\
		 \hline
		 \hline
		 \textbf{�ffentliche Bilder ansehen} & Ja & Ja & Ja \\
		 \hline
		 \textbf{Private Bilder ansehen} & Nur mit Passwort & Ja & Ja \\
		 \hline
		 \textbf{Bilder verwalten} & Nein & Ja & Ja \\
		 \hline
		 \textbf{Benutzer l�schen} & Nein & Nein & Ja \\
		 \hline
	\end{tabular}

	\item Organisation der Fotos:
	\item[] Angemeldete Benutzer k�nnen Alben erstellen, zu denen sie mehrere Bilder hochladen k�nnen. Die Alben k�nnen auch nachtr�glich noch bearbeitet werden und Bilder hinzugef�gt oder gel�scht werden.
	\item Erweiterung der Bilddaten mit Informationen:
	\item[] Aus hochgeladenen Bildern sollen automatisch Metadaten, wie Zeit der Aufnahme und GPS-Informationen ausgelesen werden. Zus�tzlich kann der Nutzer zu jedem Bild Kommentare hinzuf�gen und auch die ausgelesenen Daten nachtr�glich bearbeiten oder hinzuf�gen, falls diese nicht ausgelesen werden konnten. Zus�tzlich kann der Nutzer Bilder �ffentlich machen, damit jeder sie betrachten kann.
	\item Anzeige von Bildern/Alben:
	\item[] Alle sichtbaren Bilder zu einem Album k�nnen direkt als Slideshow oder alternativ auf einer Google-Maps Karte mit integrierten Thumbnails angezeigt werden.
	\item L�schen von Alben:
	\item[] Der Admin des Fotoarchivs, kann Alben l�schen, wenn die Inhalte gegen die Nutzungsbedingungen versto�en.
	\item L�schen von Nutzern:
	\item[] Der Admin kann ebenfalls ganze Nutzer l�schen.
\end{itemize}
\subsection{Optionale Anforderungen}
\begin{itemize}
	\item Umkreissuche von Bildern:
	\item[] Wenn sich ein Betrachter f�r eine Region besonders interessiert, so kann er sich alle sichtbaren Bilder im Umkreis eines angegebenen Punktes auf der Karte anzeigen lassen, auch �ber mehrere Alben und Nutzer.
	\item Routenplaner von Bild zu Bild:
	\item[] Bei einer lange Distanz zwischen 2 Bildern, sollen Cloud-Services von Google genutze werden, um sich die optimale Route zwischen diesen Bildern berechnen zu lassen.
	\item Entfernung zwischen 2 Fotos:
	\item[] Auch hierf�r k�nnen Cloud-Services genutzt werden um entweder die Luftlinienentfernung, oder die L�nge der optmialen Route zwischen 2 Fotos anzugeben.
\end{itemize}
\subsection{Nicht-Funktionale Anforderungen}
Zu den nicht-funktionalen Anforderungen z�hlen alle Anforderungen, die die grundlegende Funktionalit�t des Systems nicht beeinflussen.\\
F�r unser System sind das die folgenden:
\begin{itemize}
	\item Datensicherheit:
	\item[] Die vom Nutzer eingegebenen Daten m�ssen sicher gespeichert werden, so dass diese nicht von dritten ausgelesen werden k�nnen. Das betrifft vor allem, die vom Nutzer hochgeladenen Bilder, welche als Privat gekennzeichnet wurden.
	\item Passwortsicherheit:
	\item[] Das Passwort eines Nutzers soll bereits verschl�sselt �bertragen werden, damit auch beim Abh�ren des Datenverkehrs, dieses nicht rekonstruiert werden kann.
	\item Usability:
	\item[] Das System soll �ber eine ansprechende Oberfl�che leicht zu bedienen sein. Der Nutzer muss zu jedem Zeitpunkt alle ihm m�glichen Aktionen schnell und problemlos finden k�nnen.
	\item Mehrbenutzerf�higkeit:
	\item[] Auf einem System k�nnen sich mehrere Nutzer anmelden und gleichzeitig ihre Bilder verwalten ohne, dass Probleme durch Nebenl�ufige Prozesse auftreten.
\end{itemize}
%\subsection{Anforderungen des Anwendungsbereiches}
%Anforderungen des Anwendungsbereiches charakterisieren Spezialit�ten des %Anwendungsbereiches
%??? Linux, wenig Arbeitsspeicher und kleine Festplatte, Nutzerverwaltung %DB-basiert
	\chapter{Evaluation eines bestehenden Systems - Drupal}
\label{JumpDrupal}
\subsection{Allgemeines}
Viele Menschen wollten schon immer ihre eigene Website, ihren Blog, ihr Forum oder �hnliches haben, doch sind immer an den HTML- und Programmierkenntnissen gescheitert. Das ist seit dem Beginn von Content Management Systemen (CMS) vorbei. Solche Systeme erm�glichen es in den meisten F�llen ohne Programmier- und HTML-Kenntnisse eine Website im Hand umdrehen zu erstellen. Bei einigen Webspaces oder Online-Hostern sind solche Systeme schon bereits vorinstalliert und k�nnen schnell und einfach genutzt werden oder sie sind daf�r geeignet und man muss sein CMS nur dort hochladen. Zu den bekanntesten Content Management Systemen zur Zeit z�hlen Typo3, Joomla und mittlerweile auch Drupal.\\
\\
Drupal ist ein CMS und Framework, welches urspr�nglich vom belgischen Informatiker Dries Buytaert entwickelt wurde. Es handelt sich dabei um eine freie Software, die unter der General Public License (GNU) steht. \\
\\
Wer Drupal nutzen m�chte ben�tigt daf�r lediglich einen Webserver, auf dem sich PHP und eine SQL-f�hige Datenbank befinden. Das sind alles Dinge, die die meisten Webspaces kostenlos oder f�r einen kleinen Preis zur Verf�gung stellen. PHP ist eine Skriptsprache, die haupts�chlich zur Erstellung dynamischer Webseiten oder Webanwendungen verwendet wird. Sie wird deshalb ben�tigt, da Drupal in PHP programmiert wurde.\\
\\
\subsection{Aufbau und Funktionalit�t}
Der Aufbau von Drupal ist sehr einfach und l�sst sich wie folgt beschreiben. Drupal besteht aus zwei Teilen, einem Core(Kern) und Modulen. Der Core beinhaltet die Grundfunktionalit�t, welche mit weiteren Modulen erweitert werden kann. Zur Grundfunktionalit�t z�hlen Komponenten wie Template-Erstellung, Blogsystem, Benutzerverwaltung und Taxonomie. \\
Diese Core-Funktionen reichen aus um eine simple Website zu erstellen. Man kann mit Hilfe der Template-Erstellung festlegen in welchen Bereichen der Website welcher Content angezeigt werden soll und auch die farbliche Gestaltung der gesamten Seite definieren. Die Benutzerverwaltung von Drupal ist sehr ausf�hrlich und detailliert. Sie zieht sich �ber mehrere Seiten und man kann dort Benutzergruppen definieren und ihnen Rechte zuweisen. Die Rechte k�nnen von Artikelarten, die ein Benutzer schreiben darf, �ber die Bilderanzahl pro Artikel, bis hin zu Kommentarfunktion und vielen mehr gehen. Mit Hilfe von Taxonomie lassen sich Men�s erstellen und verschiedene Artikel bzw. Beitr�ge der Website einer Kategorie zuweisen. Das kann �ber vorkommende Begriffe in Beitr�gen oder �ber eine Auswahl der Kategorie beim Schreiben des Artikels passieren. \\
\\
Der andere Teil neben dem Core sind die Module. Die sind daf�r da um je nach Anforderung Funktionen nachzur�sten. Diese Module werden meist von anderen Nutzer geschrieben und stehen dann allen zur Verf�gung und k�nnen per Download integriert werden.
Was dabei zu beachten ist, ist die Menge der Module. Es gibt eine sehr gro�e Anzahl und man kann nicht immer gleich am Modulnamen erkennen, was dieses Modul kann bzw. welche Funktion es liefern soll. Zudem wurde in mehreren Test festgestellt, das viele Module noch nicht richtig funktionieren und sich oft noch im Beta-Status befinden. Das liegt daran, dass f�r viele Module noch weitere Module ben�tigt werden, die zwar vorhanden sind, aber leider nicht mit anderen vorhandenen zusammen arbeiten, oder teilweise wiederum fehlerhaft sind. So kann es sein das man nach langen Suchen ein passendes Modul finden aus der Vielzahl der Ma�e, dieses aber dann nicht mit den weiteren ben�tigten Modulen funktioniert. Das kommt daher, dass jeder sein Modul so schreibt, dass es f�r seine Zwecke funktioniert und nicht auf die Konflikte mit anderen Modulen achtet.\\
\\
\subsection{Drupal als Grundlage f�r ein Foto-Archiv}
Drupal bietet viele sch�ne Eigenschaften und Funktionen, wie ein rollenbasiertes Rechtesystem und eine Versionierung der Inhalte. Zudem verf�gt es �ber Permalinks, d.h. jede Seite ist �ber eine feste sich nicht �ndernde URL erreichbar. Diese URLs sind ebenfalls Clean URLs, so dass sie von jeder Suchmaschine gelesen werden k�nnen, da sie menschen- und maschinenlesbar sind. Auch die gegebene M�glichkeit, Bilder auf die Webseiten hochzuladen ist sehr sch�n.\\
\\
Trotzdem erf�llt Drupal nur eine geringe Anzahl der gestellten Anforderungen, die man an ein Foto-Archiv stellt. Es ist zwar gelungen eine Diashow  der hochgeladenen Bilder per Zusatzmodul nachzur�sten, doch fehlt eine M�glichkeit Bilder nur f�r einzelne Nutzer zug�nglich zu machen, da eine Einschr�nkung der Sichtbarkeit nur �ber das Rechtesystem m�glich ist. Das kann umgesetzt werden, indem man eine neue Nutzergruppe erstellt und dort die entsprechenden Rechte vergibt, zum Beispiel welche Seiten sichtbar sein sollen. Das w�rde aber bedeuten, dass man f�r jeden Nutzer, der ein Archiv betreiben m�chte und seine Bilder nur bestimmten Personen zur Verf�gung stellen will, jeweils eine neue Benutzergruppe erstellen und zuweisen m�sste. Zudem gibt es leider keine M�glichkeit einzelne Bilder nicht f�r andere freizugeben, sondern nur komplette Seiten k�nnen angezeigt bzw. gesperrt werden. Auch die Verwaltung von Geodaten f�r die einzelnen Bilder ist nur mit Zusatzmodulen m�glich, was in einem Test aber nicht erfolgreich getestet werden konnte. Es war nicht m�glich Bilder auf einer Map anzuzeigen mit entsprechenden Geodaten. Das kann aber auch daran liegen, dass kein entsprechendes Modul in der Ma�e gefunden wurde.\\
\\
So muss man als Gesamtfazit sagen, dass Drupal eine echte Alternative in Sachen CMS ist und somit zur schnellen und einfachen Erstellung einer Website oder eines WebBlogs, aber im Hinblick auf die Erstellung eines Foto-Archivs noch als eher ungeeignet geeignet erscheint. Da viele Funktionen die man in einem Foto-Archiv erwartet, wie eine Diashow oder eine Album�bersicht, �bersicht der Bilder nur schwer bis gar nicht realisieren lassen. Zudem ist es zur Zeit schwer mit Drupal eine Darstellung der Bilder auf einer Karte mittels der Geodaten zur realisieren.

	\chapter{Analysierte Technologien}
\label{JumpTechnologien}
\section{Metadatenextraktion}
\label{JumpMetadatenextraktion}
\subsection{Kameraausgabeformate}
\label{JumpKameraausgabeformate}
Laut den Anforderungen, soll unser System in der Lage sein, aus den hochgeladenen Bildern, bereits wichtige Metadaten, wie die Zeit und den Ort, an dem das Foto entstanden ist, auszulesen. Dazu ist es zun�chst notwendig, sich mit den einzelnen Formaten und deren Aufbau n�her zu besch�ftigen.
%Vor allem von Kameras verwendete Formate --> JPG,...
%Alle aktuellen Kameras speichern zumindest Zeit, viele bereits schon GPS informationen
%\subsection{Exif}
%\label{JumpExif}
\section{Exif-Metadatenextraktion}
\label{JumpMetadatenextraktion}
F�r die Darstellung der Bilder in einer GoogleMaps-Karte sind nat�rlich die GPS-Koordinaten zwingend erforderlich. Wollte man fr�her GPS-Koordinaten mit einem Bild in Verbindung bringen musste man am Ort der Aufnahme zus�tzlich noch ein GPS-Empf�nger verf�gbar haben und sich die Koordinaten notieren. Inzwischen gibt es aber Kameras, die die Koordinaten als Metadaten zu einem Bild mit abspeichern. Die in diesem Projekt verwendete Kamera speichert die Bilder im JPEG-Format. Bei JPEG gibt es die M�glichkeit, Metadaten mittels einem Exif-Datenblock abzuspeichern. Auf diese Weise kann man auf die GPS-Daten zugreifen.\\
\\
Um nun mittels eines Programms die Daten zu extrahieren gibt es zwei M�glichkeiten. Die erste ist ein selbst entwickeltes Framework, mit denen man die Daten abgreift. Dabei muss eine ausf�hrlich Analyse des Bild-Formats durchgef�hrt werden. Aufgrund der knappen Zeit und unter der Beachtung der Aufgabenstellung, dass ein Fotoarchiv und nicht ein Framework zum extrahieren von Exif-Daten entwickelt werden sollte, haben wir die zweite M�glichkeit gew�hlt. Dies war die Verwendung eines bestehenden Frameworks. Um solche Frameworks zu finden bietet sich immer eine Internetrecherche an. Diese ergab, das mittels des Frameworks "`Metadata-Extractor"' die Anforderungen an ein solches Framework erf�llt wurden. Anhand des Quickstart-Guides ist zu erkennen, dass eine Verwendung obendrein relativ einfach ist.\\
\\
Das Framework bietet vorgefertigte Methoden, um die Metaddaten auszulesen. Dazu werden die Daten in einem Metadata-Objekt abgespeichert. Mittels eines Iterators kann man anschlie�end durch s�mtliche Metadaten-Tags iterieren.
\begin{lstlisting}[caption=Extrahieren von Exif-Daten aus einem Bild]
Metadata metadata = JpegMetadataReader.readMetadata(file);
Iterator directories = metadata.getDirectoryIterator();
while (directories.hasNext()) {
Directory directory = (Directory) directories.next();
Iterator tags = directory.getTagIterator();
...
}
\end{lstlisting}
\section{Webservices}
\label{JumpWebservices}
Eine Anforderung, die gestellt wurde, war die Untersuchung welche Webservices in Verbindung mit GoogleMaps angeboten werden. Eine Recherche ergab, dass folgende Webservices zur Verf�gung gestellt werden: Google Geocoding API, Google Direction API, Google Elevation API und Google Places API. Alle diese Webservices k�nnen mittels GET-Requests angesprochen werden. Au�erdem ist es noch m�glich �ber GoogleMaps Koordinaten abzufragen. Dies geschieht ebenfalls �ber ein GET-Request. S�mtliche Webservices liefern je nach Wunsch die Daten im JSON- oder im XML-Format. Bei der Abfrage der Koordinaten �ber GoogleMaps bekommt man bei dem Ausgabeformat XML die Daten in KML. Um die Koordinaten abzufragen reicht es aus einen Ortsnamen oder eine Adresse anzugeben. Im folgenden Listing ist eine Beispiel Anfrage an GoogleMaps zu sehen.
\begin{lstlisting}[language=XML,caption=Suchanfrage an GoogleMaps]
http://maps.google.de/maps/geo?q=goerlitz,%20Obermarkt,%2017&output=xml
\end{lstlisting}
Im Anhang auf Seite \pageref{lstGoogelMaps} im Listing \autoref{lstGoogelMaps} ist die Antwort auf diese Anfrage zu sehen.\\
\\
Eine Alternative zur direkten Anfrage bei GoogleMaps ist die Google Geocoding API. Mit dieser API kann man au�erdem noch zu einer gegebenen Koordinate den zugeh�rigen Ort herausfinden. Allerdings ist die Antwort, wenn XML ausgew�hlt wurde, hier kein KML sondern ein anderes Format mit dem Namen "`GeoCodeResponse"'.
\begin{lstlisting}[language=XML,caption=Suchanfrage Google Geocoding API]
http://maps.google.com/maps/api/geocode/xml?latlng=51.1552,14.986154&sensor=false
\end{lstlisting}
Eine ebenfalls sehr n�tzliche API ist die Google Direction API. Mit dieser kann man sich Routen von einem zu einem anderen Ort berechnen lassen. Dabei kann man unterscheiden ob die Route f�r ein Fahrzeug, f�r Fahrradfahrer oder f�r ein Fu�g�nger ist. Als Antwort erh�lt man ein DirectionResponse. Das besondere dabei ist, dass f�r jede Abzweigung die Koordinaten mitgeliefert werden, so dass in eine Karte die entsprechenden Linien eingezeichnet werden k�nnen.
\begin{lstlisting}[caption=Suchanfrage Google Direction API]
http://maps.google.com/maps/api/directions/xml?origin=51.149594,14.998664&destination=51.155187,14.986122&sensor=false
\end{lstlisting}
Die Google Elevation API liefert H�hendaten von einem Ort und die Google Places API liefert allgemeine Informationen. Bei beiden APIs gibt es von Google einige Beschr�nkungen. So muss f�r die Google Places API eine Maps-Client-ID erzeugt werden und die Ergebnisse der Elevation API d�rfen nur im Zusammenhang mit Google Karten verwendet werden.
\section{KML}
\label{JumpKML}
Die Keyhole Markup Language, kurz KML,  ist eine Auszeichnungssprache, die von Google.Inc entwickelt wurde um Geodaten f�r die Client-Komponenten der Programme Google Earth und Google Maps zu beschreiben. KML ist ein Standard des Open Geospatial Consortium und aktuell in der Version 2.2 verf�gbar.\\
\\
Es gibt zwei Formate in der eine KML-Datei vorliegen kann. Zum einen das .kml-Format und das .kmz-Format. Bei dem KMZ-Format handelt es sich um die gezippte Version einer oder mehrerer KML-Dateien. 
\\
�ber den Inhalt und Aufbau eines KML-Dokumentes sollte man folgendes wissen. Es kann Geodaten im Vektor- und Rasterformat enthalten, wobei Vektorobjekte wie Punkte, Linien, lineare Ringe, Polygone oder COLLADA-Modelle als Placemark-Elemente und Luft- und Satellitenbilder als GroundOverlay-Elemente modelliert werden. Placemark-Elemente k�nnen zus�tzlich zu den Geodaten auch noch andere Daten wie Name, Beschreibung, Betrachtungswinkel und -h�he, sowie einen Zeitstempel und noch vieles mehr enthalten. Hier ein Beispiel eines Placemark-Elementes eines KML-Dokuments:
\begin{lstlisting}[caption=KML-Dokument]{Name}
<?xml version="1.0" encoding="UTF-8"?>
<kml xmlns="http://www.opengis.net/kml/2.2">
<Document>
  <Placemark>
    <name>G�rlitz</name>
    <description>Hochschule Zittau/G�rlitz</description>
    <Point>
      <coordinates>51.149044,14.997729</coordinates>
    </Point>
  </Placemark>
</Document>
</kml>
\end{lstlisting}
Der Unterschied bei GroundOverlay-Elementen ist der, dass dort ein Koordinatenausschnitt zur Georeferenzierung der Rasterdaten angegeben werden muss, statt der Geometrie.\\
\\
Hat man eine solche KML-Datei erstellt bzw. vorliegen und m�chte dieses Ver�ffentlichen, so gibt es daf�r verschiedene Varianten. Um eine KML-Datei mit dem Programm Google Earth zu nutzen reicht es aus, dass sich die KML- oder KMZ-Datei irgendwo auf dem Rechner befindet, mit dem man die Applikation nutzt. Bei Google Maps ist die Sache schon etwas komplizierter. Dort gibt es nur die M�glichkeit, seine KML- bzw. KMZ-Datei auf einen Webserver oder den von Google daf�r vorgesehenen Server zu laden. Damit es dann aber auch mit Google Maps genutzt werden kann, muss auf der Seite "'http://www.google.de/addurl/?continue=/addurl"' ein Formular ausgef�llt werden, wo man die URL auf der sich das Dokument befindet angibt. Seit neuestem ist Google Maps auch in der Lage, das KML-Dokument zu nutzen wenn man einfach in das Google Maps Suchfeld die URL eingibt, wo sich die KML- bzw. KMZ-Datei befindet.\\
\\
KML-Dateien werden in diesem Projekt nicht genutzt werden, da sich die gleichen Dinge auch mit der Primeface Komponente GMap umsetzen lassen.
So braucht man nicht immer neue KML-Dateien erstellen bzw. �ltere �ndern, sobald ein neues Bild hochgeladen oder die Sichtbarkeit ge�ndert wird.

\section{YAML}
\label{JumpYaml}
In heutigen Webseiten verfolgt man oft die strikte Trennung von Gestaltung/Layout und Inhalt. Das Layout wird sehr oft durch die Cascading Style Sheets definiert. Um dies schnell und einfach zu erledigen gibt es ein Framework namens YAML. YAML steht f�r "'Yet Another Multicolumn Layout"' und ist ein (X)HTML/CSS Framework zur Erstellung von Layouts. Es liefert ein valides Grundger�st aus XHTML- und CSS-Code, die eine hohe Browserkompatibilit�t bieten, d.h. eine browser�bergreifende korrekte Darstellung garantieren. So muss sich nicht der Programmierer oder Designer um die verschiedenen Browser mit ihren Eigenschaften und Schw�chen besch�ftigen. Durch YAML werden bereits viele Browser-Bugs abgefangen und umgangen.



\section{JSF}
\label{JumpJSF}
Schon immer wird in der Programmierung versucht bestehendes wiederzuverwenden. Dieser Ansatz findet sich auch in der Programmierung von Web-Applikationen. F�r die Programmierung von Web-Applikationen ist so ein Framework JSF. Dieses Framework basiert auf Servlets und JSPs und bietet eine gute M�glichkeit komfortabel Webseiten zu entwickeln.\\
\\
Nat�rlich gibt es noch weitere Frameworks, so dass die Frage im Raum steht warum in diesem Projekt JSF als Basis-Framework verwendet wird. Zum einem w�re hier zu nennen, dass JSF schon eine Weile auf dem Markt ist, was f�r die Programmierung einen gro�en Vorteil darstellt. Es geht nichts �ber eine gute Dokumentation eines Frameworks. Au�erdem gibt es im Internet fast immer eine L�sung f�r ein Problem, da fast immer jemand anderes vorher das Problem auch hatte. Durch diesen Vorteil konnte die Entwicklung beschleunigt werden.\\
\\
Ein weiterer Grund f�r die Verwendung von JSF ist die Anforderung des Auftraggebers. Dieser hat vorgeschlagen das Projekt mit JSF zu verwirklichen. Da dies aus oben genannten Gr�nden ein guter Vorschlag war, wurde JSF verwendet.\\
\\
In diesem Projekt wird die Version 2.0.1 von Mojarra verwendet. Sie bietet eine gute Unterst�tzung f�r Facelets und ist im Moment die aktuelle Version von JSF. Die Version von Mojarra bietet die gleichen Funktionalit�ten wie die Referenzimplementierung von Apache MyFaces. Der Grund f�r die Verwendung von Mojarra ist darin begr�ndet, dass die IDE bei der Entwicklung Mojarra als einzige M�glichkeit f�r JSF 2 vorgeschlagen hat. Da, wie bereits erw�hnt, keine Unterschiede zu anderen Implementierungen bestehen, haben wir diesen Vorschlag angenommen. Ein weiterer Grund liegt in der Verwendung von Primefaces. Wie in Abschnitt \ref{JumpPrimefacesTechnologien} beschrieben, verwenden wir Primefaces 2.2.1, diese ben�tigt unbedingt JSF 2 mit Facelets.
\section{Primefaces}
\label{JumpPrimefacesTechnologien}
In heutigen Webanwendungen werden h�ufig viele Technologien verwendet am eine ansprechende Benutzeroberfl�che zu realisieren. Da gewisse Anforderungen immer wieder auftreten, wie zum Beispiel ein Dateiupload, ist es w�nschenswert ein Framework zu verwenden. Wie in Anschnitt \ref{JumpJSF} erl�utert wird, ist das Basis-Framework f�r dieses Projekt JSF. Mit diesem k�nnen zwar grunds�tzlich alle Anforderungen erf�llt werden (u.a. Dateiupload), doch es ist immer eine Erleichterung, wenn es daf�r schon vorgefertigte Komponenten gibt. Es wird daher Primefaces verwendet. Damit k�nnen alle Anforderungen schnell umgesetzt werden.\\
\\
Au�erdem soll nat�rlich die Benutzeroberfl�che immer ansprechend gestaltet sein. Bei Primefaces sind die Komponenten schon mit einem h�bschen Design vorhanden. Dieses vorgefertigte Design muss nicht unbedingt einen Nachteil bieten, und in diesem Projekt ist es auch kein Nachteil, da dadurch ein aufwendiger Designentwurf f�r diese Komponenten entfallen konnte. Es ist sogar m�glich dieses Design ein wenig zu beeinflussen.\\
\\
Wie jeder wei� gibt es nat�rlich eine ganze Reihe solcher Frameworks, die einen das Leben erleichtern. An dieser Stelle w�ren unter anderem RichFaces und IceFaces zu nennen. Aber unsere Wahl ist trotzdem auf Primefaces gefallen, weil es zum Einem eine Anforderung war Primefaces zu nutzen und zum Anderem bietet Primefaces n�tzliche Funktionen f�r dieses Projekt. So bietet Primefaces bereits eine Komponente f�r die Einbindung einer GoogleMaps-Karte. Mit dieser Komponente ist es au�erdem m�glich, Bilder in eine GoogleMaps-Karte anzuzeigen. Die Kernanforderung, das Anzeigen von Bildern in einer GoogleMaps-Karte, kann also mit Primefaces umgesetzt werden und erspart eine aufwendige Neuentwicklung.\\
\\
Primefaces wird in diesem Projekt in der Version 2.2.1 verwendet. Dies ist die aktuelle stabile Version. Im Moment befindet sich die Version 3 in Entwicklung und es gibt auch schon Vorabversionen davon. Es wurde trotzdem die Version 2.2.1 verwendet, da einige Projektmitglieder bereits schlechte Erfahrungen mit Vorabversionen in anderen Projekten gemacht haben.
\section{eXist}
\label{JumpeXist}
%eXist ist einfach und unkompliziert --> Daten hochladen, Ausf�hren von Abfragen --> keine Einbindung zus�tzlicher Frameworks f�r DB-Zugriff!!!
%Schnell eingerichtet, einfaches Prinzip.
%Schnelle Abfragen.
%Relationale DB-Systeme bestechen durch referentielle Integrit�t, gut durchdachtes Konzept, Jahrelanger Praxiseinsatz. Doch aufw�nfige Transformationen machen die VErwendung unhandlich, oder nur bestimmtes Format der XML-Dateien zul�ssig f�r universelle TRansformation, physische Struktur der XML-Dokumente wird zerst�rt (Kommentare auch, aber das macht uns nichts aus) --> physische Struktur schon wichtig, da z.B. Reihenfolge der Bilder in DB wichtig --> bereits sortiert, nach Zeitpunkt des Eintragens, Alben auch
%Ein zentraler Bestandteil unseres Fotoarchivs ist die Datenbank. Hier werden alle wichtigen Informationen zu Nutzern, Bildern und Alben gespeichert.
%Bei der Auswahl einer passenden Datenbank zur Verwaltung von XML-Dateien, gibt es zun�chst, wie unter \url{http://dbs.uni-leipzig.de/files/projekte/XML/paper/XMLDB\_IAOforum.pdf} nachzulesen ist, 2 gro�e Kategorien. Zum Einen gibt es klassische relationale Datenbanksysteme, die mit einer XML-Erweiterung ausgestattet sind. Das hei�t, sie sind in der Lage, XML-Dokumente in ihre interne Datenstruktur und umgekehrt zu transformieren. Da diese Systeme nicht vorrangig f�r die Speicherung von XML-Dokumenten konzipiert wurden, ergeben sich aus ihrem Einsatz verschiedene Nachteile: (Siehe: \url{http://wwwlgis.informatik.uni-kl.de/archiv/wwwdvs.informatik.uni-kl.de/courses/seminar/WS0203/folien5.pdf} und \url{http://www.markwiesemann.eu/download/proseminar-folien.pdf})
%\begin{itemize}
%	\item Die XML-Dokumente m�ssen ein bestimmtes Format besitzen
%	\item Verlust von physikalischer Struktur und Kommentaren
%	\item aufw�ndige Transformation der Daten
%\end{itemize}
%Das hei�t, relationale Datenbanksysteme sind f�r unseren Anwendungsfall eher ungeeignet.
%\begin{enumerate}
%	\item Native XML-Datenbanken
%	\item XML-enabled Datenbanken
%\end{enumerate}
%Bei XML-enabled Datenbanken handelt es sich um klassische realationale Datenbanksysteme, die ein Mapping der Daten ins XML-Format erlauben. Man spricht auch von einem datenorientiertem Ansatz.
Ein zentraler Bestandteil unseres Fotoarchivs ist die Datenbank. Hier werden alle wichtigen Informationen zu Nutzern, Bildern und Alben gespeichert.
Da wir XML-Dokumente verarbeiten, liegt es nahe, eine native XML-Datenbank zum Speichern der Daten zu verwenden. Dabei haben wir uns f�r eXist entschieden, was wir im Folgenden begr�nden wollen.
\subsection{Warum eXist?}
Und warum kein relationales Datenbank-Managementsystem?
Relationale Datenbanken basieren auf einem gut durchdachtem Konzept, das sich jahrelang bew�hrt hat und bieten wichtige Vorz�ge, die man bei einer nativen XML-Datenbank wie eXist nicht vorfindet:
\begin{enumerate}
	\item Referentielle Integrit�t:
	\item[] In eXist kann dieser Punkt leider nicht garantiert werden. Hier muss von Hand sichergestellt werden, dass alle Referenzen auf ein gel�schtes Objekt ebenfalls gel�scht werden. Da unser Datenmodell aber �berschaubar ist, stellt dieser Punkt kein gro�es Problem dar.
	%\item automatische Indizierung von Inhalten beherscht eXist --> Vorteil gegen�ber anderen nativen XML-DBs
	\item Sequenzen zur Vergabe eindeutiger Primary Keys:
	\item[] Gerade bei einem Multinutzersystem ist dieser Punkt klar von Vorteil. Bei der Verwendung von eXist, muss dieses Problem auf eine andere Art und Weise gel�st werden. N�heres dazu unter \autoref{JumpDatenbank}.
\end{enumerate}
Da relationale Datenbanken aber nicht prim�r f�r die Speicherung von XML-Dateien geeignet sind, k�nnen sich eine Reihe von Problemen ergeben. Zum Einen k�nnen aufw�ndige Transformationen notwendig sein, um XML-Dokumente in die internen Strukturen zu �bersetzen. Mit der Angabe des passenden Schemas, l�sst sich dieser Prozess zwar beschleunigen, was aber dazu f�hrt, dass nur noch Dateien mit einer bestimmten Struktur akzeptiert werden...

Weitere wichtige Eigenschaften relationaler Datenbanksysteme beherrscht eXist jedoch, unter anderem:
\begin{enumerate}
	\item Effiziente und strukturierte Speicherung:
	\item[] Wie viele native XML-Datenbanken, verwendet eXist einen modellbasierten Ansatz zur Speicherung der Daten, was eine schnellere und effizientere Suche, als bei XML f�higen Datenbanken, die eine rein zeichenkettenbasierte Speicherung der Daten vornehmen, erm�glicht. 
	%\item automatische Indizierung von Inhalten beherscht eXist --> Vorteil gegen�ber anderen nativen XML-DBs
	\item Indizes:
	\item[] Deren Verwendung bewirkt ebenfalls schnellere Suchvorg�nge. Dabei erzeugt eXist bereits einige wichtige Indizes,welche in der Regel auch ausreichen um vor allem XPath und XQuery zu beschleunigen. Es k�nnen aber noch weitere Indizes in den Konfigurationsdateien zu jeder Collection vom Nutzer definiert werden.
	\item Transaktionssicherheit:
	\item[] eXist ist in der Lage nach einem Absturz, alle vollst�ndig beendeten Transaktionen wiederherzustellen und alle nicht abgeschlossenen Transaktionen zur�ckzusetzen.
\end{enumerate}
\subsection{Alternativen}
Nicht nur eXist, sondern auch andere Systeme eignen sich f�r den Umgang mit XML-Dokumenten. Im folgenden m�chten wir deshalb Alternativen vorstellen und begr�nden, warum wir uns gegen diese entschieden haben.

	\chapter{Entwicklungsumgebung}
\label{JumpEntwicklungsumgebung}
\section{Tomcat}
\label{JumpTomcat}
Der von der Hochschule bereitgestellte virtuelle Server lie� eine freie Konfiguration und Installation von Software zu. Die Eckdaten des Systems lauten: 1 vCPU (entspr. 1 Intel XEON 2,266 GHz), 1 GByte RAM und 8 GByte Festplattenspeicher. Um die Webseite im Internet zu pr�sentieren, wird ein Servlet-Container ben�tigt auf der die Webanwendung gestartet wird Da in dieser Arbeit eine Reihe neuer Technologien bereits genutzt wurden, sollte die Wahl auf ein bekanntes und schon mal verwendetes Programm fallen. Zur Auswahl standen somit Apache Tomcat und GlassFish. Da GlassFish ein JavaEE-Server und dadurch ein vollst�ndiger Application-Server ist, wurde dieser nicht gew�hlt. Es war abzusehen, dass f�r das Projekt keine der zus�tzlichen M�glichkeiten die so ein System im Gegensatz zu einem reinen Servlet-Container bietet genutzt werden sollte. Somit fiel die Wahl auf Apache Tomcat. Da dieser f�r andere Projekte bereits verwendet wurde und ein gewisses Ma� an Expertise in Umgang und Konfiguration vorhanden war, war diese Entscheidung nicht schwer zu treffen. Bei Projektstart wurde Version 6 verwendet, da dieser bereits auf dem virtuellen Server vorinstalliert war. Im Laufe des Projektes fand eine Migration auf die aktuelle Version 7 statt. Der gro�e Vorteil gegen�ber der Alten ist, dass man damit in der Lage ist aus einer XHTML-Datei heraus Methoden eines Controllers aufzurufen und gleichzeitig Parameter zu �bergeben. Dadurch konnte ein Problem behoben werden, welches in Verbindung mit Primefaces auftrat, dass Parameter die �ber den Tag "`param"' aus der Taglibrary JSF-Core, gar nicht oder erst bei einem wiederholten Request, �bergeben wurde. Dies wird im Kapitel Implementation noch n�her beleuchtet.\newline\newline
Im Allgemeinen kann gesagt werden, dass die Entscheidung f�r Apache Tomcat sehr gut war. Es sollte jedoch darauf geachtet werden, dass nur absolut notwendige Dinge in den Logs ausgegeben werden, da diese eine beachtliche Gr��e annehmen k�nnen. So waren Logdateien von mehreren Gigabyte vorhanden, die die Festplattenausnutzung fast an 100\% brachten.

\section{Hudson}
\label{JumpHudson}
Ein weiterer Bestandteil der Entwicklungsumgebung war der Continuous Integration Server Hudson. Dieser wird �ber Tomcat gestartet und in Verbindung mit dem Versionsverwaltungssystem bietet es die M�glichkeit automatisiert die Anwendung nach einem Commit neu zu deployen und zu testen. Daraus ergeben sich Vorteile, wie z.B. einer stetig aktuellen und verf�gbaren Version der Anwendung, die automatisierte Ausf�hrung von Test �ber ein Ant-Skript, sowie sofortiges Feedback dar�ber, ob Integrations-Probleme bestehen. Das Ergebnis eines neuen Buildvorganges von Hudson l�sst sich per E-Mail an die Projektmitglieder versenden oder �ber einen RSS-Feed abrufen.\newline\newline
Die Entscheidung zu Gunsten von Hudson fiel, da in einem Vortrag �ber die Entwicklung eines Web-Projektes mit diesem System gute Erfahrungen gemacht wurden. �hnliches gilt auch f�r dieses, die Installation und Konfiguration erfordert etwas Zeit, wenn man dies zum ersten Mal macht. Allerdings �berwiegen die genannten Vorteile. Somit kann die Verwendung von Hudson f�r ein �hnliches Projekt wie in dieser Belegarbeit empfohlen werden.
\section{IntelliJ}
\label{JumpIntelliJ}
Dieses Projekt wurde mit der Entwicklungsumgebung IntelliJ IDEA der Firma JetBrains implementiert. Von der Umgebung werden Programmiersprachen wie Java, PHP und noch viele mehr unterst�tzt. Features wie Refactoring, Ant und JUnit und eine Versionskontrolle sind ebenfalls in Intellij zu finden.\\
\\
Was ist uns positiv an Intellij aufgefallen und was hat uns bei der Implementierung viel Zeit gekostet bzw. die Arbeit erschwert?\\
\\
Positiv aufgefallen sind bei der Entwicklung einer Applikation, die Codevervollst�ndigung, welche unter anderem auch Vorschl�ge bei der Variablendeklaration angeboten hat. Zudem war schnell ein �berblick zwischen lokalem Quellcode und der SVN-Version m�glich, da Differenzen automatisch im Quellcode am Rand angezeigt wurden, ohne extra einen Befehl aufzurufen. \\
\\
Zu den neutralen Dingen, die aufgefallen sind, z�hlt die Anordnung der Men�s rund um das Codefenster, siehe \autoref{fig:intellij_menu} im Anhang. Das erschwerte die Arbeit am Beginn der Entwicklung, �nderte sich aber mit der Zeit zu einem Vorteil, als man wusste, wo sich was befindet. Auch der Tooltip bei der Parameter�bergabe einer Methode, war vor der Version 10.5 nicht immer verf�gbar, funktionierte aber seit der Version 10.5 stabil und sorgte f�r eine gute �bersicht beim Aufruf einer Methode.\\
\\
Zu den negativen Punkten von Intellij geh�rt die inkonsequente Umsetzung der Hotkeys. Es wird angeboten, diese auf andere Entwicklungsumgebungen anzupassen, aber das leider nicht komplett. Auch die Gl�hbirne welche ab und zu mal erschien. war sehr speziell. Sie war die einzige M�glichkeit, schnell UnitTest zu erstellen, aber leider verschwand sie oft, bevor man mit dem Mauszeiger auf ihr war. Teilweise sorgten auch die sehr �berladenen Men�s f�r einiges Suchen, wenn eine kleine Einstellung ge�ndern werden sollte. Wie auch andere bekannte Entwicklungsumgebungen verf�gt auch diese �ber verschiedener Plugins. Die Anzahl der angebotenen Plugins ist aber eher klein, da es sich bei Intellij um eine kostenpflichtige Software handelt und kostenlose Umgebungen mit einem �hnlichen Umfang zu bekommen sind. In diesem Zusammenhang muss man auch sagen, dass die Auswahl der Plugins, bei der Installation getroffen werden musste, was zu einem sehr gro�en Speicherverbrauch bei der Ausf�hrung der Anwendung sorgte, da man nicht genau wusste, welche Plugins ben�tigt werden und man somit alle ausgew�hlt hat. Mit dem Release der Version 10.5 folgte zwar auch eine Community-Edition, welche aber nicht mehr betrachtet werden konnte. 
\section{GoogleCode}
\label{JumpGooglecode}
Ein weiterer wichtiger Bestandteil der Entwicklungsumgebung ist ein System, dass automatisch allen Teammitgliedern den aktuellen Entwicklungsstand zug�nglich macht. Anbieter von sogenannten Repositorys gibt es viele, so sind sowohl kostenlose als auch kostenpflichtige zu finden. Im Laufe des Studiums wurde bisher das System von ProjectLocker\footnote{URL: http://www.projectlocker.com/}, sowie ein Subversion-Server der Hochschule genutzt. Da das Projekt 4 Studenten bearbeiteten, konnte ProjectLocker nicht verwendet werden, da dieses in der kostenlosen Variante nur 3 Benutzer erlaubt. Eine kostenpflichtige Nutzung h�tte das Problem behoben, kam aber auf Grund von verf�gbaren, kostenlosen Alternativen nicht in Frage. Die Nutzung eines Hochschul-Servers wurde in der Gruppe besprochen, jedoch angesichts des Umzuges des Fachbereiches in ein anderes Geb�ude und der daraus resultierenden Ausfallzeit abgelehnt.\newline\newline
Ein Alternative zu den genannten Varianten ist das GoogleCode-Projekt. Voraussetzung zur Er�ffnung eines Projektes bei GoogleCode ist, dass man dieses unter einer OpenSource-Lizenz ver�ffentlicht. Anschlie�end kann man den vollen Funktionsumfang der Plattform nutzen, der Link zur Projektseite lautet http://code.google.com/p/geo-picture-service/. Zu den Funktionen z�hlen neben dem SVN-Repository, ein Wiki, welches zum Publizieren der aktuellen Fortschritte bei der Programmierung, sowie zum Ver�ffentlichen der gehaltenen Zwischenstandspr�sentationen in der Lehrveranstaltung diente. Au�erdem ist ein Issue-Tracker vorhanden. Dieser dient dazu, gefundene Fehler dem Entwickler direkt mitzuteilen. Dies erm�glicht eine schnelle und f�r andere nachvollziehbare M�glichkeit aufgetretene Probleme zu kommunizieren. Diese Art der Fehlerkorrektur wurde jedoch eher selten genutzt, da gr��tenteils die Entwickler sich in ein und dem selben Raum befanden. Eine direkte Kontaktierung erm�glichte eine schnellere und effektivere Behebung des Fehlers. Bei gr��eren Projekten und einer gr��eren r�umlichen Distanz zwischen den Programmierern, ist die Nutzung des Dienstes jedoch sehr zu empfehlen.\newline\newline
W�hrend der Bearbeitung des Projektes kam es zu keinerlei gr��eren Problemen mit dem Service. Lediglich das Einreichen von ge�nderten Code-Abschnitten konnte sehr selten nicht durchgef�hrt werden. Ein Grund daf�r konnte nicht gefunden werden. Internetrecherchen ergaben nur, dass dies sehr selten und unregelm��ig auftreten kann. Nach einer kurzen Wartezeit von wenigen Minuten funktionierte es wieder und der Quellcode konnte erfolgreich in das Repository aufgenommen werden. Die GoogleCode-Plattform ist zur Verwaltung von Programmen w�hrend der Entwicklungsphase und des Produktiveinsatzes absolut zu empfehlen.
	\chapter{Systementwurf}
\label{JumpSystementwurf}
Der Systementwurf ergibt sich direkt aus den Anforderungen. Dazu werden aus den Usecases direkt System-Sequenz-Diagramme abgeleitet. Diese sind im Anhang unter \autoref{JumpSSDs} zu finden.\\
Die Klassendiagramme f�r das System sind ebenfalls im Anhang unter \autoref{JumpKlassendiagramme} zu finden.

	\chapter{Implementation}
\label{JumpImplementation}
\section{Datenbank}
\label{JumpDatenbank}
\subsection{Aufbau der Datenbank}
Bei insgesamt nur 3 Datenhaltungs-Klassen, fiel auch der Aufbau der Datenbank entsprechend einfach aus. Zun�chst wurde f�r jede Klasse eine neue Collection angelegt, was etwa einer Tabelle in einer relationalen Datenbank oder einem Ordner in einem Dateisystem entspricht. In diesen Collections k�nnen nun ein, oder mehrere XML-Dateien gespeichert werden. Wie in \autoref{JumpDBOneBIGManysmall} begr�ndet wurde, wird hier nur jeweils eine Datei gespeichert.
Zus�tzlich wurde noch eine Collection angelegt f�r s�mtliche Stored XQueries. In dieser sind mehrere .xql-Dateien hinterlegt, f�r alle Abfragen, die durchgef�hrt werden m�ssen:
\begin{itemize}
	\item exist/rest/db/
	\begin{itemize}
		\item nutzer/nutzers.xml
		\item alben/alben.xml
		\item bilder/bilder.xml
		\item queries/
		\begin{itemize}
			\item allNutzer.xql
			\item albenForNutzer.xql
			\item[] ...
			\item bilderForAlbum.xql
		\end{itemize}
	\end{itemize}
\end{itemize}
F�r den Aufbau der einzelnen XML-Dateien sind im Anhang Beispiele unter \autoref{JumpListingNutzerXML}, \autoref{JumpListingAlbumXML} und \autoref{JumpListingBildXML} zu finden.
\subsection{Mockklassen}
Da die Datenbank noch nicht zu Beginn der Implementierungsphase komplett einsatzf�hig war, musste eine Zwischenl�sung her, damit bereits einfache Funktionalit�ten getestet werden konnten. Die Idee war es Mockklassen zu verwenden. Dazu wurden zun�chst Interfaces f�r alle sogenannten Connector-Klassen angelegt\footnote{Eine �bersicht �ber alle Connector-Klassen befindet sich unter \autoref{JumpDBKlassenDiag}}. Die Mockklassen zu den einzelnen Interfaces implementieren bereits einen gro�en Teil dieser Funktionen, ohne dabei im Hintergrund mit der Datenbank zu kommunizieren. Demnach bieten die Mockklassen auch nur eine eingeschr�nkte Funktionalit�t, wie die R�ckgabe eines Benutzers zu einer ID oder die Suche nach bestimmten Alben, basierend auf einer Menge von Testdaten. Eintragen, L�schen oder Modifizieren dieser Daten ist mit den Mockklassen dabei nicht m�glich.\\
Wird nun eine solche Klasse erzeugt, werden die Testdaten automatisch generiert und k�nnen genutzt werden. Sp�ter kann die jeweilige Mockklasse einfach mit der 'echten' Implementierung ausgetauscht werden. 
Ein Beispiel einer solchen Mockklasse, in dem Fall der MockAlbumConnector, ist im Anhang unter \autoref{JumpListingMockAlbumConnector} angegeben. Der Einfachheit halber werden nur die wichtigsten Methoden dargestellt.
\subsection{Datensicherheit}
\label{JumpDatensicherheit}
%\begin{figwindow}[1, r, \includegraphics[width=0.35\textwidth]{img/Zugriffrechteverwaltung-eXist.png}, {Sperrung von Zugriffsrechten in eXist %\label{JumpFigureZugriffsrechteeXist}}]
Die Datensicherheit zu gew�hrleisten funkioniert unter eXist sehr einfach. Durch die einfache Zugriffsrechteverwaltung f�r Collections und einzelnen Dateien, k�nnen diese f�r �ffentliche Nutzer schnell und einfach gesperrt werden. Nun muss aber noch der Applikation vor einer Anfrage mitgeteilt werden, wie sich diese zu authentifizieren hat. Sonst w�rde diese sich als Gast anmelden und somit keine Zugriffsrechte besitzen. Das kann einfach umgesetzt werden, indem die Klasse java.net.Authenticator erweitert wird, dargestellt in \autoref{JumpListingAuthenticator}.
%\end{figwindow}
Vor einer Anfrage muss nun einfach der Authenticator gesetzt werden:
\begin{lstlisting}
Authenticator.setDefault(new ExistAuthentificator());
\end{lstlisting}
Damit kann das System problemlos auf die Datenbank zugreifen, w�hrend nicht autorisierte Nutzer dazu nicht in der Lage sind.
\subsection{Stored XQueries}
\label{JumpXQueries}
Den Gro�teil der Funktionalit�t, zum Schreiben, Auslesen, �ndern und L�schen von Daten �bernimmt die Datenbank. Dazu werden Stored XQueries verwendet, die einfach als .xql-Dateien in der Datenbank gespeichert werden und per HTTP-GET aufgerufen werden m�ssen. Zur�ckgegeben werden dann entweder die gew�nschten Elemente, eventuell aufgetretene Fehler oder auch nichts, bei einer Update- oder Delete-Methode, die erfolgreich gewesen ist. Der Aufbau einer Datei ist dabei immer recht �hnlich. Im Folgenden soll dieser Anhand der Funktion updateBild gezeigt werden.\\
Zun�chst m�ssen alle Parameter, die �ber die URL mitgegeben werden, ausgelesen werden. Das erledigt folgende Zeile f�r einen Parameter:
\begin{lstlisting}[language=XML,caption=Auslesen von Parametern aus der URL,label=JumpXQueryParamsRead]
let $id:= request:get-parameter("id",0)
\end{lstlisting}
\$id ist dabei eine Variable, der der entsprechende Wert zugewiesen werden soll. Das erfolgt �ber den Methodenaufruf request:get-parameter. Diese Methode erwartet als ersten Eingabeparameter den Namen des URL-Parameters und als zweiten Parameter, einen Alternativwert f�r die Variable. Das ist bei Update-Methoden besonders sinnvoll. Hier ist nicht gefordert, dass alle Parameter �bergeben werden, doch im Anschluss wird das komplette Element, also das komplette Bild �berschrieben. Also muss als Alternativwert immer der entsprechende Wert des vorhandenen Elementes verwendet werden. Dazu wird nat�rlich die ID ben�tigt:
\begin{lstlisting}[language=XML,caption=Sinnvolle Festlegung des Alternativwertes,label=JumpXQueryParamsAlternativ]
let $name:= request:get-parameter("name",$bilder//bild[id=$id]//name/text())
\end{lstlisting}
Im Anschluss wird das neue Element, welches das alte �berschreiben soll erzeugt und einer Variable zugewiesen. Wichtig ist dabei, dass das Album eines Bildes dabei nicht ge�ndert werden kann:
\begin{lstlisting}[language=XML,caption=Erzeugen eines neuen Elements,label=JumpXQueryCreateElem]
let $new-bild:=
<bild>
  <id>{$id}</id>
  <name>{$name}</name>
  ...
  {$bilder//bild[id=$id]//album}
  <location>
    ...
  </location>
</bild>
\end{lstlisting}
Wie man sehen konnte spielt aber die ID dabei eine wichtige Rolle. Wurde diese nicht �bergeben, so darf nat�rlich nicht einfach das Bild mit der ID 0 �berschrieben werden. Die Pr�fung, ob ein bestimmter Parameter �bergeben wurde, erfolgt dabei innerhalb des return Statements um in diesem Fall, eine passende Fehlermeldung zur�ckgeben zu k�nnen:
\begin{lstlisting}[language=XML,caption=Pr�fen ob ein Parameter �bergeben wurde,label=JumpXQueryParamsGetCheck]
if (not($id))
 then (
   <error>
     <message>Es wurde keine ID angegeben.</message>
   </error>)
\end{lstlisting}
Zudem k�nnen hier noch weitere Pr�fungen vorgenommen werden, z.B. ob ein Bild mit der angegebenen ID �berhaupt existiert oder ob der neue Name des Bildes in dem Album schon vorhanden ist:
\begin{lstlisting}[language=XML,caption=Weitere Pr�fungen,label=JumpXQueryParamsCheckOther]
if(count($bilder//bild[id=$id])=0)
  then(
    <error>
      <message>Ein Bild mit der id {$id} existiert nicht!</message>
    </error>
  )
  else(
    if(count($bilder//bild[id!=$id][album=$bilder//bild[id=$id]//album][name=$name])>0)
      then(
        <error>
          <message>Ein Bild mit dem angegebenen Namen existiert bereits in diesem Album!</message>
        </error>
      )
\end{lstlisting}
Wenn alle Vorraussetzungen erf�llt sind, kann im Anschluss die eigentliche Funktionalit�t ausgef�hrt werden. Im Falle eines Updates wird dabei nichts zur�ckgegeben, sondern einfach die entsprechende Methode ausgef�hrt:
\begin{lstlisting}[language=XML,caption=Update eines Bildes,label=JumpXQueryFinalUpdateBild]
(update replace $bilder//bild[id=$id] with $new-bild)
\end{lstlisting}
Die Abfragen nach bestimmten Elementen bereiten dabei auch keine Schwierigkeiten. So k�nnen, wie im folgenden Beispiel, Methoden beliebig kombiniert werden, um komplexere Abfragen zu realisieren. Hier sollen alle Alben gefunden werden, die im Namen einen bestimmten Text enthalten. Dabei soll die Abfrage nicht Case-Sensitiv sein:
\begin{lstlisting}[language=XML,caption=Abfrage aller Alben mit bestimmtem Namen,label=JumpXQueryAlbenWithNameContaining]
let $collection := collection('/db/alben')//album[contains(upper-case(name), upper-case($name))]
\end{lstlisting}
\subsection{Zugriff auf die Datenbank}
Im vorherigen Abschnitt wurde gezeigt, dass Abfragen mit Stored XQueries sehr einfach implementiert und bereitgestellt werden k�nnen. Der Aufruf muss nun nur noch �ber ein einfaches HTTP-GET erfolgen. Diese Funktionalit�t wird im DBConnector in der Methode executeGetRequest implementiert. Der DBConnector kennt die Adresse der Datenbank und ben�tigt zum Ausf�hren nur noch den Pfad zur auszuf�hrenden .xql-Datei und die zu �bergebenden Parameter.\\
Diese Angaben werden von der aufrufenden Methode zusammengestellt. Dabei muss jede Methode den Pfad zur aufzurufenden Datei selber kennen. Au�erdem m�ssen die Parameter vorher in ein einheitliches Format gebracht werden. Dazu wird eine HashMap verwendet, da die Anzahl der Parameter variabel ist. In dieser Map werden dann Parametername als Schl�ssel und Parameterwert als Wert verwendet. Im Folgenden wird ein Beispielaufruf f�r die Erzeugung eines Nutzers angegeben:
\begin{lstlisting}[caption=�bergabe der Paramter an den DBConnector,label=JumpDBZugriffParameter]
Map<String,String> params = new HashMap<String,String>();
params.put("name", name);
params.put("password", password);
params.put("email", email);
...
Document doc = DBConnector.getInstance().executeGetRequest("queries/createAlbum.xql", params, 0);
\end{lstlisting}
Mit diesen Angaben wird dann die URL zusammengesetzt. Dabei werden auch die zu �bergebenden Parameter im UTF-8 Format codiert, damit auch Sonderzeichen oder Umlaute wie �, �, � fehlerfrei �bertragen werden:
\begin{lstlisting}[caption=Zusammensetzen der URL f�r ein HTTP-GET,label=JumpDBZugriffURL]
String paramString = "";
boolean firstSet = false;
for(Map.Entry<String,String> entry : params.entrySet()){
  if(entry.getValue() != null && !entry.getValue().isEmpty()){
    if(firstSet){
      paramString += "&" + entry.getKey() + "=" + URLEncoder.encode(entry.getValue(), "UTF-8");
    }else{
      paramString += "?" + entry.getKey() + "=" + URLEncoder.encode(entry.getValue(), "UTF-8");
      firstSet = true;
    }
  }
}
URL url = new URL(this.location + file + paramString);
\end{lstlisting}
Im Anschluss muss nur noch das zur�ckgegebene Dokument geparst werden:
\begin{lstlisting}[caption=Parsen des Documents,label=JumpDBZugriffDocument]
DocumentBuilderFactory dbf = DocumentBuilderFactory.newInstance();
DocumentBuilder db = dbf.newDocumentBuilder();
doc = db.parse(url.openStream());
\end{lstlisting}
Eventuelle dabei auftretende Exceptions werden gekapselt und nach oben geworfen. Je nach geworfener Exception entscheidet dann die aufrufende Methode, wie diese zu behandeln ist. In der Regel werden die Exceptions dabei gleich an die Controller weitergerreicht, damit dem Nutzer eine entsprechende Fehlermeldung ausgegeben werden kann. Nat�rlich kann auch ein Fehler, angegeben durch einen $<$error$>$ Tag im XML-Dokument, zur�ckgegeben werden. In diesem Fall wird in der Regel durch die aufrufende Methode ebenfalls eine Exception erzeugt, und zum aufrufenden Controller geworfen:
\begin{lstlisting}[caption=Auswerten des error-Tags,label=JumpDBZugriffErrorTag]
if(doc.getElementsByTagName("error").getLength()>0){
  throw new IllegalArgumentException(doc.getElementsByTagName("message").item(0).getTextContent());
}
\end{lstlisting}
Tritt kein Fehler auf, so kann in den meisten F�llen gleich das erhaltene Dokument zur�ckgegeben werden oder bei einem Insert, die aus dem Dokument ausgelesene ID.
\subsection{Verhinderung gleichzeitiger Zugriffe}
\label{JumpDBZugriffe}
Der letzte Punkt der noch sichergestellt werden musste, war die Verhinderung gleichzeitiger Zugriffe. Beim Eintragen neuer Elemente muss sichergestellt werden, dass nicht 2 Elemente die selbe ID erhalten. Da es in eXist keine Sequenzen gibt zum automatischen, synchronisierten Erzeugen neuer IDs, wird einfach die gr��te gefundene ID + 1 verwendet. Werden nun etwa 2 Bilder zur selben Zeit eingetragen, so kann es dadurch aber passieren, dass beide die selbe ID erhalten, da beide Anfragen von unterschiedlichen Threads abgearbeitet werden. Um das zu verhindern, wird im DBConnector die Anzahl der gleichzeitigen Requests auf 1 reduziert. Dazu wird das Singleton-Pattern verwendet (siehe \autoref{JumpDBZugriffSingleton}). Der Konstruktor des DBConnectors ist privat und innerhalb der Klasse wird dieser nur ein einziges mal aufgerufen um eine statische Instanz zu Erzeugen. Dadurch wird sichergestellt, dass alle DB-Anfragen �ber diese eine Instanz abgearbeitet werden. Zus�tzlich wird die Methode executeGetRequest synchronisiert um zu gew�hrleisten, dass nur eine Anfrage zur selben Zeit ausgef�hrt wird.

\section{Applikation}
\label{JumpApplikation}
%TODO anderen Namen vergeben?, Unterabschnitte anpassen
\subsection{Controller}
\subsubsection{UserController}
Die gesamte Benutzerverwaltung regelt der Usercontroller. Er stellt u.a. Methoden f�r die Kontoerstellung, die Anmeldung und das Zusenden eines neuen Passworts zur Verf�gung. Au�erdem wird in einer Variable die Information bereitgestellt, ob ein Benutzer angemeldet ist oder nicht. Dies ist f�r die Darstellung der Webseiten von gro�er Bedeutung. Da nicht alles f�r unangemeldete Benutzer angezeigt werden soll, werden "`private"' Bereiche mittels "`rendered"'-Attribut nur angezeigt, wenn der Benutzer angemeldet ist.\\
\\
Das Anlegen eines neuen Benutzkontos l�uft wie folgt ab. Der Benutzer w�hlt einen Benutzernamen und gibt seine EMail-Adresse an. An diese Adresse wird ein zuf�lllig generiertes Passwort gesendet. Auf diese Weise soll verifiziert werden, ob es die EMail-Adresse wirklich gibt. Sollte ein Benutzer n�mlich sein Passwort vergessen, wird ein neues Passwort an diese EMail-Adresse gesendet. F�r das zuf�llige Erzeugen des Passworts wird mittels einer Zufallszahl ein Buchstabe oder eine Zahl aus einer vorgegebenen Auswahl von Zeichen gewh�hlt und in einem String verkettet. Den Code der Methode findet man im Anhang auf Seite \pageref{lstRandomPasswort} im Listing \autoref{lstRandomPasswort}.\\
\\
Beim Umgang mit Benutzerdaten gibt es immer auch Anforderungen an die Sicherheit. So soll es vermieden werden, dass Passw�rter unverschl�sselt gespeichert werden. Unter Beachtung dieses Hinweises werden die Passw�rter deswegen verschl�ssselt gespeichert. Als Verschl�sselungsmethode wird der Hash-Algorithmus MD5 verwendet. Sobald ein Passwort von der View an den Controller �bergeben wird, erfolgt die Verschl�sselung bereits im Setter f�r die Passwortvariable. Auf diese Weise wird verhindert, dass irgendwo in der Anwendung ein unverschl�sseltes Passwort zur Verf�gung steht. Um nun festzustellen ob das vom Benutzer eingegebene Passwort das Richtige ist, wird das bei der Eingabe verschl�sselte Passwort mit dem Passwort aus der Datenbank verglichen. Sobald diese beiden Passw�rter �bereinstimmen erfolgt die Anmeldung. Die Methode, die das Verschl�sseln bewerkstelligt findet man wieder im Anhang im Listing \autoref{lstEncryptPasswort} auf Seite \pageref{lstEncryptPasswort}.\\
\\
Wie bereits weiter oben erw�hnt, werden EMails zur Verifizierung bzw. Zum Zur�cksetzen des Passworts verschickt. Daf�r wird die JavaMailAPI von Oracle verwendet. Aus dieser API werden allerdings nur die "`mailapi.jar"' und die "`smtp.jar"' verwendet, da die anderen jar-Dateien wie z.B "`imap.jar"' nicht ben�tigt werden. Die JavaMailAPI bietet Methoden an, damit der Inhalt der EMail und Verbindungseigenschaften zum Server gesetzt werden k�nnen. So kann z.B. eingestellt werden, dass die Verbindung zum Server �ber SSL hergestellt werden soll. Im Listing \autoref{lstMailversand} auf Seite \pageref{lstMailversand} im Anhang findet man die Methode, mit der die EMails versendet werden.

\subsubsection{CreateEditAlbumController}
\label{subsubsec:CreateEditAlbumController}
Hinter diesem eigent�mlichen Namen befindet sich der Controller zum Erstellen und Bearbeiten von Alben. Es liegt auf der Hand, dass sich eine Webseite zum Erstellen und zum Bearbeiten eines Albums nicht sehr unterscheiden. Daher bietet es sich an f�r beide Aufgaben eine Webseite mit einem Controller zu erstellen. Die wenigen Dinge die sich auf der Webseite unterscheiden werden wieder mit einem "`rendered"'-Attribut versehen. Im Controller gibt es eine Boolean-Variable "`isNewAlbum"' mit der festgestellt werden kann ob ein neues Album erstellt oder bearbeitet wird.\\
\\
Bei der Erstellung eines neuen Albums werden nat�rlich auch Bilder hinzugef�gt. Das Hochladen der Bilder erfolgt �ber eine Primafaces-Komponente die auf Seite \pageref{subsec:FileUpload} im Abschnitt \autoref{subsec:FileUpload} beschrieben wird. Diese Komponente braucht eine Methode, die das von der Komponente ausgel�ste FileUploadEvent entgegen nimmt und verarbeitet. Den Code der Methode findet man im Anhang auf Seite \pageref{lstHandleFileUpload} im Listing \autoref{lstHandleFileUpload}.\\
\\
Wir haben uns dazu entschieden das Bild im Dateisystem abzuspeichern und in der Datenbank lediglich den Pfad abzuspeichern. Der Grund daf�r ist, dass sonst bei der Speicherung in der Datenbank die Bilder erst aufwendig encodiert werden m�ssten. Sobald dann sp�ter die Bilder aus der Datenbank geholt werden, m�sste man au�erdem wieder die Daten zu einem JPEG-Bild konvertieren. Diesen Mehraufwand konnten wir durch die Speicherung im Dateisystem umgehen. Au�erdem gibt es mit unserem Verfahren weniger Probleme wenn mehrere Bilder auf einmal hochgeladen werden. Das Upload-Verzeichnis an sich liegt irgendwo im Dateisystem. Die einfachste M�glichkeit w�re zwar der web-Ordner der Applikation gewesen, aber dort w�re die Bilder bei jedem Neudeploy gel�scht worden. Aus diesem Grund gibt es ein extra Verzeichnis wo alle Bilder abgespeichert werden.\\
\\
%TODO getImage beschreiben.
\subsection{AlbumController}
Der AlbumController geh�rt zu der Seite showAlbum.xhtml und k�mmert sich darum die Pfade und Bildinformationen aus der Datenbank zum ausgew�hlten Album und von der Festplatte zu laden. Mit Hilfe der Methode getAlbumByName(String name) des IAlbumConnector bekommt man ein XML-Dokument geliefert, welches alle Informationen zu dem �ber den Parameter �bergebenen Albumname enth�lt. Unteranderem ist auch die AlbumID dabei, diese wird f�r die Abfrage der Bilder des Albums ben�tigt. Hierf�r wird die Methode getBilderForAlbum(int albumID) von IBildconnector verwendet. Wiederum erh�lt man ein XML-Dokument, dass iterativ nach den Bildern durchsucht werden kann. Anschlie�end werden sie einem Listen-Objekt vom Typ Bild hinzugef�gt. Durch eine Getter-Methode die in einem Galleria-Tag aufgerufen wird, werden diese der Galerie hinzugef�gt und angezeigt. Diese Variante funktioniert aber nur, wenn die Bilder statisch im Web-Ordner liegen. Um dynamische Bilder anzeigen zu k�nnen, wie hier ben�tigt, muss ein Umweg gegangen werden. So muss sichergestellt werden, dass es ein Objekt vom Typ StreamedContent gibt und per Getter erreichbar ist. Dieses muss bei jedem Wechsel des Bildes neu geladen werden. Dazu wird in der XHTML-Seite folgender Quellcode ben�tigt:
\begin{lstlisting}[caption=showAlbum.xml Galleria]
<h:outputLabel value="#{albumController.getImage(pics.name)}"/>
<p:graphicImage value="#{albumController.picture}"
title="#{pics.name}" alt="#{pics.beschreibung}">
</p:graphicImage>
\end{lstlisting}
Wenn das Outputlabel gerendert wird, dann erfolgt die Ausf�hrung der Methode albumController.getImage(pics.name). Dies ist erst seit Tomcat 7 m�glich. Im Controller wird das Bild mit dem �bergebenen Bildnamen geladen und in der Variable picture abgelegt. Diese wird von der Primefaces-Komponente graphicImage geladen und angezeigt. Allerdings funktioniert diese Variante unter Primefaces 2 nicht korrekt. So wird auf Grund eines Caching bereits beim Aufrufen der Seite alle Bilder durchiteriert und somit wird nur das letzte Bild angezeigt. Das Fehlverhalten soll mit der neuen Version 3 von Primefaces behoben werden. Dies konnte in dem bisher verf�gbaren Milestone jedoch nicht best�tigt werden. Daf�r soll mit der neuen Fassung die Komponente ImageSwitch die Funktionalit�t erhalten, ob dies der Fall ist, wurde nicht untersucht.
\subsection{MapBean}
F�r die Nutzung der Google-Maps-Komponente wird eine Bean ben�tigt. Diese hat die Aufgabe die Marker f�r die Karte zu erstellen, die Linie dazwischen einzuzeichnen und auf Anfrage das Bild zu einem Marker zu laden. Die ersten zwei m�ssen beim Erstellen der HTML-Seite durchgef�hrt werden, dazu wird die Funktion makeMarker() ben�tigt, welche im Listing \ref{lstMakeMarker} auf Seite \pageref{lstMakeMarker} gezeigt wird. Die Methode �berschreibt die MapModel-Variable, welche vom gmap-Tag benutzt wird (siehe dazu: Kapitel \ref{GMap} auf Seite \pageref{GMap}). Dem Model werden die Overlays hinzugef�gt, die dann auf der Karte angezeigt werden. F�r einen Marker wird ein Objekt vom Typ Marker aus dem Paket org.primefaces.model.map ben�tigt. Der Konstruktor fordert ein Objekt vom Typ LatLng, einen String der den Titel repr�sentiert und ein Object, dass beim triggern des Events ausgewertet werden kann. In diesem Fall wird es dazu verwendet, um den Pfad des Bildes zu �bergeben, damit es von der Festplatte gelesen und als Streamed Content zur�ckgegeben und angezeigt werden kann. Das Objekt LatLng verlangt im Konstruktor zwei Werte vom Typ double, sie repr�sentieren Latitude und Longitude, also die geografische Breite und L�nge. Um zwei Marker miteinander verbinden zu k�nnen und somit eine Route darzustellen, bekommt das Objekt Polyline die Koordinaten der Marker �bergeben. Nachdem alle enthalten sind, wird es dem MapModel hinzugef�gt. Das Aussehen der Linie l�sst sich mit Hilfe diverser Methode individuell anpassen, so wurde in diesem Fall die Breite, Farbe und Durchsichtigkeit angepasst. Die verwendeten Methodenaufrufe sind im Listing zur MapBean enthalten.
\subsection{Scopes}
\subsection{Beanzugriff}
\subsection{MessageProperties}
Bei der Entwicklung einer Webapplikation ist ein wichtiges Thema was man vorher planen sollte die Internationalisierung. Wei� man von Anfang an, dass es die Applikation nur in einer Sprache geben soll, kann man die Nachrichten bzw. Texte fest hinein schreiben. Wei� man das aber zu Beginn noch nicht genau, oder m�chte die Applikation in mehreren Sprachen ver�ffentlichen, so sollte man die JSF-Komponente message bundle verwenden. Das sorgt daf�r, dass die Zeichenketten der Nachrichten unabh�ngig von der Applikation gespeichert werden und schnell einfacher ausgetauscht werden k�nnen, oder schon in mehreren Sprachen dort vorliegen. Um zu entscheiden welche der Sprachversionen genutzt werden soll kann man sich des Lokalisierungscodes eine JSF-Anwendung bei der Ausf�hrung bedienen. Dieser Lokalisierungscode besteht aus zwei Teilen, dem Code der Sprache und die Festlegung eines Stattes in der die Applikation laufen soll. Welche Codes die Applikation unterst�tzt kann in der faces-config.xml festgelegt werden.
\begin{lstlisting}[caption=faces-config.xml Lokalisierungscode]{Name}
<locale-config>
            <default-locale>de</default-locale>
            <!--<supported-locale>de</supported-locale>-->
</locale-config>
\end{lstlisting}
Dort kann ein Standardwert festgelegt werden mit dem Tag "'<default-locale>"' und die weiteren Lokalisierungen, die unterst�tzt werden. In dieser Applikation wird lediglich die Lokalisierung f�r Deutschland unterst�tzt.
Des weiteren muss man festlegen, wo die Applikation die Nachrichten bzw. Texte finden kann. Das passiert ebenfalls in der faces-config.xml durch das "'<message-bundle>"' Tag.
\begin{lstlisting}[caption=faces-config.xml message bundle]
<message-bundle>bundle.message</message-bundle>
\end{lstlisting}
In diesem Fall das Package bundle und dort die Datei "'message.properties"'. \\
\\
Die Texte der Applikation werden dann wie folgt in die "'message.properties"' eingetragen:
\begin{lstlisting}[caption=message.properties]
home = Startseite
ungueltigerBenutzerName = Der Benutzername ist ung�ltig!
welcometext = Herzlich Willkommen auf GPicS!
\end{lstlisting}
Das sind nat�rlich nicht alle Nachrichten, sondern nur eine sehr kleine Auswahl um das Prinzip zu verdeutlichen.\\
\\
Um nun in einem XHTML-File auf die Nachrichten zuzugreifen, wurde in der faces-config.xml der Pfad sowie eine Variable festgelegt:
\begin{lstlisting}[caption=message.properties XHMTL-Variable]
<resource-bundle>
    <base-name>/bundle/message</base-name>
    <var>msg</var>
</resource-bundle>
\end{lstlisting}
Der Text wird dann wie folge aufgerufen:
\begin{lstlisting}[caption=message.properties Aufruf mit XHMTL-Variable]
<h:outputLabel value="#{msg.bildname} "/>
\end{lstlisting}
Mit diesem Aufruf wird folgende Zeile der "'message.properties"' aufgerufen:
\begin{lstlisting}[caption=message.properties XHMTL-Variable Teil 2]
bildname = Name
\end{lstlisting}
Somit wird das Label den Namen "'Name"' bekommen.\\
\\
Will man in einem Controller auf die "'message-properties"' zugreifen, geht das wie folgt.
Die jeweiligen Nachrichten werden dann durch folgende Methode an der jeweiligen Stelle der Applikation aus der "'message.properties"' geladen:
\begin{lstlisting}[caption=message.properties Aufruf aus Controller]
MessagePropertiesBean msgPB = new MessagePropertiesBean();
String pfad = msgPB.getPropertiesMessage("defaultPicturePath");
\end{lstlisting}

Daf�r wird die die Methode "'getPropertiesMessage()"'  der Klasse MessagesPropertiesBean aus dem Package util genutzt.

\begin{lstlisting}[caption=message.properties Aufruf aus Controller Teil 2]
public String getPropertiesMessage(String key) {

		FacesContext context = FacesContext.getCurrentInstance();

		String text = GPicSUtil.getMessageResourceString(context.getApplication()
                .getMessageBundle(), key, null, context.getViewRoot()
                .getLocale());

return text;
	}
\end{lstlisting}

Der Methode wird ein String �bergeben, der wiederum einer weiteren Methode der Klasse GPicSUtil �bergeben wird.

\begin{lstlisting}[caption=message.properties Aufruf aus Controller Teil 3]
public static String getMessageResourceString(
    String bundleName,
    String key,
    Object params[],
    Locale locale){

        String text = null;

        ResourceBundle bundle =
                ResourceBundle.getBundle(bundleName, locale,
                                        getCurrentClassLoader(params));

        try{
            text = bundle.getString(key);
        } catch(MissingResourceException e){
            text = "?? key " + key + " not found ??";
        }

        if(params != null){
            MessageFormat mf = new MessageFormat(text, locale);
            text = mf.format(params, new StringBuffer(), null).toString();
        }

    return text;
}
\end{lstlisting}
Diese Methode durchsucht dann das "'message-properties"'-File und gibt den Text zur�ck, soweit er gefunden wird.
\subsection{EMail}
F�r die Benutzerkontoerstellung und das Zusenden eines neuen Passworts haben wir in unserem Projekt die M�glichkeit vorgesehen EMails zu versenden. Da der EMail-Versand zwingend ein EMail-Account erfordert, haben wir einen Mailserver eingerichtet, womit diese EMails verschickt werden konnten. Allerdings haben wir relative schnell Probleme festgestellt. So funktionierte das Schicken der Mails nur an die Hochschul-EMail-Adresse. Bei anderen Mail-Anbietern kamen keine EMails an. Sie wurden noch nicht einmal im Spam Ordner angezeigt. Der Grund hierf�r liegt wahrscheinlich darin, dass die anderen Mail-Anbieter unsere Adresse nicht als vertrauensw�rdig einstuften.\\
\\
Dies war aber nicht das einzigste Problem. Probleme mit unserem VServer erzwangen ein Neustart des Mail-Servers. Danach hat das Senden von Mails nicht mehr funktioniert. Wo die Gr�nde liegen ist leider nicht ohne eine aufwendige Problemanalyse raus zu bekommen. In Anbetracht der knappen Zeit haben wir uns daf�r entschieden, bei GMX ein Mailkonto einzurichten und �ber diese Adresse die Mails zu schicken.\\
\\
Nun steht nat�rlich die Frage im Raum warum wir das nicht von Anfang an gemacht haben und uns mit dem Aufwand, dem Einrichten eines Mailservers, viel Arbeit aufgeladen haben. Der Grund daf�r waren bedenken, einen "`Sinnlos-Account"' einzurichten. Es werden ja nur einige wenige EMails versendet. Au�erdem macht eine eigene EMail-Adresse immer einen besseren Eindruck. W�re mehr Zeit gewesen h�tten wir sicher das Problem mit unserem Mailserver l�sen k�nnen, aber so mussten wir eben diese Alternative verwenden.
\section{Validatoren}
\label{JumpValidatoren}
Oft haben Applikationen mit fehlerhaften Eingaben zu kämpfen, die der Nutzer nicht einmal bewusst tätigt. Um daraus bedingten Fehlfunktionen und einem reibunglosen Ablauf zu garantieren, gibt es die Möglichkeit sogenannte Validatoren zu benutzen.
Sie sind dafür da, eine Eingabe auf eine bestimmte Art von Muster bzw. Standard zu überprüfen. Da das Ganze aber automatisch bei jeder Eingabe ablaufen soll, kann dabei nur auf syntaktische Korrektheit geprüft werden. \\

Um einen Validator in JSF nutzen zu können muss er in der faces-config.xml bekannt gemacht werden. Das sieht wie folgt aus.
\begin{lstlisting}[caption=Einbinden eines Validators ind die faces-config.xml]{Name}
<validator>
   <validator-id>EmailValidator</validator-id>
   <validator-class>de.hszigr.gpics.validation.EmailValidator</validator-class>
</validator>
\end{lstlisting}
Alle Validatoren, die genutzt werden sollen, müssen sich in dem Tag "validator" der faces-config befinden. Jeder Validator muss über eine eindeutige ID verfügen und der Klassenname muss auch hinterlegt werden. \\
Das allein reicht noch nicht aus damit der Validator auch aktiv wird. An den Stellen, wo die Eingabe validiert werden soll, muss er noch der jeweiligen Komponente zugewiesen werden. Das geschieht so:
\begin{lstlisting}[caption=Zuweisung eines Validators einer Komponente]{Name}
<h:inputText id="createUserMail" value="#{userController.email}" required="true" requiredMessage="#{msg.forgotEmail}">
  <f:validator validatorId="EmailValidator"/>
</h:inputText>
\end{lstlisting}
Wie hier zu sehen ist, wird der Email-Validator einem Input-Textfeld zugewiesen. Das "f:" am Anfang des Tags ist das alias für jsf-core.



\section{Primefaces}
\label{JumpPrimefacesImplementation}
\subsection{FileUpload}
\label{subsec:FileUpload}
F�r das Hochladen der Bilder wird im Projekt die File-Upload Komponente von PrimeFaces verwendet. Dadurch konnte diese Anforderung relativ schnell umgesetzt werden. Zu Beginn traten allerdings einige Probleme auf. So ben�tigt diese Komponente eine Reihe von Biblotheken damit der Upload funktioniert. Dies wurde allerdings weder auf der PrimeFaces-Webseite noch in der PDF-Dokumentation erw�hnt. Letztendlich musste eine Recherche im Internet gemacht werden, wo dann in einem Forum erw�hnt wurde, dass die Bibliotheken "`commons-logging"', "`commons-io"' und "`commons-fileupload"' ben�tigt werden. Diese sind alle Projekte der Apache Foundation und somit trat bei der Verwendung auch keine lizenzrechtlichen Probleme auf.\\
\\
Grundvoraussetzung f�r die Verwendung der Fileupload-Komponente ist, dass einige zus�tzliche Filter in die web.xml eingetragen werden. Welche Filter einzubinden sind, findet man in der PDF-Dokumentation von Primefaces.
\begin{lstlisting}[caption=Fileupload-Filter, label=lstFielUploadFilter]
<filter>
        <filter-name>PrimeFaces FileUpload Filter</filter-name>
        <filter-class>
            org.primefaces.webapp.filter.FileUploadFilter
        </filter-class>
    </filter>
    <filter-mapping>
        <filter-name>PrimeFaces FileUpload Filter</filter-name>
        <servlet-name>Faces Servlet</servlet-name>
    </filter-mapping>
\end{lstlisting}
Ansonsten ist die Verwendung der Komponente ziemlich einfach. Als erstes muss die Komponente in eine HTML-Seite eingebunden werden.
\begin{lstlisting}[caption=Fileupload-Komponente einbinden, label=lstFileupload]
<p:fileUpload widgetVar="uploader" customUI="true" label="#{msg.uploadImage}" description="*.jpg;*.JPG" allowTypes="*.jpg;*.JPG" fileUploadListener="#{createEditAlbumController.handleFileUpload}" multiple="true" id="fileUploader" update="messages, cgrid"/>
\end{lstlisting}
Es ist sogar m�glich die Schaltfl�chen umzubenennen. Diese M�glichkeit wurde nat�rlich genutzt. Wie in dem Listing \autoref{lstFileupload} zu sehen ist, wurden die Beschriftungen allerdings nicht fest reingeschrieben sondern mit einem Eintrag in der Message.properties verkn�pft. Au�erdem wurden nur Dateien f�r den Upload zugelassen, die die Endung "`.jpg"' bzw. "`.JPG"' haben. Sobald man ein Bild hochl�dt, wird ein FileUploadEvent ausgel�st. Mit dem entsprechenden Event-Objekt kann man auf die Daten zugreifen. 
%\begin{lstlisting}[caption=FileUploadEvent]
%public void handleFileUpload(FileUploadEvent event) {
        %UploadedFile file = event.getFile();
        %try {
            %...
            %out = new FileOutputStream(uploadDir + username + "_" + file.getFileName());
            %out.write(file.getContents());
            %out.flush();
            %out.close();
            %....
    %}
%\end{lstlisting}
%TODO Referenz einf�gen
Die genaue Implementation des FileUploadEvents wird noch im Abschitt \autoref{subsubsec:CreateEditAlbumController} beschrieben.

\subsection{GMap}
Primefaces bietet noch weitere n�tzliche Komponenten. So wird bereits eine Komponente zum Anzeigen einer GMap-Karte bereitgestellt. Um diese einzubinden muss nur der in Listing \autoref{lstGMap} zu sehende Code in die Webseite eingebunden werden.
\begin{lstlisting}[caption=GMap-Tag, label=lstGMap]
<p:gmap center="#{albumController.bilder[0].latitude}, #{albumController.bilder[0].longitude}" zoom="13" type="HYBRID" style="width:600px;height:400px"
                    model="#{mapBean.simpleModel}" overlaySelectListener="#{mapBean.onMarkerSelect}">
</p:gmap>
\end{lstlisting}
Voraussetzung f�r die Verwendung des GMap-Tags ist allerdings der folgende JavaScript-Code, mit dem die GoogleMaps-API eingebunden wird.
\begin{lstlisting}[caption=Einbinden der GoogleMaps-API]
<script src="http://maps.google.com/maps/api/js?sensor=true" type="text/javascript"></script>
\end{lstlisting}
F�r das Anzeigen der Bilder in der GMap-Karte gibt es verschiedene M�glichkeiten. So wird bereits in dem Showcase auf der Primefaces-Webseite vorgeschlagen f�r jedes Bild ein Marker zu verwenden und bei einem Klick auf den Marker eine Sprechblase mit dem Bild anzuzeigen. Daf�r muss zwischen den �ffnenden und den schlie�enden GMap-Tag der Inhalt des Listings \autoref{lstInfoWindow} eingetragen werden.
\begin{lstlisting}[caption=GMap-InfoWindow, label=lstInfoWindow]
<p:gmapInfoWindow>
                        <p:graphicImage value="#{mapBean.image}" width="200" height="150"/>
                        <br/>
                        <h:outputText value="#{mapBean.beschreibung}"/>
                    </p:gmapInfoWindow>
\end{lstlisting}
Das Hinzuf�gen der Marker und was bei einem Klick auf einem Marker passiert, muss in der dazu geh�rigen ManagedBean erfolgen. Dies wird in Abschnitt ???????????? noch erl�utert.\\
\\
Eine andere M�glichkeit w�re, die Bilder direkt in der GMap anzuzeigen. In einem Wegwerfprojekt, das durchgef�hrt wurde um Primefaces genauer kennen zu lernen, wurde festgestellt, dass dies m�glich ist. Dazu m�ssten die Marker nicht mehr das Bild des Markers anzeigen sondern das gew�nschte Bild. Es wurde allerdings festgestellt, dass dies nur mit Bildern gehen w�rde, die im Web-Ordner des Tomcats zur Verf�gung stehen. Dazu m�sste der Pfad bekannt sein. Mit dynamisch gestreamten Bildern von der Festplatte war das Ersetzen des Marker-Bildes nicht m�glich. Au�erdem stellte sich bei diesem Versuch heraus, dass ein Anzeigen der Bilder direkt in der Map zu viel von der eigentlichen Karte verdeckt. Daher wird die erste M�glichkeit in diesem Projekt verwendet.

\subsection{Gallery}
Bei einem Fotoarchiv muss man nat�rlich auch die M�glichkeit haben sich die Fotos anzeigen zu lassen und dies nicht nur in einer kleinen Karte von Google anzuzeigen. Daf�r sind verschiedene M�glichkeiten von Slideshows bei Primefaces vorgesehen. Es wurden einige M�glichkeiten ausprobiert. So wurde z.B. versucht die Slideshow mit dem LightBox-Tag zu verwirklichen. Allerdings stellte sich heraus, dass diese Komponente nur Bilder statische Bilder, also Bilder mit einem festen Pfad im Web-Ordner, anzeigen kann. Da nur dynamisch gestreamte Bilder verwendet werden, war diese Komponente in diesem Projekt nicht zu verwenden. Als Alternative dazu gibt es die Gallery-Komponente, die seit Primefaces Version 2 verf�gbar ist. Daf�r muss in die Seite folgender Code eingebaut werden.
\begin{lstlisting}[caption=Gallery-Komponente, label=lstGallery]
 <p:galleria id="images" effect="fade" effectSpeed="1000">
...
<p:galleria>
\end{lstlisting}
Damit die Bilder dynamisch angezeigt werden, muss noch im Controller eine Methode implementiert werden. Dies wird in Abschnitt ???????????????????? beschrieben.

\subsection{Calendar}
F�r jedes Bild sollte m�glich sein, den Zeitstempel der Aufnahme manuell zu �ndern. Dies k�nnte durch eine manuelle Eingabe des Datums �ber ein Textfeld geschsehen. Da dabei aber unbedingt eine �berpr�fung der Eingabe erfolgen muss und weil diese L�sung nicht sehr professionell aussieht, wird eine vorgefertige Komponente von Primefaces daf�r verwendet.
\begin{lstlisting}[caption=Calender-Komponente, label=lstCalender]
<p:calendar value="#{bildController.timestamp}" pattern="dd.MM.yyyy" showOn="button"/>
\end{lstlisting}
Leider ist es bei Version 2 von Primefaces nicht m�glich die Uhrzeit mit einzugeben. Dieser Missstand wird aber in Version 3 von Primefaces behoben.

\subsection{DataGrid}
Sobald Bilder hochgeladen werden m�chte der Benutzer gw�hnlich eine �bersicht haben, welche Dateien bereits hochgeladen sind. Primefaces bietet unter anderem daf�r die DataGrid-Komponente an.
\begin{lstlisting}[caption=DataGrid-Komponente, label=lstDataGrid]
<p:dataGrid id="cgrid" var="b" value="#{createEditAlbumController.bilder}" columns="3" rows="12" paginator="false" effect="true" paginatorTemplate="{CurrentPageReport} {FirstPageLink} {PreviousPageLink} {PageLinks} {NextPageLink} {LastPageLink} {RowsPerPageDropdown}" rowsPerPageTemplate="9,12,15">
                <p:column>
                    <p:panel header="#{b.name}" style="text-align:center">
                        <h:panelGrid columns="1" style="width:100%">
....
</h:panelGrid>
                    </p:panel>
                </p:column>
            </p:dataGrid>
\end{lstlisting}
F�r jedes Bild muss innerhalb des DataGrid noch ein PanelGrid hinzugef�gt werden. Innerhalb des PanelGrids werden dann die Bilder mittel GraphicImage angezeigt. Damit immer die richtigen Bilder geladen werden, muss man im Controller noch eine entsprechende Methode bereitstellen. Dies wird in Abschnitt ??????????????????? auf Seite ????????????????????? erl�utert.

\subsection{Dialog}
An einigen Stellen ist es in einem Projekt immer notwendig, den Benutzer noch einmal gesondert auf einen bestimmten Punkt hinzuweisen. In diesem Projekt ist das beim L�schen eines Bildes notwendig. Dort soll dem Benutzer noch einmal eine Warnmeldung ausgegeben werden. Primefaces stellt daf�r die Dialog-Komponente zur Verf�gung.
\begin{lstlisting}[caption=Dialog, ,label=lstDialog]
<p:commandLink value="#{msg.deleteImage}" onclick="dialogDeleteBildInfo.show();"/>
	<p:dialog header="#{msg.deletePic}" widgetVar="dialogDeleteBildInfo" modal="true" height="200">
  ....
  </p:dialog>
\end{lstlisting}
Innerhalb des Dialog-Tags wird der Inhalt des Dialog-Fensters definiert. Der Dialog wird mittels eines JavaScript-Aufrufs angezeigt. Der Name des Dialog-Fensters wird im Dialog-Tag mittels des Attributs "`widgetVar"' festgelegt.
\section{Benutzeroberfl�che}
\label{JumpBenutzeroberflaeche}
%TODO Anpassen, Facelets rein
	\chapter{Tests}
\label{JumpTests}
%TODO Aufteilen in Unterabschnitte
\section{UserControllerTest}
Um nach �nderungen sicher zu sein das die Klasse noch richtig funktioniert m�ssen Tests durchgef�hrt werden. Aus diesem Grund wurden beim UserController die Methoden sendPasswortEmail, Login, Logout, erzeugeBenutzer, setPasswort und updateBenutzer getestet. Dabei soll vor allem �berpr�ft werden, ob auch wirklich im Controller die Attribute richtig gesetzt werden. So wird zum Beispiel in der Methode testSendPasswortEmail nicht �berpr�ft, ob eine EMail versendet wird, sondern ob ein neues zuf�lliges Passwort generiert wurde.\\
\\
Leider treten bei den meisten Tests Fehler auf. Das liegt haupts�chlich daran, dass aus einem Controller heraus Nachrichten auf der Webseite gesetzt werden sollen. Dazu wird der FacesContext verwendet. Dieser ist allerdings bei einem Test stets null. Daher sind wirklich sinnvolle Tests nicht mehr m�glich. Ein Framework was bei diesem Punkt Abhilfe schaffen k�nnte w�re JSFUnit. F�r eine Evaluierung dieses Frameworks war aber keine Zeit mehr sodass es in diesem Projekt nicht eingesetzt wird.

\section{ImageDataExtractorTest und CoordinateCalculatorTest}
F�r die Extrahierung der Geo-Daten aus den Bildern ist auch ein Test erstellt worden. Damit sollte sichergestellt werden, dass die GPS-Koordinaten richtig aus einem Bild geholt werden. Die Erzeugung eines Thumbnails aus dem Bild mit dem Framework "`Metadata Extractor"' wurde ebenfalls �berpr�ft. Dazu ist auf dem Server auch ein Testbild mit Geoinformationen bereitgestellt worden, so dass der Build-Server die Test ordentlich durchf�hren kann.\\
\\
Der CoordinateCalculatorTest �berpr�ft ob die Umrechnung der Koordinaten von z.B. N 51''8'55.8, E 14''59'55.985 in das von Google gew�nschte Dezimalformat\\ (z.B.: 51.148833333333336, 14.998884722222222) richtig arbeitet.\\
\\
\section{GPicSUtilTest und PasswortUtilTest}
In der Klasse GPicSUtil werden einige Hilfsmethoden bereitgestellt, die an mehreren Stellen bei den Controllern ben�tigt werden. Es wurde hier besonders die Methode getStreamedContent �berpr�ft. Diese soll von ein Bild von der Festplatte holen und es als StreamedContent zur�ckgeben. Nach einem Aufruf der Methode darf also das zur�ckgegebene StreamedContent-Objekt nicht null sein. Sollte dies nicht der Fall sein, ist der Test erfolgreich durchgelaufen.\\
\\
Die Klasse PasswortUtil stellt zwei Methoden zur Verf�gung. Eine soll ein Passwort mit MD5 verschl�sseln und die zweite ein zuf�lliges Passwort erzeugen. F�r den Verschl�sselungstest wurde ein Wort ausgew�hlt, wo der MD5-Hash schon bekannt ist. Dann wurde dieses Wort an die Methode �bergeben und der R�ckgabewert mit dem bekannten MD5-Hash verglichen. Der Test ist erfolgreich, wenn die beiden Werte gleich sind. F�r den zweiten Test wird die Methode zum Erzeugen von Zufallspassw�rtern zweimal aufgerufen und anschlie�end die beiden R�ckgabewerte miteinander verglichen. Diese beiden Werte d�rfen nicht gleich sein. Sollte dies dennoch der Fall sein schl�gt der Test fehl.
	\chapter{Aufteilung des Projekts}
\label{JumpProjektaufteilung}	
	\chapter{Zusammenfassung}
\label{JumpZusammenfassung}
Basierend auf der Aufgabenstellung wurde das existierende Content Management System Drupal untersucht. Das Ergebnis war, dass nicht alle gestellten Anforderungen damit umgesetzt werden konnten und somit die Erstellung eines neuen Systemes von N�ten war. Die formulierten W�nsche an das System wurden in Pflicht- und Zusatzaufgaben eingeteilt. Nach der Erstellung des Systementwurfes, welcher die geforderten XML-Technologien und Frameworks ber�cksichtigte, wurde eine Evaluierung von neuen Technologien durchgef�hrt. Implementiert wurde das Projekt mit Hilfe der IDE IntelliJ. In diesem Zusammenhang wurde eine Evaluierung der Entwicklungsumgebung vollzogen. Als Ergebnis steht ein Software-Projekt, welches die Pflichtanforderungen erf�llt. Gleichzeitig wurden die verwendeten Technologien dahingehend untersucht, ob sie sich zu der Entwicklung einer solchen Anwendung eignen und welche St�rken und Schw�chen sie besitzen.\newline\newline
Die Verwendung einer nativen XML-Datenbank in Verbindung mit dem Austausch von Daten �ber XML-Dateien, eignet sich hervorragend um Webprojekte zu erstellen. Unterst�tzt durch die Entwicklungsumgebung, die dem Entwickler den Freiraum f�r Kreativit�t l�sst, da er die Umgebung individuell an seine Bed�rfnisse anpassen kann. Diese Freiheit hat einen Preis, da Lizenzkosten anfallen. Die Integration von Primefaces-Komponenten bietet eine einfache M�glichkeit komplexe Anforderungen, wie eine Integration von personalisierten GoogleMaps-Karten, umzusetzen. Allerdings ist die Dokumentation an einigen Stellen l�ckenhaft und das Verhalten nicht immer so wie erwartet. Trotz dieser M�ngel eignet sich Primefaces f�r die Entwicklung von Webanwendungen, da fr�here Projekte gezeigt haben, dass andere Frameworks dieser Art, wie RichFaces oder ICEFaces, �hnliche Schw�chen gezeigt haben.
%	\input{Fazit}
	\bibliographystyle{amsalpha}
\bibliography{literatur}
\nocite{*}
	\chapter{Anhang}
	\section{CD-Inhalt}
\label{secCDInhalt}
Auf der beiliegenden CD befinden sich folgende \textit{Dateien} und \textbf{Ordner}:
\begin{itemize}
	\item \textbf{Beleg}
	\begin{itemize}
		\item \textit{Beleg.pdf} - dieses Dokument im PDF-Format
		\item \textbf{src} - enth�lt alle LaTeX-Quelldateien f�r die Erstellung des Belegs
		\begin{itemize}
			\item \textbf{img} - enth�lt alle im Beleg eingebundenen Grafiken
		\end{itemize}
	\end{itemize}
\end{itemize}

\section{UML-Diagramme}
\label{secAnhang}
%	\label{pageValidationCall}
%	\includepdf[landscape, turn=true, pages=-, fitpaper=true]{img/Validation_Call.pdf}
	
	\newpage
	\input{Eidesstattliche-Erkl�rung}
\end{document}