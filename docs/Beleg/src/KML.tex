\section{KML}
\label{JumpKML}
Die Keyhole Markup Language, kurz KML,  ist eine Auszeichnungssprache, die von Google.Inc entwickelt wurde um Geodaten f�r die Client-Komponenten der Programme Google Earth und Google Maps zu beschreiben. KML ist ein Standard des Open Geospatial Consortium und aktuell in der Version 2.2 verf�gbar.
\\
Es gibt zwei Formate in der ein KML-File vorliegen kann. Zum einen das .kml-Format und das .kmz-Format. Bei dem KMZ-Format handelt es sich um die gezippte Version eines oder mehrerer KML-Files. 
\\
�ber den Inhalt und Aufbau eines KML-Dokumentes sollte man folgendes wissen. Es kann Geodaten im Vektor- und Rasterformat enthalten, wobei Vektorobjekte wie Punkte, Linien, lineare Ringe, Polygone oder COLLADA-Modelle als Placemark-Elemente und Luft- und Satellitenbilder als GroundOverlay-Elemente modelliert werden. Placemark-Elemente k�nnen zus�tzlich zu den Geodaten auch noch andere Daten wie Name, Beschreibung, Betrachtungswinkel und -h�he, sowie einen Zeitstempel und noch vieles mehr enthalten. Hier ein Beispiel eines Placemark-Elementes eines KML-Dokuments:
\begin{lstlisting}[caption=KML-Dokument]{Name}
<?xml version="1.0" encoding="UTF-8"?>
<kml xmlns="http://www.opengis.net/kml/2.2">
<Document>
  <Placemark>
    <name>G�rlitz</name>
    <description>Hochschule Zittau/G�rlitz</description>
    <Point>
      <coordinates>51.149044,14.997729</coordinates>
    </Point>
  </Placemark>
</Document>
</kml>
\end{lstlisting}
Der Unterschied bei GroundOverlay-Elementen ist der, dass dort ein Koordinatenausschnitt zur Georeferenzierung der Rasterdaten angegeben werden muss, statt der Geometrie.
\\
Hat man ein solches KML-File erstellt bzw. vorliegen und m�chte dieses Ver�ffentlichen, so gibt es daf�r verschiedene Varianten. Um ein KML-File mit dem Programm Google Earth zu nutzen reicht es aus, dass sich die KML- oder KMZ-datei irgendwo auf dem Rechner befindet, mit dem man die Applikation nutzt. Bei Google Maps ist die Sache schon etwas komplizierter. Dort gibt es nur die M�glichkeit, sein KML- bzw. KMZ-File auf einen Webserver oder den von Google daf�r vorgesehenen Server zu laden. Damit es dann aber auch mit Google Maps genutzt werden kann, muss auf der Seite "'http://www.google.de/addurl/?continue=/addurl"' ein Formular ausgef�llt werden, wo man die URL auf der sich das Dokument befindet angibt. Seit neuestem ist Google Maps auch in der Lage das KML-Dokument zu nutzen wenn man einfach in das Google Maps Suchfeld die URL eingibt, wo sich das KML- bzw. KMZ-File befindet.
\\
KML-Files werden in diesem Projekt nicht genutzt werden, da sich die gleichen Dinge auch mit der Primeface Komponente GMap umsetzen lassen.
So brauchen wir nicht immer neue KML-Files erstellen bzw. �ltere �ndern, sobald ein neues Bild hochgeladen oder die Sichtbarkeit ge�ndert wird.
